\section{Postenlauf:Innovation im Bauwesen}
\begin{frame}{Postenlauf: Innovation im Bauwesen}
    ca. 90 min Länge, Bearbeitung in der Lektion (keine SLE resp. Hausaufgabe)
    \begin{itemize}
        \item [\textbullet] Wähle 3 der 5 Texte aus. Die Texte sind auf Teams in den Ordner \textit{Innovation im Bauwesen} hochgeladen. Arbeitsform: Einzelarbeit.
        \begin{itemize}
            \item [\textbullet] Erstelle zu einem Text ein Mindmap. 
            \item [\textbullet] Erstelle zu einem Text eine schriftliche Zusammenfassung von ca. 150 Wörtern.
            \item [\textbullet] Erstelle zu einem Text mindestens 5 Kontrollfragen (inkl. Lösung) für deinen Tischnachbarn.
        \end{itemize}
        \item [\textbullet] Die Dokumente müssen in Teams hochgeladen  werden in den jeweiligen Aufgaben und werden mit bestanden / nicht bestanden bewertet.
    \end{itemize}
    \end{frame}

\begin{frame}{Übergeordnete Lernziele}
    Die übergeordnete Lernziele für diese Unterrichtseinheit sind: 
    \begin{itemize}
        \item Bauführer und Bauführerinnen informieren sich über neue Methoden und Technologien und den Einsatz von
        multifunktionalen und intelligenten Baustoffen in ihrem Arbeitsbereich.
        \begin{itemize}
            \item Sie informieren sich aus Fachpresse und Messen über Innovationen. \textbf{(K2)}
            \item Sie betreiben ein firmeninternes Wissensmanagement zukunftsorientiert. \textbf{(K4)}
            \item Sie erarbeiten Dokumentationen zur Einführung von kreislauffähigen Materialien und Baumethoden
            in ihrem Bereich. \textbf{(K4)}
        \end{itemize}
        \item Bauführer und Bauführerinnen leiten den fachgerechten und vorschriftsmässigen Einsatz neuer Methoden, Technologien und Baustoffe.
        \begin{itemize}
            \item Sie wenden neue Methoden, Technologien und Baustoffe bei Bauarbeiten an. \textbf{(K3)}
            \item Sie instruieren die Mitarbeitenden in neuen Bauabläufen. \textbf{(K3)}
            \item Sie führen Evaluationen zum Einsatz von neuen Baustoffen durch. \textbf{(K4)}
        \end{itemize}
    \end{itemize}
\end{frame}
