\section{Postenlauf: Innovation im Bauwesen: Block I}
\BlueSectionSlide

\begin{frame}{Postenlauf: Innovation im Bauwesen: Block I}
    ca. 75 min Länge, Bearbeitung in der Lektion (keine SLE resp. Hausaufgabe)
    \begin{itemize}
        \item [\textbullet] Wähle 3 der 5 Texte aus. Die Texte sind auf Teams in den Ordner \textit{30\_Innovation im Bauwesen} hochgeladen. Arbeitsform: Einzelarbeit.
        \begin{itemize}
            \item [\textbullet] Erstelle zu einem Text ein Mindmap. 
            \item [\textbullet] Erstelle zu einem Text eine schriftliche Zusammenfassung von ca. 150 Wörtern.
            \item [\textbullet] Erstelle zu einem Text mindestens 5 Kontrollfragen (inkl. Lösung) für deinen Tischnachbarn.
        \end{itemize}
        \item [\textbullet] Die Dokumente müssen in Teams hochgeladen  werden in den jeweiligen Aufgaben und werden mit bestanden / nicht bestanden bewertet.
    \end{itemize}
    \end{frame}


\begin{frame}{Texte}
    \begin{itemize}
        \item [\textbullet] Alternativen zur Bauwerkserhaltung
        \item [\textbullet] Asphalt geht auch kalt
        \item [\textbullet] Autonomer Bagger baut Trockensteinmauer
        \item [\textbullet] CO$_2$-neutraler Beton mit Pflanzenkohle-Pellets
        \item [\textbullet] Schraubfundamente -- Belastbare Ergebnisse im Handumdrehen
    \end{itemize}

\end{frame}



\begin{frame}{Übergeordnete Lernziele (1/2)}
    Die übergeordnete Lernziele für diese Unterrichtseinheit sind: 
    \begin{itemize}
        \item[\textbullet]  Bauführer und Bauführerinnen informieren sich über neue Methoden und Technologien und den Einsatz von
        multifunktionalen und intelligenten Baustoffen in ihrem Arbeitsbereich.
        \begin{itemize}
            \item [\textbullet]  Sie informieren sich aus Fachpresse und Messen über Innovationen. \textbf{(K2)}
            \item [\textbullet] Sie betreiben ein firmeninternes Wissensmanagement zukunftsorientiert. \textbf{(K4)}
            \item [\textbullet] Sie erarbeiten Dokumentationen zur Einführung von kreislauffähigen Materialien und Baumethoden
            in ihrem Bereich. \textbf{(K4)}
        \end{itemize}
    \end{itemize}
\end{frame}

\begin{frame}{Übergeordnete Lernziele (2/2)}
    Die übergeordnete Lernziele für diese Unterrichtseinheit sind: 
    \begin{itemize}
        \item [\textbullet] Bauführer und Bauführerinnen leiten den fachgerechten und vorschriftsmässigen Einsatz neuer Methoden, Technologien und Baustoffe.
        \begin{itemize}
            \item[\textbullet]  Sie wenden neue Methoden, Technologien und Baustoffe bei Bauarbeiten an. \textbf{(K3)}
            \item [\textbullet] Sie instruieren die Mitarbeitenden in neuen Bauabläufen. \textbf{(K3)}
            \item [\textbullet] Sie führen Evaluationen zum Einsatz von neuen Baustoffen durch. \textbf{(K4)}
        \end{itemize}
    \end{itemize}
\end{frame}
