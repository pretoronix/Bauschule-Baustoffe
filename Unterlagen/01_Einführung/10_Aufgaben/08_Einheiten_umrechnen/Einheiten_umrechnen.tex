% !TEX root = /Users/patricpf/Documents/repos/Bauschule-Baustoffe/Unterlagen/01_Einführung/10_Aufgaben/08_Einheiten_umrechnen/Einheiten_umrechnen.tex
% !TEX program = lualatex
\documentclass[
    %answers,
    a4paper,ngerman,12pt, addpoints]{exam}

\usepackage[utf8]{inputenc}
%\usepackage[T1]{fontenc}
%\usepackage[ngerman]{babel}


\usepackage{polyglossia}
\setdefaultlanguage[variant = swiss]{german}
\usepackage{fontspec}
\setmainfont{Aptos} % Bauschule CI Manual
\setsansfont{Aptos} % Bauschule CI Manual


\usepackage[ a4paper,
 total={165mm,250mm},
 left=25mm,
 top=25mm,
 headsep=10mm
 %footsep=12mm
 %,showframe
  ]{geometry}

\usepackage{graphicx}
\usepackage{siunitx}
\usepackage{booktabs} % schöne Tabellen
\usepackage{float}
\floatplacement{figure}{H}
\usepackage{xcolor}
\usepackage{pdfpages}
\usepackage{enumitem}
\usepackage{mdframed} % Boxen
\usepackage{amsmath,amssymb}
\usepackage{tcolorbox}
\usepackage{lastpage} % For the total number of pages
\usepackage{gensymb}
\usepackage{xspace}
\usepackage{tabularx}
\usepackage{multicol}
\usepackage[
    version=3,
    arrows=pgf-filled,
]{mhchem} % für chemische Formeln
%\usepackage{microtype}
\usepackage{subfigure}
\usepackage[hidelinks]{hyperref}
\usepackage{cleveref}
\usepackage{luacode}
\usepackage{amsmath}
\usepackage{textcomp}


\sisetup{
  locale = DE,
  inter-unit-product = \ensuremath{{\cdot}},
  detect-all,
}

% Colors
\definecolor{blau_bauschule}{RGB}{22,65,148}
\CorrectChoiceEmphasis{\color{blau_bauschule}}
\SolutionEmphasis{\color{blau_bauschule}}

\setlength{\parindent}{0em} % Verhindert einrücken
\setlength\linefillheight{0.3in}


%% COMMMANDS
\author{Patrick Pfändler}
\newcommand{\dozent}{Patrick Pfändler}
\newcommand{\fach}{Baustoffe}


\newcommand{\punkte}[1]{%
    \begin{infobox}%
        #1
    \end{infobox}}%
\newcommand{\FinRes}[1]{\underline{\underline{#1}}}

\newmdenv[linecolor=black,backgroundcolor=gray!15,frametitle={Punktverteilung},leftmargin=1cm,rightmargin=1cm]{infobox}

\newcommand{\pagebreaksol}{
    \ifprintanswers
        \clearpage
    \else
        {}
    \fi
}

\newcommand{\pagebreakexam}{
    \ifprintanswers
        {}
    \else
        \clearpage
    \fi
}

\SolutionEmphasis{\color{blau_bauschule}}
\makeatletter%
\newcommand{\solutiontable}[1]{\ifprintanswers\begingroup\Solution@Emphasis#1\if@shadedsolutions%
            {\cellcolor{SolutionColor}}%
        \else%
        \fi\endgroup\else\phantom{#1}\fi}%
\makeatother%

\newcommand{\myNmm}[1]
{
    \sisetup{per-mode=symbol}
    \SI{#1}{\newton\per\mm\squared}
}

\renewcommand{\thequestion}{\fontsize{12pt}{2pt} \selectfont  \bfseries \arabic{question}}
\sisetup{per-mode=symbol}



%% Translation

\pointpoints{Punkt}{Punkte}
\bonuspointpoints{Bonuspunkt}{Bonuspunkte}
\renewcommand{\solutiontitle}{\noindent\textbf{Lösung:}\enspace}
\chqword{Frage}
\chpgword{Seite}
\chpword{Punkte}
\chbpword{Bonus Punkte}
\chsword{Erreicht}
\chtword{Gesamt}
\hpword{Punkte:}
\hsword{Ergebnis:}
\hqword{Aufgabe:}
\htword{Summe:}


\renewcommand{\questionshook}{%
  %\setlength{\leftmargin}{0pt}% removes the indentation from the left
  \setlength{\labelwidth}{1.25cm}% adjusts label width
  \setlength{\itemindent}{0cm}% aligns the start of the item with the above
  \setlength{\labelsep}{0.25cm}% space between the label and the item text
}




%% header and footer
\pagestyle{headandfoot}
\firstpageheadrule
\runningheadrule

% Adjust the font size for the header
\firstpageheader{\fontsize{9}{11}\selectfont\fach}{}{\fontsize{9}{11}\selectfont\dozent \\ \blattname}
\runningheader{\fontsize{9}{11}\selectfont\fach}{}{\fontsize{9}{11}\selectfont\dozent \\ \blattname}

% Adjust the font size for the footer
\firstpagefooter{\includegraphics[width=2.5cm]{bauschule-logo-5cm.png}}{}{\fontsize{9}{11}\selectfont\thepage\,/\,\pageref{LastPage}}
\runningfooter{\includegraphics[width=2.5cm]{bauschule-logo-5cm.png}}{}{\fontsize{9}{11}\selectfont\thepage\,/\,\pageref{LastPage}}



%% header and footer

\pagestyle{headandfoot}
\firstpageheadrule
\runningheadrule
\firstpageheader{\fach}{}{\fontsize{9pt}{2pt}\selectfont \dozent \\ \blattname}
\runningheader{\fach}{}{\fontsize{9pt}{2pt}\selectfont\dozent \\ \blattname}
\firstpagefooter{\includegraphics[width=2.5cm]{../../../../template/bauschule-logo-5cm.png}}{}{\fontsize{9pt}{2pt}\selectfont \thepage\,/\,\numpages}
\runningfooter{\includegraphics[width=2.5cm]{../../../../template/bauschule-logo-5cm.png}}{}{\fontsize{9pt}{2pt}\selectfont \thepage\,/\,\numpages}



\newcommand{\blattname}{Einheiten umrechnen}



\begin{document}

%{\fontsize{22pt}{2pt}\selectfont \textbf{\blattname}}
\section*{\blattname}
\vspace{1cm}

\punkte{Alle möglichen Nachkommastellen beim Resultat angeben.
Für die Lösung wurde gerechnet mit \SI{10}{\metre\per\second\squared}.
}
\vspace{0.1cm}
Hinweis: Nehmen Sie an, dass wir uns auf der Erde befinden.
\vspace{0.5cm}

\begin{questions}
\question[1]
Rechnen Sie \SI{6.9908}{\N} in \si{kg} um.
%\fillwithgrid{3cm}
%\vspace*{0.1cm}

\begin{solutionorbox}[3cm]
\SI{699080}{\kilogram}
\end{solutionorbox}



\question[1]
Rechnen Sie \SI{210656.3}{\N} in \si{\mega\N} um.
%\fillwithgrid{3cm}
\vspace*{0.1cm}

\begin{solutionorbox}[3cm]
\SI{0.2106563}{\mega\N}
\end{solutionorbox}



\question[1]
Rechnen Sie \SI{258.74}{\kilogram} in t um.
%\fillwithgrid{3cm}
\vspace*{0.1cm}

\begin{solutionorbox}[3cm]
0.25874 t
\end{solutionorbox}


\question[1]
Rechnen Sie \SI{0.004536}{\mega\N} in \si{kg} um.
%\fillwithgrid{3cm}
\vspace*{0.1cm}

\begin{solutionorbox}[3cm]
\SI{453.6}{\kilogram}
\end{solutionorbox}

\question[1]
Rechnen Sie \SI{142.566}{\m} in \si{\mm} um.
%\fillwithgrid{3cm}
\vspace*{0.1cm}

\begin{solutionorbox}[3cm]
\SI{142566}{\mm}
\end{solutionorbox}

\ifprintanswers
\clearpage
\else
{}
\fi

\question[1]
Rechnen Sie $327\cdot 10^4$ N in \si{\mega\N} um.
%\fillwithgrid{3cm}
\vspace*{0.1cm}

\begin{solutionorbox}[3cm]
\SI{3.27}{\mega\N}
\end{solutionorbox}



\question[5]
Addieren Sie: 221 hl + 470 dl + 7.969 hl + $346 \cdot 10^4$ cl + 0.984 $m^3$ und geben Sie das Resultat in Litern an.
%\fillwithgrid{3cm}
\vspace*{0.1cm}

\begin{solutionorbox}[3cm]
\SI{58527.9}{\litre}
\end{solutionorbox}



\question[1]
Rechnen Sie $145\cdot 10^4$ kg in \si{\mega\N} um.
%\fillwithgrid{3cm}
\vspace*{0.1cm}

\begin{solutionorbox}[3cm]
\SI{14.5}{\mega\N}
\end{solutionorbox}



\question[2]
Rechnen Sie \num{238832} $m^2$ in ha. (Hinweis: 1 ha = $10^4 m^2$)
%\fillwithgrid{3cm}
\vspace*{0.1cm}

\begin{solutionorbox}[3cm]
\num{23.8832}ha
\end{solutionorbox}


\question[1]
Rechnen Sie $7.84\cdot 10^5$ kg in \si{\mega\N} um.
%\fillwithgrid{3cm}
\vspace*{0.1cm}

\begin{solutionorbox}[3cm]
\SI{0.784}{\mega\N}
\end{solutionorbox}


\question[1]
Rechnen Sie \SI{0.1099}{\kilo\N} in \si{\kilogram} um.
%\fillwithgrid{3cm}
\vspace*{0.1cm}

\begin{solutionorbox}[3cm]
\SI{10.99}{\kg}
\end{solutionorbox}



\question[4]
Berechnen Sie:  $76400 cm^3 + 6453 l -87\cdot 10^3 dm^3 + 12222 cm^3-0.87543 m3 - 0.5401 m^3$ und geben Sie das Resultat in der Einheit Liter an.
%\fillwithgrid{3cm}
\vspace*{0.1cm}

\begin{solutionorbox}[3cm]
\SI{-81873.908}{\litre}
\end{solutionorbox}

%\ifprintanswers
%\clearpage
%\else
%{}
%\fi

\question[4]
Berechnen Sie:  $1904 kW -5.7 MW + 0.426 kW + 269945 mW  + 675 \cdot 10^3 W $ und geben Sie das Resultat in der Einheit Watt an.
%\fillwithgrid{3cm}
\vspace*{0.1cm}

\begin{solutionorbox}[3cm]
\SI{-3120304.055}{\watt}
\end{solutionorbox}

\end{questions}

%\vspace*{\stretch{1}}
%\begin{center}
%\gradetable[v][questions]
%\end{center}
\end{document}