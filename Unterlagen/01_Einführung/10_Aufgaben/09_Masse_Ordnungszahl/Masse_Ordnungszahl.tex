% !TEX program = lualatex
% !TEX root =/Users/patricpf/Documents/repos/Bauschule-Baustoffe/Unterlagen/01_Einführung/10_Aufgaben/09_Masse_Ordnungszahl/Masse_Ordnungszahl.tex
% !TEX program = lualatex
\documentclass[
    %answers,
    a4paper,ngerman,12pt, addpoints]{exam}

\usepackage[utf8]{inputenc}
%\usepackage[T1]{fontenc}
%\usepackage[ngerman]{babel}


\usepackage{polyglossia}
\setdefaultlanguage[variant = swiss]{german}
\usepackage{fontspec}
\setmainfont{Aptos} % Bauschule CI Manual
\setsansfont{Aptos} % Bauschule CI Manual


\usepackage[ a4paper,
 total={165mm,250mm},
 left=25mm,
 top=25mm,
 headsep=10mm
 %footsep=12mm
 %,showframe
  ]{geometry}

\usepackage{graphicx}
\usepackage{siunitx}
\usepackage{booktabs} % schöne Tabellen
\usepackage{float}
\floatplacement{figure}{H}
\usepackage{xcolor}
\usepackage{pdfpages}
\usepackage{enumitem}
\usepackage{mdframed} % Boxen
\usepackage{amsmath,amssymb}
\usepackage{tcolorbox}
\usepackage{lastpage} % For the total number of pages
\usepackage{gensymb}
\usepackage{xspace}
\usepackage{tabularx}
\usepackage{multicol}
\usepackage[
    version=3,
    arrows=pgf-filled,
]{mhchem} % für chemische Formeln
%\usepackage{microtype}
\usepackage{subfigure}
\usepackage[hidelinks]{hyperref}
\usepackage{cleveref}
\usepackage{luacode}
\usepackage{amsmath}
\usepackage{textcomp}


\sisetup{
  locale = DE,
  inter-unit-product = \ensuremath{{\cdot}},
  detect-all,
}

% Colors
\definecolor{blau_bauschule}{RGB}{22,65,148}
\CorrectChoiceEmphasis{\color{blau_bauschule}}
\SolutionEmphasis{\color{blau_bauschule}}

\setlength{\parindent}{0em} % Verhindert einrücken
\setlength\linefillheight{0.3in}


%% COMMMANDS
\author{Patrick Pfändler}
\newcommand{\dozent}{Patrick Pfändler}
\newcommand{\fach}{Baustoffe}


\newcommand{\punkte}[1]{%
    \begin{infobox}%
        #1
    \end{infobox}}%
\newcommand{\FinRes}[1]{\underline{\underline{#1}}}

\newmdenv[linecolor=black,backgroundcolor=gray!15,frametitle={Punktverteilung},leftmargin=1cm,rightmargin=1cm]{infobox}

\newcommand{\pagebreaksol}{
    \ifprintanswers
        \clearpage
    \else
        {}
    \fi
}

\newcommand{\pagebreakexam}{
    \ifprintanswers
        {}
    \else
        \clearpage
    \fi
}

\SolutionEmphasis{\color{blau_bauschule}}
\makeatletter%
\newcommand{\solutiontable}[1]{\ifprintanswers\begingroup\Solution@Emphasis#1\if@shadedsolutions%
            {\cellcolor{SolutionColor}}%
        \else%
        \fi\endgroup\else\phantom{#1}\fi}%
\makeatother%

\newcommand{\myNmm}[1]
{
    \sisetup{per-mode=symbol}
    \SI{#1}{\newton\per\mm\squared}
}

\renewcommand{\thequestion}{\fontsize{12pt}{2pt} \selectfont  \bfseries \arabic{question}}
\sisetup{per-mode=symbol}



%% Translation

\pointpoints{Punkt}{Punkte}
\bonuspointpoints{Bonuspunkt}{Bonuspunkte}
\renewcommand{\solutiontitle}{\noindent\textbf{Lösung:}\enspace}
\chqword{Frage}
\chpgword{Seite}
\chpword{Punkte}
\chbpword{Bonus Punkte}
\chsword{Erreicht}
\chtword{Gesamt}
\hpword{Punkte:}
\hsword{Ergebnis:}
\hqword{Aufgabe:}
\htword{Summe:}


\renewcommand{\questionshook}{%
  %\setlength{\leftmargin}{0pt}% removes the indentation from the left
  \setlength{\labelwidth}{1.25cm}% adjusts label width
  \setlength{\itemindent}{0cm}% aligns the start of the item with the above
  \setlength{\labelsep}{0.25cm}% space between the label and the item text
}




%% header and footer
\pagestyle{headandfoot}
\firstpageheadrule
\runningheadrule

% Adjust the font size for the header
\firstpageheader{\fontsize{9}{11}\selectfont\fach}{}{\fontsize{9}{11}\selectfont\dozent \\ \blattname}
\runningheader{\fontsize{9}{11}\selectfont\fach}{}{\fontsize{9}{11}\selectfont\dozent \\ \blattname}

% Adjust the font size for the footer
\firstpagefooter{\includegraphics[width=2.5cm]{bauschule-logo-5cm.png}}{}{\fontsize{9}{11}\selectfont\thepage\,/\,\pageref{LastPage}}
\runningfooter{\includegraphics[width=2.5cm]{bauschule-logo-5cm.png}}{}{\fontsize{9}{11}\selectfont\thepage\,/\,\pageref{LastPage}}

\newcommand{\blattname}{Chemie: Masse und Ordnungszahl}
\newcommand{\fortyptfont}{\fontsize{35pt}{40pt}\selectfont}

\printanswers

%% header and footer

\pagestyle{headandfoot}
\firstpageheadrule
\runningheadrule
\firstpageheader{\fach}{}{\fontsize{9pt}{2pt}\selectfont \dozent \\ \blattname}
\runningheader{\fach}{}{\fontsize{9pt}{2pt}\selectfont\dozent \\ \blattname}
\firstpagefooter{\includegraphics[width=2.5cm]{../../../../template/bauschule-logo-5cm.png}}{}{\fontsize{9pt}{2pt}\selectfont \thepage\,/\,\numpages}
\runningfooter{\includegraphics[width=2.5cm]{../../../../template/bauschule-logo-5cm.png}}{}{\fontsize{9pt}{2pt}\selectfont \thepage\,/\,\numpages}


\begin{document}

{\fontsize{22pt}{2pt}\selectfont \textbf{\blattname}}
\vspace{0.3cm}

	
	\subsection*{Theorie}
	
	\begin{equation*}
		\text{Atom}_{\text{Ordnungszahl}}^{\text{Massenzahl}}
	\end{equation*}
	
	\begin{equation*}
		\text{Massenzahl} = \text{Anzahl der Protonen} + \text{Anzahl der Neutronen}
	\end{equation*}
	
	\begin{equation*}
		\text{Ordnungszahl} = \text{Anzahl der Protonen} 
	\end{equation*}
	
	\begin{itemize}%
		\item Die Ordnungszahl bestimmt das chemische Element. Sie gibt auch die Stellung eines chemischen Elements im Periodensystem der Elemente an.
		\item Im elektrisch neutralen Atom ist die Ordnungszahl auch gleich der Anzahl der Elektronen.
		\item \textcolor{green}{\textit{Isotope: }}Elemente können verschieden viele Neutronen enthalten und werden Isotope genannt.
		\item \textcolor{green}{\textit{Ionen:}} Stimmt bei einem Atom die Anzahl der Elektronen in der Hülle mit der Anzahl der Protonen im Kern nicht überein, so ist das Atom nach aussen hin nicht elektrisch neutral. Solche Teilchen mit einem Überhang an negativen oder positiven Ladungsträgern (Elektronen) nennt man Ionen.
	\end{itemize}
	
	
	\subsection*{Aufgaben}
	Ergänzen Sie die folgende Tabelle. 
	
	Zeit: ca. 7 min
	
	\ifprintanswers
	\begin{table}[H]
		\centering
		\begin{tabular}{|l|c|c|c|c|c|}
			\hline            & $\mathrm{Na}_{11}^{23}$ & $\mathrm{Cl}_{17}^{35}$ & $\mathrm{Ca}_{20}^{40}$ & $\mathrm{C}_{6}^{12}$ & $\mathrm{C}_{6}^{14}$ \\
			\hline Protonen   & 11                      & 17                      & 20                      & 6                     & 6                     \\
			\hline Neutronen  & 12                      & 18                      & 20                      & 6                     & 8                     \\
			\hline Elektronen & 11                      & 17                      & 20                      & 6                     & 6                     \\
			\hline
		\end{tabular}
	\end{table}
	\else
	
	\begin{table}[H]
	\fortyptfont
		\centering
		\begin{tabular}{|l|c|c|c|c|c|}
			\hline            & $\mathrm{Na}_{11}^{23}$ & $\mathrm{Cl}^{35}$ & $\mathrm{Ca}$ & C & $?^{14}$ \\
			\hline Protonen   &                         & 17                 & 20            & 6 &          \\
			\hline Neutronen  &                         &                    & 20            & 6 &          \\
			\hline Elektronen &                         &                    &               &   & 6        \\
			\hline
		\end{tabular}
	\end{table}
	\fi
	
	
	
	
\end{document}
