% !TEX root = /Users/patricpf/Documents/repos/Bauschule-Baustoffe/Unterlagen/01_Einführung/10_Aufgaben/06_UWert/U-Wert.tex
% !TEX program = lualatex
\documentclass[
    %answers,
    a4paper,ngerman,12pt, addpoints]{exam}

\usepackage[utf8]{inputenc}
%\usepackage[T1]{fontenc}
%\usepackage[ngerman]{babel}


\usepackage{polyglossia}
\setdefaultlanguage[variant = swiss]{german}
\usepackage{fontspec}
\setmainfont{Aptos} % Bauschule CI Manual
\setsansfont{Aptos} % Bauschule CI Manual


\usepackage[ a4paper,
 total={165mm,250mm},
 left=25mm,
 top=25mm,
 headsep=10mm
 %footsep=12mm
 %,showframe
  ]{geometry}

\usepackage{graphicx}
\usepackage{siunitx}
\usepackage{booktabs} % schöne Tabellen
\usepackage{float}
\floatplacement{figure}{H}
\usepackage{xcolor}
\usepackage{pdfpages}
\usepackage{enumitem}
\usepackage{mdframed} % Boxen
\usepackage{amsmath,amssymb}
\usepackage{tcolorbox}
\usepackage{lastpage} % For the total number of pages
\usepackage{gensymb}
\usepackage{xspace}
\usepackage{tabularx}
\usepackage{multicol}
\usepackage[
    version=3,
    arrows=pgf-filled,
]{mhchem} % für chemische Formeln
%\usepackage{microtype}
\usepackage{subfigure}
\usepackage[hidelinks]{hyperref}
\usepackage{cleveref}
\usepackage{luacode}
\usepackage{amsmath}
\usepackage{textcomp}


\sisetup{
  locale = DE,
  inter-unit-product = \ensuremath{{\cdot}},
  detect-all,
}

% Colors
\definecolor{blau_bauschule}{RGB}{22,65,148}
\CorrectChoiceEmphasis{\color{blau_bauschule}}
\SolutionEmphasis{\color{blau_bauschule}}

\setlength{\parindent}{0em} % Verhindert einrücken
\setlength\linefillheight{0.3in}


%% COMMMANDS
\author{Patrick Pfändler}
\newcommand{\dozent}{Patrick Pfändler}
\newcommand{\fach}{Baustoffe}


\newcommand{\punkte}[1]{%
    \begin{infobox}%
        #1
    \end{infobox}}%
\newcommand{\FinRes}[1]{\underline{\underline{#1}}}

\newmdenv[linecolor=black,backgroundcolor=gray!15,frametitle={Punktverteilung},leftmargin=1cm,rightmargin=1cm]{infobox}

\newcommand{\pagebreaksol}{
    \ifprintanswers
        \clearpage
    \else
        {}
    \fi
}

\newcommand{\pagebreakexam}{
    \ifprintanswers
        {}
    \else
        \clearpage
    \fi
}

\SolutionEmphasis{\color{blau_bauschule}}
\makeatletter%
\newcommand{\solutiontable}[1]{\ifprintanswers\begingroup\Solution@Emphasis#1\if@shadedsolutions%
            {\cellcolor{SolutionColor}}%
        \else%
        \fi\endgroup\else\phantom{#1}\fi}%
\makeatother%

\newcommand{\myNmm}[1]
{
    \sisetup{per-mode=symbol}
    \SI{#1}{\newton\per\mm\squared}
}

\renewcommand{\thequestion}{\fontsize{12pt}{2pt} \selectfont  \bfseries \arabic{question}}
\sisetup{per-mode=symbol}



%% Translation

\pointpoints{Punkt}{Punkte}
\bonuspointpoints{Bonuspunkt}{Bonuspunkte}
\renewcommand{\solutiontitle}{\noindent\textbf{Lösung:}\enspace}
\chqword{Frage}
\chpgword{Seite}
\chpword{Punkte}
\chbpword{Bonus Punkte}
\chsword{Erreicht}
\chtword{Gesamt}
\hpword{Punkte:}
\hsword{Ergebnis:}
\hqword{Aufgabe:}
\htword{Summe:}


\renewcommand{\questionshook}{%
  %\setlength{\leftmargin}{0pt}% removes the indentation from the left
  \setlength{\labelwidth}{1.25cm}% adjusts label width
  \setlength{\itemindent}{0cm}% aligns the start of the item with the above
  \setlength{\labelsep}{0.25cm}% space between the label and the item text
}




%% header and footer
\pagestyle{headandfoot}
\firstpageheadrule
\runningheadrule

% Adjust the font size for the header
\firstpageheader{\fontsize{9}{11}\selectfont\fach}{}{\fontsize{9}{11}\selectfont\dozent \\ \blattname}
\runningheader{\fontsize{9}{11}\selectfont\fach}{}{\fontsize{9}{11}\selectfont\dozent \\ \blattname}

% Adjust the font size for the footer
\firstpagefooter{\includegraphics[width=2.5cm]{bauschule-logo-5cm.png}}{}{\fontsize{9}{11}\selectfont\thepage\,/\,\pageref{LastPage}}
\runningfooter{\includegraphics[width=2.5cm]{bauschule-logo-5cm.png}}{}{\fontsize{9}{11}\selectfont\thepage\,/\,\pageref{LastPage}}

\newcommand{\mytitle}{U-Wert Berechnungen von alten Prüfungen}
\newcommand{\blattname}{\mytitle}

%% header and footer

\pagestyle{headandfoot}
\firstpageheadrule
\runningheadrule
\firstpageheader{\fach}{}{\fontsize{9pt}{2pt}\selectfont \dozent \\ \blattname}
\runningheader{\fach}{}{\fontsize{9pt}{2pt}\selectfont\dozent \\ \blattname}
\firstpagefooter{\includegraphics[width=2.5cm]{../../../../template/bauschule-logo-5cm.png}}{}{\fontsize{9pt}{2pt}\selectfont \thepage\,/\,\numpages}
\runningfooter{\includegraphics[width=2.5cm]{../../../../template/bauschule-logo-5cm.png}}{}{\fontsize{9pt}{2pt}\selectfont \thepage\,/\,\numpages}

%\printanswers




\begin{document}
\section*{\textbf{\mytitle}}

\subsection*{Formel}

\begin{equation}
    U = \dfrac{1}{R_{\text{Innen}} + \sum\dfrac{d_i}{\lambda_i} + R_{\text{Aussen}}}
\end{equation}

\vspace{2cm}


\section{HTa-24 (2021)}
\begin{questions}
    \question
    \begin{parts}
        \part[2\half]
        
        
        \newcommand{\UWertZiel}{0.20}
        \newcommand{\mydicke}{25}
        Leider fehlt in den Plänen die Angabe über den $\lambda$-Wert des Dämmmaterials. 
		Sie erkennen jedoch, dass der U-Wert \SI{\UWertZiel}{\W\per\m^2\K} sein sollte und dass, die Dämmstärke \SI{\mydicke}{\cm} beträgt. Sie können mit $R_{\text{Aussen}} = 0.04$ und $R_{\text{Innen}} = 0.130$ rechnen.
        Bestimmen Sie den $\lambda$-Wert des Dämmstoffes.
        
        \begin{solutionorbox}[7cm]
        Wir setzen wieder in Gleichung (1) ein. 
            \begin{equation*}
                \UWertZiel = \dfrac{1}{0.130 + \dfrac{\SI{\mydicke}{\cm}}{\lambda} + 0.040}
            \end{equation*}
            
            Wir formen die Gleichung um: 
            
            \begin{equation*}
                \dfrac{ \SI{\mydicke}{\cm} }{\lambda} = ( \dfrac{1}{\UWertZiel} - 0.040 - 0.130)
            \end{equation*}
            
            und weiter zu: 
            
            \newcommand{\resA}{\directlua{tex.sprint((\mydicke/100)/(1/\UWertZiel-0.040-0.130))}}
            
            
            \begin{eqnarray*}
                \lambda =& \dfrac{\SI{\mydicke}{\cm}}{ \dfrac{1}{\UWertZiel} - 0.040 - 0.130} \\ 
                %=& \dfrac{\SI{\mydicke}{\cm}}{\dfrac{1}{\UWertZiel} - 0.040 - 0.130} \\
                =& \FinRes{\SI{ \resA}{\W\per\m\per\K}} \\
                %\approx& \FinRes{\SI{ 0.0705}{\W\per\m\per\K}}
            \end{eqnarray*}
            \end{solutionorbox}
            
            \part[1]
            \label{q:Uwertaufgabe}
            \renewcommand{\UWertZiel}{0.15}
            \newcommand{\myLambda}{0.039}
            
            Sie haben EPS mit einem $\lambda$-Wert von \SI{\myLambda}{\W\per\m\per\K}. 
			Was ist der U-Wert bei einer Dämmstärke von \SI{25}{\cm}? 
            Sie können mit $R_{\text{Aussen}} = 0.04$ und $R_{\text{Innen}} = 0.130$ rechnen.
            \begin{solutionorbox}[3cm]
                Wir nehmen Gleichung 1 und setzen die Werte ein.
                \newcommand{\resB}{\directlua{tex.sprint(1/(0.130+(\mydicke/100)/(\myLambda)+0.04))}}
                \begin{equation*}
                    U = \dfrac{1}{0.130 + \dfrac{\SI{\mydicke}{\cm}}{\SI{\myLambda}{\W\per\m\K}} + 0.04}
                   %= \resB
                  = \FinRes{\SI{\resB}{\W\per\m^2\K}} 
                %\approx \FinRes{\SI{0.156}{\W\per\m^2\K}}
                \end{equation*}
            \end{solutionorbox}
            
			\clearpage
            \part[1]
            Leider wurde nur \SI{75}{\percent} der Dämmung verwendet, d.h. es wurde \SI{75}{\percent} von \SI{25}{\cm} gedämmt. Was ist der wirkliche U-Wert der Wand? Sie können mit $R_{\text{Aussen}} = 0.04$ und $R_{\text{Innen}} = 0.130$ rechnen und verwenden Sie den $\lambda$-Wert von Aufgabe (\ref{q:Uwertaufgabe}).
            
            \renewcommand{\mydicke}{18.75}
            
            \begin{solutionorbox}[3cm]
                Wir nehmen Gleichung 1 und setzen die Werte ein.
                \newcommand{\resC}{\directlua{tex.sprint(1/(0.130+(\mydicke/100)/(\myLambda)+0.04))}}
                \begin{equation*}
                    U = \dfrac{1}{0.130 + \dfrac{\SI{\mydicke}{\cm}}{\SI{\myLambda}{\W\per\m\K}} + 0.04}
                   %= \resB
                  = \FinRes{\SI{\resC}{\W\per\m^2\K}} 
                %\approx \FinRes{\SI{0.156}{\W\per\m^2\K}}
                \end{equation*}
            \end{solutionorbox}
\end{parts}
\end{questions}




\section{HTb-24 (2021)}
\begin{questions}
    \question
    \begin{parts}
        \part[2\half]
        
        
        \newcommand{\UWertZiel}{0.20}
        \newcommand{\mydicke}{22}
        Leider fehlt in den Plänen die Angabe über den $\lambda$-Wert des Dämmmaterials.
		Sie erkennen jedoch, dass der U-Wert \SI{\UWertZiel}{\W\per\m^2\K} sein sollte und dass, die Dämmstärke \SI{\mydicke}{\cm} beträgt. Sie können mit $R_{\text{Aussen}} = 0.04$ und $R_{\text{Innen}} = 0.130$ rechnen.
        Bestimmen Sie den $\lambda$-Wert des Dämmstoffes.
        
        \begin{solutionorbox}[7cm]
        Wir setzen wieder in Gleichung (1) ein. 
            \begin{equation*}
                \UWertZiel = \dfrac{1}{0.130 + \dfrac{\SI{\mydicke}{\cm}}{\lambda} + 0.040}
            \end{equation*}
            
            Wir formen die Gleichung um: 
            
            \begin{equation*}
                \dfrac{ \SI{\mydicke}{\cm} }{\lambda} = ( \dfrac{1}{\UWertZiel} - 0.040 - 0.130)
            \end{equation*}
            
            und weiter zu: 
            
            \newcommand{\resA}{\directlua{tex.sprint((\mydicke/100)/(1/\UWertZiel-0.040-0.130))}}
            
            
            \begin{eqnarray*}
                \lambda =& \dfrac{\SI{\mydicke}{\cm}}{ \dfrac{1}{\UWertZiel} - 0.040 - 0.130} \\ 
                %=& \dfrac{\SI{\mydicke}{\cm}}{\dfrac{1}{\UWertZiel} - 0.040 - 0.130} \\
                =& \FinRes{\SI{ \resA}{\W\per\m\per\K}} \\
                %\approx& \FinRes{\SI{ 0.0705}{\W\per\m\per\K}}
            \end{eqnarray*}
            \end{solutionorbox}
            
            \part[2]
            \renewcommand{\UWertZiel}{0.15}
            Sie wollen nun einen Dämmwert von \SI{\UWertZiel}{\W\per\m^2\K} erreichen. Welchen $\lambda$-Wert muss das Material haben?
            
            
                    \begin{solutionorbox}[4cm]
            Wir setzen wieder in Gleichung (1) ein. 
            \begin{equation*}
                \UWertZiel = \dfrac{1}{0.130 + \dfrac{\SI{\mydicke}{\cm}}{\lambda} + 0.040}
            \end{equation*}
            
            Wir formen die Gleichung um: 
            
            \begin{equation*}
                \dfrac{ \SI{\mydicke}{\cm} }{\lambda} = ( \dfrac{1}{\UWertZiel} - 0.040 - 0.130)
            \end{equation*}
            
            und weiter zu: 
            
            \newcommand{\resA}{\directlua{tex.sprint((\mydicke/100)/(1/\UWertZiel-0.040-0.130))}}
            
            
            \begin{eqnarray*}
                \lambda =& \dfrac{\SI{\mydicke}{\cm}}{ \dfrac{1}{\UWertZiel} - 0.040 - 0.130} \\ 
                %=& \dfrac{\SI{\mydicke}{\cm}}{\dfrac{1}{\UWertZiel} - 0.040 - 0.130} \\
                =& \FinRes{\SI{ \resA}{\W\per\m\per\K}} \\
                %\approx& \FinRes{\SI{ 0.0705}{\W\per\m\per\K}}
            \end{eqnarray*}
            \end{solutionorbox}
\end{parts}
\end{questions}
\end{document}