% !TEX program = lualatex
% !TEX root = /Users/patricpf/Documents/repos/Bauschule-Baustoffe/Unterlagen/01_Einführung/10_Aufgaben/10_pHWert/pHWert.tex



% !TEX program = lualatex
\documentclass[
    %answers,
    a4paper,ngerman,12pt, addpoints]{exam}

\usepackage[utf8]{inputenc}
%\usepackage[T1]{fontenc}
%\usepackage[ngerman]{babel}


\usepackage{polyglossia}
\setdefaultlanguage[variant = swiss]{german}
\usepackage{fontspec}
\setmainfont{Aptos} % Bauschule CI Manual
\setsansfont{Aptos} % Bauschule CI Manual


\usepackage[ a4paper,
 total={165mm,250mm},
 left=25mm,
 top=25mm,
 headsep=10mm
 %footsep=12mm
 %,showframe
  ]{geometry}

\usepackage{graphicx}
\usepackage{siunitx}
\usepackage{booktabs} % schöne Tabellen
\usepackage{float}
\floatplacement{figure}{H}
\usepackage{xcolor}
\usepackage{pdfpages}
\usepackage{enumitem}
\usepackage{mdframed} % Boxen
\usepackage{amsmath,amssymb}
\usepackage{tcolorbox}
\usepackage{lastpage} % For the total number of pages
\usepackage{gensymb}
\usepackage{xspace}
\usepackage{tabularx}
\usepackage{multicol}
\usepackage[
    version=3,
    arrows=pgf-filled,
]{mhchem} % für chemische Formeln
%\usepackage{microtype}
\usepackage{subfigure}
\usepackage[hidelinks]{hyperref}
\usepackage{cleveref}
\usepackage{luacode}
\usepackage{amsmath}
\usepackage{textcomp}


\sisetup{
  locale = DE,
  inter-unit-product = \ensuremath{{\cdot}},
  detect-all,
}

% Colors
\definecolor{blau_bauschule}{RGB}{22,65,148}
\CorrectChoiceEmphasis{\color{blau_bauschule}}
\SolutionEmphasis{\color{blau_bauschule}}

\setlength{\parindent}{0em} % Verhindert einrücken
\setlength\linefillheight{0.3in}


%% COMMMANDS
\author{Patrick Pfändler}
\newcommand{\dozent}{Patrick Pfändler}
\newcommand{\fach}{Baustoffe}


\newcommand{\punkte}[1]{%
    \begin{infobox}%
        #1
    \end{infobox}}%
\newcommand{\FinRes}[1]{\underline{\underline{#1}}}

\newmdenv[linecolor=black,backgroundcolor=gray!15,frametitle={Punktverteilung},leftmargin=1cm,rightmargin=1cm]{infobox}

\newcommand{\pagebreaksol}{
    \ifprintanswers
        \clearpage
    \else
        {}
    \fi
}

\newcommand{\pagebreakexam}{
    \ifprintanswers
        {}
    \else
        \clearpage
    \fi
}

\SolutionEmphasis{\color{blau_bauschule}}
\makeatletter%
\newcommand{\solutiontable}[1]{\ifprintanswers\begingroup\Solution@Emphasis#1\if@shadedsolutions%
            {\cellcolor{SolutionColor}}%
        \else%
        \fi\endgroup\else\phantom{#1}\fi}%
\makeatother%

\newcommand{\myNmm}[1]
{
    \sisetup{per-mode=symbol}
    \SI{#1}{\newton\per\mm\squared}
}

\renewcommand{\thequestion}{\fontsize{12pt}{2pt} \selectfont  \bfseries \arabic{question}}
\sisetup{per-mode=symbol}



%% Translation

\pointpoints{Punkt}{Punkte}
\bonuspointpoints{Bonuspunkt}{Bonuspunkte}
\renewcommand{\solutiontitle}{\noindent\textbf{Lösung:}\enspace}
\chqword{Frage}
\chpgword{Seite}
\chpword{Punkte}
\chbpword{Bonus Punkte}
\chsword{Erreicht}
\chtword{Gesamt}
\hpword{Punkte:}
\hsword{Ergebnis:}
\hqword{Aufgabe:}
\htword{Summe:}


\renewcommand{\questionshook}{%
  %\setlength{\leftmargin}{0pt}% removes the indentation from the left
  \setlength{\labelwidth}{1.25cm}% adjusts label width
  \setlength{\itemindent}{0cm}% aligns the start of the item with the above
  \setlength{\labelsep}{0.25cm}% space between the label and the item text
}




%% header and footer
\pagestyle{headandfoot}
\firstpageheadrule
\runningheadrule

% Adjust the font size for the header
\firstpageheader{\fontsize{9}{11}\selectfont\fach}{}{\fontsize{9}{11}\selectfont\dozent \\ \blattname}
\runningheader{\fontsize{9}{11}\selectfont\fach}{}{\fontsize{9}{11}\selectfont\dozent \\ \blattname}

% Adjust the font size for the footer
\firstpagefooter{\includegraphics[width=2.5cm]{bauschule-logo-5cm.png}}{}{\fontsize{9}{11}\selectfont\thepage\,/\,\pageref{LastPage}}
\runningfooter{\includegraphics[width=2.5cm]{bauschule-logo-5cm.png}}{}{\fontsize{9}{11}\selectfont\thepage\,/\,\pageref{LastPage}}


\newcommand{\blattname}{Aufgaben zum pH-Wert}

%% header and footer

\pagestyle{headandfoot}
\firstpageheadrule
\runningheadrule
\firstpageheader{\fach}{}{\fontsize{9pt}{2pt}\selectfont \dozent \\ \blattname}
\runningheader{\fach}{}{\fontsize{9pt}{2pt}\selectfont\dozent \\ \blattname}
\firstpagefooter{\includegraphics[width=2.5cm]{../../../../template/bauschule-logo-5cm.png}}{}{\fontsize{9pt}{2pt}\selectfont \thepage\,/\,\numpages}
\runningfooter{\includegraphics[width=2.5cm]{../../../../template/bauschule-logo-5cm.png}}{}{\fontsize{9pt}{2pt}\selectfont \thepage\,/\,\numpages}


\printanswers





\renewcommand{\thequestion}{\fontsize{15pt}{2pt} \selectfont  \bfseries \arabic{question}}





\newcommand{\pHeinfachLUA}[1]{%
\question[1]
Berechnen Sie den pH-Wert der Lösung mit einer Konzentration von \num{#1} an Hydroniumionen.


    \begin{solutionorbox}[3cm]
Der pH-Wert der Lösung beträgt
\directlua{calcpH(#1)}
.%
    \end{solutionorbox}


}

\begin{luacode}
function calcpH(zahl)
	    tex.print{string.format("%0.2f",-math.log(zahl,10))}
end
\end{luacode}



\begin{document}
{\fontsize{22pt}{2pt}\selectfont \textbf{\blattname}}
\vspace{0.3cm}

\section*{Theorie}

Der pH-Wert wird vereinfacht berechnet mit der Formel:

\begin{equation}
    pH = -log_{10}([H_3O^+])
\end{equation}



\section*{Aufgaben}


Lösen Sie die folgenden Aufgaben. Der Zeitbedarf ist ungefähr 20 Minuten.
%\maketitle
%\punkte{Runden Sie die Resultate sinnvoll.}

\begin{questions}
%\section{Sehr einfache Aufgaben}

\pHeinfachLUA{0.000000001}
\pHeinfachLUA{0.001}
\pHeinfachLUA{0.00001}


%\section{Einfache Aufgaben}

\pHeinfachLUA{0.000000005}

\ifprintanswers
{}
\else
\clearpage
\fi



\pHeinfachLUA{0.0008987}
\pHeinfachLUA{0.01271}

%\section{Mittlere Aufgaben}
\question[6]
Berechnen Sie die pH-Werte falls die Konzentrationen jeweils halbiert werden. 
Nehmen Sie dabei 6 Konzentrationen aus den vorherigen Aufgaben.

   \ifprintanswers
   {
    \begin{solution}
        
        \begin{enumerate}
            \item \directlua{calcpH(0.000000001*0.5)}
            \item \directlua{calcpH(0.001*0.5)}
            \item \directlua{calcpH(0.00001*0.5)}
            \item \directlua{calcpH(0.000000005*0.5)}
            \item \directlua{calcpH(0.0008987*0.5)}
            \item \directlua{calcpH(0.01271*0.5)}
        \end{enumerate}
    \end{solution}
}
    \else
        \makeemptybox{3cm}
        \makeemptybox{3cm}
        \makeemptybox{3cm}
        \makeemptybox{3cm}
        \makeemptybox{3cm}
        \makeemptybox{3cm}
    \fi



%\section{Weitere Aufgaben}

\question[1]
Welche Konzentration von Hydroniumionen hat Wasser?

\begin{solutionorbox}[3cm]
$10^{-7}$
\end{solutionorbox}

\question[1]
Berechnen Sie die Konzentration von \ce{H3O+} von  Beton. Sie dürfen den pH-Wert des Betons abschätzen.
\begin{solutionorbox}[3cm]
\begin{equation*}
    -log(x) = 13
\end{equation*}
\begin{equation*}
    x = 10^{-13}
\end{equation*}
\end{solutionorbox}

\question[1]

Berechnen Sie die Konzentration von \ce{H3O+} von karbonatisiertem Beton. Sie müssen den pH-Wert von karbonatisiertem Beton abschätzen.

\begin{solutionorbox}[3cm]
\begin{equation*}
    -log(x) = 10
\end{equation*}
\begin{equation*}
    x = 10^{-10}
\end{equation*}
\end{solutionorbox}

\ifprintanswers
{}
\else
\clearpage
\fi

\question[1]
Wie können Sie die Karbonatisierungstiefe bestimmen?
\begin{solutionorbox}[3cm]
Bohrkern entnehmen und Phenolphthalein (Indikator) besprühen
\end{solutionorbox}

\question[1]
Sie haben eine Lösung mit einem pH-Wert von 10 vor sich. Sie wollen einen pH-Wert von 9 erreichen. 
Wie stark müssen Sie die Lösung verdünnen?
\begin{solutionorbox}[3cm]
10-mal.
\end{solutionorbox}


%\section{Fachabschluss}
%
%\begin{solution}
%
%\begin{enumerate}
%	\item \directlua{calcpH(0.000000005)}
%	\item \directlua{calcpH(0.000000005*1/2)}
%	\item \directlua{calcpH(0.000000005*1/10)}
%
%\end{enumerate}
%\end{solution}





\end{questions}
\end{document}



	
	
\end{document}
