% !TEX root = /Users/patricpf/Documents/repos/Bauschule-Baustoffe/Unterlagen/01_Einführung/10_Aufgaben/07_Leistung_Berg/Leistung_Berg.tex
% !TEX program = lualatex
\documentclass[
    %answers,
    a4paper,ngerman,12pt, addpoints]{exam}

\usepackage[utf8]{inputenc}
%\usepackage[T1]{fontenc}
%\usepackage[ngerman]{babel}


\usepackage{polyglossia}
\setdefaultlanguage[variant = swiss]{german}
\usepackage{fontspec}
\setmainfont{Aptos} % Bauschule CI Manual
\setsansfont{Aptos} % Bauschule CI Manual


\usepackage[ a4paper,
 total={165mm,250mm},
 left=25mm,
 top=25mm,
 headsep=10mm
 %footsep=12mm
 %,showframe
  ]{geometry}

\usepackage{graphicx}
\usepackage{siunitx}
\usepackage{booktabs} % schöne Tabellen
\usepackage{float}
\floatplacement{figure}{H}
\usepackage{xcolor}
\usepackage{pdfpages}
\usepackage{enumitem}
\usepackage{mdframed} % Boxen
\usepackage{amsmath,amssymb}
\usepackage{tcolorbox}
\usepackage{lastpage} % For the total number of pages
\usepackage{gensymb}
\usepackage{xspace}
\usepackage{tabularx}
\usepackage{multicol}
\usepackage[
    version=3,
    arrows=pgf-filled,
]{mhchem} % für chemische Formeln
%\usepackage{microtype}
\usepackage{subfigure}
\usepackage[hidelinks]{hyperref}
\usepackage{cleveref}
\usepackage{luacode}
\usepackage{amsmath}
\usepackage{textcomp}


\sisetup{
  locale = DE,
  inter-unit-product = \ensuremath{{\cdot}},
  detect-all,
}

% Colors
\definecolor{blau_bauschule}{RGB}{22,65,148}
\CorrectChoiceEmphasis{\color{blau_bauschule}}
\SolutionEmphasis{\color{blau_bauschule}}

\setlength{\parindent}{0em} % Verhindert einrücken
\setlength\linefillheight{0.3in}


%% COMMMANDS
\author{Patrick Pfändler}
\newcommand{\dozent}{Patrick Pfändler}
\newcommand{\fach}{Baustoffe}


\newcommand{\punkte}[1]{%
    \begin{infobox}%
        #1
    \end{infobox}}%
\newcommand{\FinRes}[1]{\underline{\underline{#1}}}

\newmdenv[linecolor=black,backgroundcolor=gray!15,frametitle={Punktverteilung},leftmargin=1cm,rightmargin=1cm]{infobox}

\newcommand{\pagebreaksol}{
    \ifprintanswers
        \clearpage
    \else
        {}
    \fi
}

\newcommand{\pagebreakexam}{
    \ifprintanswers
        {}
    \else
        \clearpage
    \fi
}

\SolutionEmphasis{\color{blau_bauschule}}
\makeatletter%
\newcommand{\solutiontable}[1]{\ifprintanswers\begingroup\Solution@Emphasis#1\if@shadedsolutions%
            {\cellcolor{SolutionColor}}%
        \else%
        \fi\endgroup\else\phantom{#1}\fi}%
\makeatother%

\newcommand{\myNmm}[1]
{
    \sisetup{per-mode=symbol}
    \SI{#1}{\newton\per\mm\squared}
}

\renewcommand{\thequestion}{\fontsize{12pt}{2pt} \selectfont  \bfseries \arabic{question}}
\sisetup{per-mode=symbol}



%% Translation

\pointpoints{Punkt}{Punkte}
\bonuspointpoints{Bonuspunkt}{Bonuspunkte}
\renewcommand{\solutiontitle}{\noindent\textbf{Lösung:}\enspace}
\chqword{Frage}
\chpgword{Seite}
\chpword{Punkte}
\chbpword{Bonus Punkte}
\chsword{Erreicht}
\chtword{Gesamt}
\hpword{Punkte:}
\hsword{Ergebnis:}
\hqword{Aufgabe:}
\htword{Summe:}


\renewcommand{\questionshook}{%
  %\setlength{\leftmargin}{0pt}% removes the indentation from the left
  \setlength{\labelwidth}{1.25cm}% adjusts label width
  \setlength{\itemindent}{0cm}% aligns the start of the item with the above
  \setlength{\labelsep}{0.25cm}% space between the label and the item text
}




%% header and footer
\pagestyle{headandfoot}
\firstpageheadrule
\runningheadrule

% Adjust the font size for the header
\firstpageheader{\fontsize{9}{11}\selectfont\fach}{}{\fontsize{9}{11}\selectfont\dozent \\ \blattname}
\runningheader{\fontsize{9}{11}\selectfont\fach}{}{\fontsize{9}{11}\selectfont\dozent \\ \blattname}

% Adjust the font size for the footer
\firstpagefooter{\includegraphics[width=2.5cm]{bauschule-logo-5cm.png}}{}{\fontsize{9}{11}\selectfont\thepage\,/\,\pageref{LastPage}}
\runningfooter{\includegraphics[width=2.5cm]{bauschule-logo-5cm.png}}{}{\fontsize{9}{11}\selectfont\thepage\,/\,\pageref{LastPage}}

\newcommand{\blattname}{Physik: Leistung}


%% header and footer

\pagestyle{headandfoot}
\firstpageheadrule
\runningheadrule
\firstpageheader{\fach}{}{\fontsize{9pt}{2pt}\selectfont \dozent \\ \blattname}
\runningheader{\fach}{}{\fontsize{9pt}{2pt}\selectfont\dozent \\ \blattname}
\firstpagefooter{\includegraphics[width=2.5cm]{../../../../template/bauschule-logo-5cm.png}}{}{\fontsize{9pt}{2pt}\selectfont \thepage\,/\,\numpages}
\runningfooter{\includegraphics[width=2.5cm]{../../../../template/bauschule-logo-5cm.png}}{}{\fontsize{9pt}{2pt}\selectfont \thepage\,/\,\numpages}


\begin{document}

{\fontsize{22pt}{2pt}\selectfont \textbf{\blattname}}
\vspace{0.3cm}




\begin{questions}

\question
Die Leistung des Motors in einem Personenwagen beträgt maximal \SI{50}{\kilo\watt}. Das Auto hat eine Gesamtmasse von \SI{1000}{\kilo\gram}. Wie lange dauert es mindestens, bis der Wagen eine Höhendifferenz von \SI{100}{\meter} (z. B. auf einer Passstrasse) überwunden hat?

\begin{solution}

\textbf{Gegeben:}
\begin{itemize}
    \item Leistung $P = \SI{50}{\kilo\watt}$
    \item Masse $m = \SI{1000}{\kilo\gram}$
    \item Höhendifferenz $\Delta h = \SI{100}{\meter}$
\end{itemize}

\textbf{Gesucht:}
\begin{itemize}
    \item Zeit $t$
\end{itemize}

Die Leistung berechnet sich nach der Formel 
\[ P = \frac{\Delta E}{t} \]
Für die Lageenergie gilt 
\[ \Delta E = m \cdot g \cdot \Delta h \]
(Reibung unberücksichtigt). Daraus folgt
\[ t = \frac{m \cdot g \cdot \Delta h}{P} \]
Einsetzen der gegebenen Werte ergibt
\[ t = \frac{\SI{1000}{\kilo\gram} \cdot \SI{9.81}{\meter\per\second\squared} \cdot \SI{100}{\meter}}{\SI{50000}{\watt}} = \SI{20}{\second} \]

Das Auto benötigt mindestens \SI{20}{\second} für \SI{100}{\meter} Höhenunterschied.

\end{solution}


% Aufgabe 2
\question
Ein voller Lastwagen hat eine Gesamtmasse von \SI{24000}{\kilo\gram} und eine Leistung von \SI{200}{\kilo\watt}. Wie hoch ist seine maximale Geschwindigkeit auf einer horizontalen Strecke, wenn der Luftwiderstand und andere Reibungskräfte vernachlässigt werden?

\begin{solution}
Da keine Reibungskräfte berücksichtigt werden und der Lastwagen horizontal fährt, entspricht die Leistung des Lastwagens der kinetischen Energie:
\[ P = \frac{1}{2} m v^2 \]
Umstellen nach der Geschwindigkeit \( v \) ergibt:
\[ v = \sqrt{\frac{2P}{m}} \]
Einsetzen der gegebenen Werte:
\[ v = \sqrt{\frac{2 \cdot \SI{200000}{\watt}}{\SI{24000}{\kilo\gram}}} \approx \SI{29}{\meter\per\second} \]
\end{solution}

% Aufgabe 3
\question
Ein leerer Lastwagen mit einer Masse von \SI{12000}{\kilo\gram} beschleunigt in \SI{20}{\second} auf \SI{60}{\kilo\meter\per\hour}. Berechne die durchschnittliche Leistung während der Beschleunigungsphase.

\begin{solution}
Die kinetische Energie am Ende der Beschleunigungsphase ist:
\[ E_k = \frac{1}{2} m v^2 \]
Umrechnung der Geschwindigkeit in \si{\meter\per\second}:
\[ v = \SI{60}{\kilo\meter\per\hour} = \SI{16.67}{\meter\per\second} \]
Einsetzen in die Formel für kinetische Energie:
\[ E_k = \frac{1}{2} \cdot \SI{12000}{\kilo\gram} \cdot (\SI{16.67}{\meter\per\second})^2 \]
Die durchschnittliche Leistung ist dann:
\[ P = \frac{E_k}{t} \]
Einsetzen und Berechnen ergibt:
\[ P \approx \SI{16670}{\watt} \]
\end{solution}

\question
Ein Kran hebt eine Last von \SI{3000}{\kilo\gram} auf eine Höhe von \SI{15}{\meter}. Berechne die dafür benötigte Energie unter Vernachlässigung der Reibung.

\begin{solution}
Die benötigte Energie entspricht der potenziellen Energie am höchsten Punkt:
\[ E_p = m \cdot g \cdot h \]
Einsetzen der gegebenen Werte:
\[ E_p = \SI{3000}{\kilo\gram} \cdot \SI{9.81}{\meter\per\second\squared} \cdot \SI{15}{\meter} = \SI{441450}{\joule} \]
\end{solution}

% Aufgabe 2
\question
Ein Bauarbeiter schiebt einen voll beladenen Schubkarren mit einer Gesamtmasse von \SI{150}{\kilo\gram} über eine Strecke von \SI{100}{\meter}. Der Widerstandskoeffizient zwischen Schubkarren und Boden beträgt 0.05. Berechne die dafür notwendige Arbeit.

\begin{solution}
Die Arbeit gegen den Widerstand berechnet sich mit:
\[ W = f \cdot d \]
wobei \( f \) die Reibungskraft ist:
\[ f = \mu \cdot m \cdot g \]
Einsetzen und Berechnen der Reibungskraft:
\[ f = 0.05 \cdot \SI{150}{\kilo\gram} \cdot \SI{9.81}{\meter\per\second\squared} = \SI{73.575}{\newton} \]
Somit ist die Arbeit:
\[ W = \SI{73.575}{\newton} \cdot \SI{100}{\meter} = \SI{7357.5}{\joule} \]
\end{solution}

% Aufgabe 3
\question
Eine Baufirma möchte die Energieeffizienz ihres Bürogebäudes verbessern. Das Gebäude hat eine Aussenfläche von \SI{1200}{\meter\squared} und verliert durch diese Fläche im Durchschnitt \SI{40}{\watt\per\meter\squared}. Wie viel Energie in Kilowattstunden wird pro Tag (24 Stunden) verloren?

\begin{solution}
Der tägliche Energieverlust berechnet sich durch:
\[ E = P \cdot A \cdot t \]
wobei \( P \) der Wärmeverlust pro Quadratmeter, \( A \) die Fläche und \( t \) die Zeit in Stunden ist.
Einsetzen der Werte ergibt:
\[ E = \SI{40}{\watt\per\meter\squared} \cdot \SI{1200}{\meter\squared} \cdot \SI{24}{\hour} \]
\[ E = \SI{1152}{\kilo\watt\hour} \]
\end{solution}



\end{questions}

\end{document}

