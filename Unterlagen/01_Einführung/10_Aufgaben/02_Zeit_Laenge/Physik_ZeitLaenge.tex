% !TEX program = lualatex
\documentclass[
    %answers,
    a4paper,ngerman,12pt, addpoints]{exam}

\usepackage[utf8]{inputenc}
%\usepackage[T1]{fontenc}
%\usepackage[ngerman]{babel}


\usepackage{polyglossia}
\setdefaultlanguage[variant = swiss]{german}
\usepackage{fontspec}
\setmainfont{Aptos} % Bauschule CI Manual
\setsansfont{Aptos} % Bauschule CI Manual


\usepackage[ a4paper,
 total={165mm,250mm},
 left=25mm,
 top=25mm,
 headsep=10mm
 %footsep=12mm
 %,showframe
  ]{geometry}

\usepackage{graphicx}
\usepackage{siunitx}
\usepackage{booktabs} % schöne Tabellen
\usepackage{float}
\floatplacement{figure}{H}
\usepackage{xcolor}
\usepackage{pdfpages}
\usepackage{enumitem}
\usepackage{mdframed} % Boxen
\usepackage{amsmath,amssymb}
\usepackage{tcolorbox}
\usepackage{lastpage} % For the total number of pages
\usepackage{gensymb}
\usepackage{xspace}
\usepackage{tabularx}
\usepackage{multicol}
\usepackage[
    version=3,
    arrows=pgf-filled,
]{mhchem} % für chemische Formeln
%\usepackage{microtype}
\usepackage{subfigure}
\usepackage[hidelinks]{hyperref}
\usepackage{cleveref}
\usepackage{luacode}
\usepackage{amsmath}
\usepackage{textcomp}


\sisetup{
  locale = DE,
  inter-unit-product = \ensuremath{{\cdot}},
  detect-all,
}

% Colors
\definecolor{blau_bauschule}{RGB}{22,65,148}
\CorrectChoiceEmphasis{\color{blau_bauschule}}
\SolutionEmphasis{\color{blau_bauschule}}

\setlength{\parindent}{0em} % Verhindert einrücken
\setlength\linefillheight{0.3in}


%% COMMMANDS
\author{Patrick Pfändler}
\newcommand{\dozent}{Patrick Pfändler}
\newcommand{\fach}{Baustoffe}


\newcommand{\punkte}[1]{%
    \begin{infobox}%
        #1
    \end{infobox}}%
\newcommand{\FinRes}[1]{\underline{\underline{#1}}}

\newmdenv[linecolor=black,backgroundcolor=gray!15,frametitle={Punktverteilung},leftmargin=1cm,rightmargin=1cm]{infobox}

\newcommand{\pagebreaksol}{
    \ifprintanswers
        \clearpage
    \else
        {}
    \fi
}

\newcommand{\pagebreakexam}{
    \ifprintanswers
        {}
    \else
        \clearpage
    \fi
}

\SolutionEmphasis{\color{blau_bauschule}}
\makeatletter%
\newcommand{\solutiontable}[1]{\ifprintanswers\begingroup\Solution@Emphasis#1\if@shadedsolutions%
            {\cellcolor{SolutionColor}}%
        \else%
        \fi\endgroup\else\phantom{#1}\fi}%
\makeatother%

\newcommand{\myNmm}[1]
{
    \sisetup{per-mode=symbol}
    \SI{#1}{\newton\per\mm\squared}
}

\renewcommand{\thequestion}{\fontsize{12pt}{2pt} \selectfont  \bfseries \arabic{question}}
\sisetup{per-mode=symbol}



%% Translation

\pointpoints{Punkt}{Punkte}
\bonuspointpoints{Bonuspunkt}{Bonuspunkte}
\renewcommand{\solutiontitle}{\noindent\textbf{Lösung:}\enspace}
\chqword{Frage}
\chpgword{Seite}
\chpword{Punkte}
\chbpword{Bonus Punkte}
\chsword{Erreicht}
\chtword{Gesamt}
\hpword{Punkte:}
\hsword{Ergebnis:}
\hqword{Aufgabe:}
\htword{Summe:}


\renewcommand{\questionshook}{%
  %\setlength{\leftmargin}{0pt}% removes the indentation from the left
  \setlength{\labelwidth}{1.25cm}% adjusts label width
  \setlength{\itemindent}{0cm}% aligns the start of the item with the above
  \setlength{\labelsep}{0.25cm}% space between the label and the item text
}




%% header and footer
\pagestyle{headandfoot}
\firstpageheadrule
\runningheadrule

% Adjust the font size for the header
\firstpageheader{\fontsize{9}{11}\selectfont\fach}{}{\fontsize{9}{11}\selectfont\dozent \\ \blattname}
\runningheader{\fontsize{9}{11}\selectfont\fach}{}{\fontsize{9}{11}\selectfont\dozent \\ \blattname}

% Adjust the font size for the footer
\firstpagefooter{\includegraphics[width=2.5cm]{bauschule-logo-5cm.png}}{}{\fontsize{9}{11}\selectfont\thepage\,/\,\pageref{LastPage}}
\runningfooter{\includegraphics[width=2.5cm]{bauschule-logo-5cm.png}}{}{\fontsize{9}{11}\selectfont\thepage\,/\,\pageref{LastPage}}

\newcommand{\blattname}{Physik: Zeit und Länge, Flächen, Leistung}

\usepackage{amsmath}
\usepackage{geometry}
%% header and footer
\pagestyle{headandfoot}
\firstpageheadrule
\runningheadrule
\firstpageheader{\fach}{}{\fontsize{9pt}{2pt}\selectfont \dozent \\ \blattname}
\runningheader{\fach}{}{\fontsize{9pt}{2pt}\selectfont\dozent \\ \blattname}
\firstpagefooter{\includegraphics[width=2.5cm]{../../../../template/bauschule-logo-5cm.png}}{}{\fontsize{9pt}{2pt}\selectfont \thepage\,/\,\numpages}
\runningfooter{\includegraphics[width=2.5cm]{../../../../template/bauschule-logo-5cm.png}}{}{\fontsize{9pt}{2pt}\selectfont \thepage\,/\,\numpages}


%\printanswers
\usepackage{siunitx}

\sisetup{
  locale = DE,
  inter-unit-product = \ensuremath{{\cdot}},
}


\begin{document}

{\fontsize{22pt}{2pt}\selectfont \textbf{\blattname}}
\vspace{0.3cm}




\begin{questions}
  \question Ein Fundament für ein neues Gebäude hat die Form eines Rechtecks mit den Massen \SI{5}{\metre} Länge und \SI{3}{\metre} Breite. Berechnen Sie die Fläche des Fundaments.
  \begin{solution}
  Fläche = Länge $\times$ Breite = \SI{5}{\metre} $\times$ \SI{3}{\metre} = \FinRes{\SI{15}{\square\metre}}
  \end{solution}

  \question Für die Pflasterung eines quadratischen Vorplatzes ist die Fläche zu bestimmen. Der Platz hat eine Seitenlänge von \SI{4}{\metre}. Wie gross ist seine Fläche?
  \begin{solution}
  Fläche = Seite² = \SI{4}{\metre} $\times$ \SI{4}{\metre} = \FinRes{\SI{16}{\square\metre}}
  \end{solution}

  \question Ein Baufahrzeug fährt mit einer Durchschnittsgeschwindigkeit von \SI{60}{\kilo\metre\per\hour} zur Baustelle. Die Strecke beträgt \SI{120}{\kilo\metre}. Wie lange ist das Fahrzeug unterwegs?
  \begin{solution}
  Zeit = Strecke / Geschwindigkeit = \SI{120}{\kilo\metre} / \SI{60}{\kilo\metre\per\hour} = \FinRes{\SI{2}{\hour}}
  \end{solution}

  \question Berechnen Sie die Fläche eines dreieckigen Gartenteils, der an der breitesten Stelle \SI{6}{\metre} breit ist und eine Tiefe von \SI{4}{\metre} hat.
  \begin{solution}
  Fläche = (Basis $\times$ Höhe) / 2 = (\SI{6}{\metre} $\times$ \SI{4}{\metre}) / 2 = \FinRes{\SI{12}{\square\metre}}
  \end{solution}

  \question Auf dem Gelände soll ein Rundpool mit einem Durchmesser von \SI{10}{\metre} gebaut werden. Berechnen Sie die Fläche, die für den Pool benötigt wird.
  \begin{solution}
  Radius = Durchmesser / 2 = \SI{5}{\metre}. Fläche = $\pi \times$ Radius² $\approx$ 3.1416 $\times$ \SI{5}{\metre}² $\approx$ \FinRes{\SI{78.54}{\square\metre}}
  \end{solution}

  \question Ein Arbeiter benötigt \SI{2}{\hour}, um eine Fläche von \SI{100}{\square\metre} zu ebnen. Wie lange würde es dauern, eine Fläche von \SI{250}{\square\metre} zu bearbeiten?
  \begin{solution}
  Verhältnis = \SI{250}{\square\metre} / \SI{100}{\square\metre} = 2.5. Zeit = \SI{2}{\hour} $\times$ 2.5 = \FinRes{\SI{5}{\hour}}
  \end{solution}
  \pagebreaksol
  \question Die Fläche eines Lagerplatzes auf der Baustelle, der die Form eines Parallelogramms mit einer Basis von \SI{8}{\metre} und einer Höhe von \SI{3}{\metre} hat, soll berechnet werden.
  \begin{solution}
  Fläche = Basis $\times$ Höhe = \SI{8}{\metre} $\times$ \SI{3}{\metre} = \FinRes{\SI{24}{\square\metre}}
  \end{solution}

  \question Ein neuer Arbeitsbereich auf der Baustelle ist rechteckig und misst \SI{6}{\metre} in der Länge und \SI{2}{\metre} in der Breite. Wie gross ist die Fläche dieses Bereichs?
  \begin{solution}
  Fläche = Länge $\times$ Breite = \SI{6}{\metre} $\times$ \SI{2}{\metre} = \FinRes{\SI{12}{\square\metre}}
  \end{solution}

  \question Ein Lieferfahrzeug legt eine Strecke von \SI{180}{\kilo\metre} zur Baustelle zurück und benötigt dafür \SI{3}{\hour}. Wie hoch ist die Durchschnittsgeschwindigkeit des Fahrzeugs?
  \begin{solution}
  Geschwindigkeit = Strecke / Zeit = \SI{180}{\kilo\metre} / \SI{3}{\hour} = \FinRes{\SI{60}{\kilo\metre\per\hour}}
  \end{solution}

  \question Berechnen Sie die Fahrzeit für ein Baufahrzeug, das mit einer Durchschnittsgeschwindigkeit von \SI{80}{\kilo\metre\per\hour} fährt, um eine Baustelle in \SI{200}{\kilo\metre} Entfernung zu erreichen.
  \begin{solution}
  Zeit = Strecke / Geschwindigkeit = \SI{200}{\kilo\metre} / \SI{80}{\kilo\metre\per\hour} = \FinRes{\SI{2.5}{\hour}}
  \end{solution}

  \question Ein Betonmischer hat ein zylindrisches Fass mit einem Durchmesser von \SI{2.5}{\metre} und einer Höhe von \SI{3}{\metre}. Berechnen Sie das Volumen des Fasses.
  \begin{solution}
  Radius = Durchmesser / 2 = \SI{1.25}{\metre}. Volumen = $\pi \times$ Radius² $\times$ Höhe $\approx$ 3.1416 $\times$ \SI{1.25}{\metre}² $\times$ \SI{3}{\metre} \approx \FinRes{\SI{14.68}{\cubic\metre}}
  \end{solution}

  \question Für das Giessen einer Betonplatte wird ein rechteckiger Behälter mit den Innenmassen \SI{6}{\metre} Länge, \SI{4}{\metre} Breite und \SI{0.5}{\metre} Höhe verwendet. Wie viel Kubikmeter Beton kann der Behälter maximal aufnehmen?
  \begin{solution}
  Volumen = Länge $\times$ Breite $\times$ Höhe = \SI{6}{\metre} $\times$ \SI{4}{\metre} $\times$ \SI{0.5}{\metre} = \FinRes{\SI{12}{\cubic\metre}}
  \end{solution}
\end{questions}

\end{document}