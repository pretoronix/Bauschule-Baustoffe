% !TEX program = lualatex
\documentclass[
    %answers,
    a4paper,ngerman,12pt, addpoints]{exam}

\usepackage[utf8]{inputenc}
%\usepackage[T1]{fontenc}
%\usepackage[ngerman]{babel}


\usepackage{polyglossia}
\setdefaultlanguage[variant = swiss]{german}
\usepackage{fontspec}
\setmainfont{Aptos} % Bauschule CI Manual
\setsansfont{Aptos} % Bauschule CI Manual


\usepackage[ a4paper,
 total={165mm,250mm},
 left=25mm,
 top=25mm,
 headsep=10mm
 %footsep=12mm
 %,showframe
  ]{geometry}

\usepackage{graphicx}
\usepackage{siunitx}
\usepackage{booktabs} % schöne Tabellen
\usepackage{float}
\floatplacement{figure}{H}
\usepackage{xcolor}
\usepackage{pdfpages}
\usepackage{enumitem}
\usepackage{mdframed} % Boxen
\usepackage{amsmath,amssymb}
\usepackage{tcolorbox}
\usepackage{lastpage} % For the total number of pages
\usepackage{gensymb}
\usepackage{xspace}
\usepackage{tabularx}
\usepackage{multicol}
\usepackage[
    version=3,
    arrows=pgf-filled,
]{mhchem} % für chemische Formeln
%\usepackage{microtype}
\usepackage{subfigure}
\usepackage[hidelinks]{hyperref}
\usepackage{cleveref}
\usepackage{luacode}
\usepackage{amsmath}
\usepackage{textcomp}


\sisetup{
  locale = DE,
  inter-unit-product = \ensuremath{{\cdot}},
  detect-all,
}

% Colors
\definecolor{blau_bauschule}{RGB}{22,65,148}
\CorrectChoiceEmphasis{\color{blau_bauschule}}
\SolutionEmphasis{\color{blau_bauschule}}

\setlength{\parindent}{0em} % Verhindert einrücken
\setlength\linefillheight{0.3in}


%% COMMMANDS
\author{Patrick Pfändler}
\newcommand{\dozent}{Patrick Pfändler}
\newcommand{\fach}{Baustoffe}


\newcommand{\punkte}[1]{%
    \begin{infobox}%
        #1
    \end{infobox}}%
\newcommand{\FinRes}[1]{\underline{\underline{#1}}}

\newmdenv[linecolor=black,backgroundcolor=gray!15,frametitle={Punktverteilung},leftmargin=1cm,rightmargin=1cm]{infobox}

\newcommand{\pagebreaksol}{
    \ifprintanswers
        \clearpage
    \else
        {}
    \fi
}

\newcommand{\pagebreakexam}{
    \ifprintanswers
        {}
    \else
        \clearpage
    \fi
}

\SolutionEmphasis{\color{blau_bauschule}}
\makeatletter%
\newcommand{\solutiontable}[1]{\ifprintanswers\begingroup\Solution@Emphasis#1\if@shadedsolutions%
            {\cellcolor{SolutionColor}}%
        \else%
        \fi\endgroup\else\phantom{#1}\fi}%
\makeatother%

\newcommand{\myNmm}[1]
{
    \sisetup{per-mode=symbol}
    \SI{#1}{\newton\per\mm\squared}
}

\renewcommand{\thequestion}{\fontsize{12pt}{2pt} \selectfont  \bfseries \arabic{question}}
\sisetup{per-mode=symbol}



%% Translation

\pointpoints{Punkt}{Punkte}
\bonuspointpoints{Bonuspunkt}{Bonuspunkte}
\renewcommand{\solutiontitle}{\noindent\textbf{Lösung:}\enspace}
\chqword{Frage}
\chpgword{Seite}
\chpword{Punkte}
\chbpword{Bonus Punkte}
\chsword{Erreicht}
\chtword{Gesamt}
\hpword{Punkte:}
\hsword{Ergebnis:}
\hqword{Aufgabe:}
\htword{Summe:}


\renewcommand{\questionshook}{%
  %\setlength{\leftmargin}{0pt}% removes the indentation from the left
  \setlength{\labelwidth}{1.25cm}% adjusts label width
  \setlength{\itemindent}{0cm}% aligns the start of the item with the above
  \setlength{\labelsep}{0.25cm}% space between the label and the item text
}




%% header and footer
\pagestyle{headandfoot}
\firstpageheadrule
\runningheadrule

% Adjust the font size for the header
\firstpageheader{\fontsize{9}{11}\selectfont\fach}{}{\fontsize{9}{11}\selectfont\dozent \\ \blattname}
\runningheader{\fontsize{9}{11}\selectfont\fach}{}{\fontsize{9}{11}\selectfont\dozent \\ \blattname}

% Adjust the font size for the footer
\firstpagefooter{\includegraphics[width=2.5cm]{bauschule-logo-5cm.png}}{}{\fontsize{9}{11}\selectfont\thepage\,/\,\pageref{LastPage}}
\runningfooter{\includegraphics[width=2.5cm]{bauschule-logo-5cm.png}}{}{\fontsize{9}{11}\selectfont\thepage\,/\,\pageref{LastPage}}


\newcommand{\blattname}{Physik: Wärmeausdehnung von Baustoffen}

\usepackage{amsmath}
\usepackage{geometry}
%% header and footer

\pagestyle{headandfoot}
\firstpageheadrule
\runningheadrule
\firstpageheader{\fach}{}{\dozent \\ \blattname}
\runningheader{\fach}{}{\dozent \\ \blattname}
\firstpagefooter{}{}{\thepage\,/\,\numpages}
\runningfooter{}{}{\thepage\,/\,\numpages}

%\printanswers

\usepackage{siunitx}

\sisetup{
  locale = DE,
  inter-unit-product = \ensuremath{{\cdot}},
}


\begin{document}

{\fontsize{22pt}{2pt}\selectfont \textbf{\blattname}}
\vspace{0.3cm}



\section*{Theoretische Grundlagen}
Die Wärmeausdehnung beschreibt die Änderung der Abmessungen (Länge, Volumen) eines Körpers bei Temperaturänderung. Die lineare Ausdehnung kann mit der folgenden Formel beschrieben werden:
\[
\Delta L = \alpha \cdot L_0 \cdot \Delta T
\]
wobei:
\begin{itemize}
    \item $\Delta L$ die Längenänderung ist,
    \item $\alpha$ der lineare Ausdehnungskoeffizient,
    \item $L_0$ die ursprüngliche Länge,
    \item $\Delta T$ die Temperaturänderung.
\end{itemize}

\section*{Ausdehnungskoeffizienten verschiedener Baustoffe}
In der folgenden Tabelle sind die linearen Ausdehnungskoeffizienten einiger gängiger Baustoffe aufgeführt:

\begin{table}[h]
\centering
\begin{tabular}{@{}lc@{}}
\toprule
\textbf{Baustoff} & \textbf{Linearer Ausdehnungskoeffizient [\si{mm/(m\cdot K)}]} \\ 
\midrule
Stahl & \num{0.013} \\
Beton & \num{0.012} \\
Backsteinmauerwerk & \num{0.005
}\\
Kalksandsteinmauerwerk & \num{0.008
}\\
Kupfer & \num{0.019
}\\
Aluminium & \num{0.023} \\
Glas & \num{0.009} \\
Kunststoff (angenommen) & \num{0.080} \\
%Glas & \num{9e-6} \\
\bottomrule
\end{tabular}
\caption{Ausdehnungskoeffizienten verschiedener Baustoffe.}
\label{tab:ausdehnungskoeffizienten}
\end{table}

\newpage 
\section*{Aufgaben}
Löse die untenstehenden Aufgaben. Runde das Ergebnis auf eine sinnvolle Anzahl von Dezimalstellen. Für die Bearbeitung der Aufgaben kannst du die obenstehende Tabelle (\cref{tab:ausdehnungskoeffizienten}) verwenden.

Für die Bearbeitung sollten \SI{20}{\min} ausreichen.

\begin{questions}
\question Berechnen Sie die Längenänderung eines 10 Meter langen Stahlträgers, wenn er sich von 20°C auf 35°C erwärmt. Der lineare Ausdehnungskoeffizient von Stahl beträgt \num{0.013} \si{mm/(m\cdot K)}.
\begin{solution}
    \[ \Delta L = \alpha \cdot L_0 \cdot \Delta T = \num{0.013} \cdot 10 \cdot 15 = \num{1.95} \, \si{mm} \]
\end{solution}

\question Ein Betonweg ist ursprünglich 50 Meter lang. Bei extremen Temperaturschwankungen zwischen Winter und Sommer kann sich die Temperatur um bis zu 30°C ändern. Berechnen Sie die maximale Längenänderung des Weges, wenn der Ausdehnungskoeffizient von Beton \num{0.012} \si{mm/(m\cdot K)} ist.
\begin{solution}
    \[ \Delta L = \num{0.012} \cdot 50 \cdot 30 = \num{18} \, \si{mm} \]
\end{solution}

\question Ein Kupferrohr hat bei einer Temperatur von 15°C eine Länge von 25 Metern. Wie lang ist das Rohr bei einer Temperatur von 60°C? Der Ausdehnungskoeffizient von Kupfer beträgt \num{0.019} \si{mm/(m\cdot K)}.
\begin{solution}
    \[ \Delta L = \num{0.019} \cdot 25 \cdot 45 = \num{21.375} \, \si{mm} \]

    Länge bei 60°C: $25 \, \text{m} + 21.375 \, \text{mm}$  = \num{25.021375} \si{m}
\end{solution}


\question Ein Aluminiumstab mit einer Anfangslänge von 30 Metern wird von -10°C auf 40°C erwärmt. Berechnen Sie die Längenänderung des Stabes. Der lineare Ausdehnungskoeffizient von Aluminium beträgt \num{0.023} \si{mm/(m\cdot K)}.
\begin{solution}
    \[ \Delta L = \alpha \cdot L_0 \cdot \Delta T = \num{0.023} \cdot 30 \cdot (40 - (-10)) = \num{34.5} \, \si{mm} \]
\end{solution}

\question Ein Glasfenster misst im Winter bei -5°C 2m x 3m. Wie gross ist seine Fläche im Sommer bei 35°C? Der lineare Ausdehnungskoeffizient von Glas beträgt \num{0.009} \si{mm/(m\cdot K)}. Nehmen Sie an, dass die Ausdehnung in beiden Dimensionen gleich ist.
\begin{solution}
    \[ \Delta L = \alpha \cdot L_0 \cdot \Delta T = \num{0.009} \cdot 2 \cdot (35 - (-5)) = \num{0.72} \, \si{mm} \]
    \[ \Delta B = \alpha \cdot B_0 \cdot \Delta T = \num{0.009} \cdot 3 \cdot (35 - (-5)) = \num{1.08} \, \si{mm} \]
    \[ Neue Flaeche = (2 \, \text{m} + 0.72 \, \text{mm}) \times (3 \, \text{m} + 1.08 \, \text{mm}) \] $= \num{6.00432078} \si{mm^2}$
\end{solution}

\question Ein Kunststoffrohr hat eine ursprüngliche Länge von 15 Metern bei 25°C. Wie lang ist das Rohr bei einer Temperatur von -20°C? Der Ausdehnungskoeffizient von Kunststoff (angenommen) beträgt \num{0.080} \si{mm/(m\cdot K)}.
\begin{solution}
    \[ \Delta L = \alpha \cdot L_0 \cdot \Delta T = \num{0.080} \cdot 15 \cdot (-20 - 25) = \num{-54} \, \si{mm} \]
\end{solution}



\question Ein Stahlbetonträger hat eine Länge von 30 Metern. Berechnen Sie den Unterschied in der Längenänderung zwischen dem Stahl und dem Beton, wenn sich die Temperatur von 10°C auf 40°C ändert. Der lineare Ausdehnungskoeffizient von Stahl beträgt \num{0.013} \si{mm/(m\cdot K)} und der von Beton \num{0.012} \si{mm/(m\cdot K)}.
\begin{solution}
    \[ \Delta L_{\text{Stahl}} = \num{0.013} \cdot 30 \cdot (40 - 10) = \num{11.7} \, \si{mm} \]
    \[ \Delta L_{\text{Beton}} = \num{0.012} \cdot 30 \cdot (40 - 10) = \num{10.8} \, \si{mm} \]
    \[ \text{Unterschied} = \Delta L_{\text{Stahl}} - \Delta L_{\text{Beton}} = \num{11.7} \, \si{mm} - \num{10.8} \, \si{mm} = \num{0.9} \, \si{mm} \]
\end{solution}

\end{questions}

\end{document}