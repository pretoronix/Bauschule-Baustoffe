
% !TEX root = /Users/patricpf/Documents/repos/Bauschule-Baustoffe/Unterlagen/01_Einführung/10_Aufgaben/05_Dichteanamolie/Dichteanamolie.tex
% !TEX program = lualatex
\documentclass[
    %answers,
    a4paper,ngerman,12pt, addpoints]{exam}

\usepackage[utf8]{inputenc}
%\usepackage[T1]{fontenc}
%\usepackage[ngerman]{babel}


\usepackage{polyglossia}
\setdefaultlanguage[variant = swiss]{german}
\usepackage{fontspec}
\setmainfont{Aptos} % Bauschule CI Manual
\setsansfont{Aptos} % Bauschule CI Manual


\usepackage[ a4paper,
 total={165mm,250mm},
 left=25mm,
 top=25mm,
 headsep=10mm
 %footsep=12mm
 %,showframe
  ]{geometry}

\usepackage{graphicx}
\usepackage{siunitx}
\usepackage{booktabs} % schöne Tabellen
\usepackage{float}
\floatplacement{figure}{H}
\usepackage{xcolor}
\usepackage{pdfpages}
\usepackage{enumitem}
\usepackage{mdframed} % Boxen
\usepackage{amsmath,amssymb}
\usepackage{tcolorbox}
\usepackage{lastpage} % For the total number of pages
\usepackage{gensymb}
\usepackage{xspace}
\usepackage{tabularx}
\usepackage{multicol}
\usepackage[
    version=3,
    arrows=pgf-filled,
]{mhchem} % für chemische Formeln
%\usepackage{microtype}
\usepackage{subfigure}
\usepackage[hidelinks]{hyperref}
\usepackage{cleveref}
\usepackage{luacode}
\usepackage{amsmath}
\usepackage{textcomp}


\sisetup{
  locale = DE,
  inter-unit-product = \ensuremath{{\cdot}},
  detect-all,
}

% Colors
\definecolor{blau_bauschule}{RGB}{22,65,148}
\CorrectChoiceEmphasis{\color{blau_bauschule}}
\SolutionEmphasis{\color{blau_bauschule}}

\setlength{\parindent}{0em} % Verhindert einrücken
\setlength\linefillheight{0.3in}


%% COMMMANDS
\author{Patrick Pfändler}
\newcommand{\dozent}{Patrick Pfändler}
\newcommand{\fach}{Baustoffe}


\newcommand{\punkte}[1]{%
    \begin{infobox}%
        #1
    \end{infobox}}%
\newcommand{\FinRes}[1]{\underline{\underline{#1}}}

\newmdenv[linecolor=black,backgroundcolor=gray!15,frametitle={Punktverteilung},leftmargin=1cm,rightmargin=1cm]{infobox}

\newcommand{\pagebreaksol}{
    \ifprintanswers
        \clearpage
    \else
        {}
    \fi
}

\newcommand{\pagebreakexam}{
    \ifprintanswers
        {}
    \else
        \clearpage
    \fi
}

\SolutionEmphasis{\color{blau_bauschule}}
\makeatletter%
\newcommand{\solutiontable}[1]{\ifprintanswers\begingroup\Solution@Emphasis#1\if@shadedsolutions%
            {\cellcolor{SolutionColor}}%
        \else%
        \fi\endgroup\else\phantom{#1}\fi}%
\makeatother%

\newcommand{\myNmm}[1]
{
    \sisetup{per-mode=symbol}
    \SI{#1}{\newton\per\mm\squared}
}

\renewcommand{\thequestion}{\fontsize{12pt}{2pt} \selectfont  \bfseries \arabic{question}}
\sisetup{per-mode=symbol}



%% Translation

\pointpoints{Punkt}{Punkte}
\bonuspointpoints{Bonuspunkt}{Bonuspunkte}
\renewcommand{\solutiontitle}{\noindent\textbf{Lösung:}\enspace}
\chqword{Frage}
\chpgword{Seite}
\chpword{Punkte}
\chbpword{Bonus Punkte}
\chsword{Erreicht}
\chtword{Gesamt}
\hpword{Punkte:}
\hsword{Ergebnis:}
\hqword{Aufgabe:}
\htword{Summe:}


\renewcommand{\questionshook}{%
  %\setlength{\leftmargin}{0pt}% removes the indentation from the left
  \setlength{\labelwidth}{1.25cm}% adjusts label width
  \setlength{\itemindent}{0cm}% aligns the start of the item with the above
  \setlength{\labelsep}{0.25cm}% space between the label and the item text
}




%% header and footer
\pagestyle{headandfoot}
\firstpageheadrule
\runningheadrule

% Adjust the font size for the header
\firstpageheader{\fontsize{9}{11}\selectfont\fach}{}{\fontsize{9}{11}\selectfont\dozent \\ \blattname}
\runningheader{\fontsize{9}{11}\selectfont\fach}{}{\fontsize{9}{11}\selectfont\dozent \\ \blattname}

% Adjust the font size for the footer
\firstpagefooter{\includegraphics[width=2.5cm]{bauschule-logo-5cm.png}}{}{\fontsize{9}{11}\selectfont\thepage\,/\,\pageref{LastPage}}
\runningfooter{\includegraphics[width=2.5cm]{bauschule-logo-5cm.png}}{}{\fontsize{9}{11}\selectfont\thepage\,/\,\pageref{LastPage}}




\newcommand{\blattname}{Dichteanomalie des Wassers}

%% header and footer

\pagestyle{headandfoot}
\firstpageheadrule
\runningheadrule
\firstpageheader{\fach}{}{\fontsize{9pt}{2pt}\selectfont \dozent \\ \blattname}
\runningheader{\fach}{}{\fontsize{9pt}{2pt}\selectfont\dozent \\ \blattname}
\firstpagefooter{\includegraphics[width=2.5cm]{../../../../template/bauschule-logo-5cm.png}}{}{\fontsize{9pt}{2pt}\selectfont \thepage\,/\,\numpages}
\runningfooter{\includegraphics[width=2.5cm]{../../../../template/bauschule-logo-5cm.png}}{}{\fontsize{9pt}{2pt}\selectfont \thepage\,/\,\numpages}





\begin{document}

{\fontsize{22pt}{2pt}\selectfont \textbf{\blattname}}
\vspace{0.3cm}


\textit{Hinweis: }Diese Aufgaben lassen sich nur teilweise mit Skript oder den Folien lösen. Gewisse Aufgaben oder Teilaufgaben müssen im Selbststudium erarbeitet werden. Sie dürfen mit rund \SI{1000}{\hecto\Pa} Atmospährendruck rechnen.


\begin{questions}
    \question
    Was ist die Dichte-Anomalie von Wasser resp. wie unterscheidet sich die Dichte von Wasser in Abhängigkeit von der Temperatur im Vergleich zu anderen Stoffen.


    \begin{solution}
        Bei praktisch allen Stoffen ist die Dichte im festen Aggregatszustand höher als im flüssigen; zudem nimmt die Dichte einer Flüssigkeit normalerweise mit steigender Temperatur ab.
        Beides trifft auf Wasser nicht zu.
        (Annahme: Druckverhältnisse ändern sich nicht).
    \end{solution}


    \question
    Beschreiben Sie die Auswirkungen dieser Anomalie für Gewässer im Winter  kurz.

    \begin{solution}
        Eis schwimmt auf Wasser, d.h. Eis hat eine geringere Dichte als Wasser im flüssigen Zustand.
        Wasserorganismen können nahe des Grundes des Gewässers den Winter überleben.
    \end{solution}

    \question
    Gefrierendes Wasser kann Steine zum Platzen bringen. Erklären Sie diese Sprengwirkung.

    \begin{solution}
        Beim Gefrieren wird flüssiges Wasser zu Eis . Da Eis ein grösseres Volumen beansprucht als flüssiges Wasser, werden Kräfte auf das Gestein ausgeübt. Diese Kräfte können teilweise kleinere Felsbrocken "wegsprengen".
    \end{solution}


    \question
    Welches Volumen hat 10 kg Eis und ein 10 kg flüssiges Wasser bei 0°C.

    ($\rho_{\text{Eis,} 0\degree C}=0.9170 g/cm^3$ und $\rho_{\text{Wasser,} 0\degree C}= \SI{0.9998}{\g\per\cm^3}$)



    \begin{solution}
        \begin{equation}
            \rho = \dfrac{m}{V}
        \end{equation}

        Diese Gleichung wird aufgelöst nach dem Volumen $V$.

        \begin{equation*}
            V = \dfrac{m}{\rho}
        \end{equation*}

        Für Eis bei 0°C gilt somit:
        \begin{equation*}
            V = \dfrac{m}{\rho} = \dfrac{10'000\, g}{0.9170\,g/cm^3} = 10'905\,cm^3 = \FinRes{10.905\,dm^3}
        \end{equation*}

        Für Wasser bei 0°C gilt somit:
        \begin{equation*}
            V = \dfrac{m}{\rho} = \dfrac{10'000\,g}{0.9998\, g/cm^3} = 10'002\,cm^3 = \FinRes{10.002\,dm^3}
        \end{equation*}
    \end{solution}

    \question
    Zeichnen eines Diagramms mit den folgenden Achsen: x-Achse Temperatur und y-Achse: Dichte von Wasser, jeweils unter Angabe der korrekten Einheit der Achse und einer Kurve. Die Kurve sollte einen Bereich von mindestens \SI{0}{\degreeCelsius} bis mindestesns \SI{10}{\degreeCelsius} abdecken

    \begin{solution}
        \begin{figure}[H]
            \includegraphics[width=0.75\textwidth]{Dichte_Wasser_Diagramm}
        \end{figure}
    \end{solution}

    \question
    Bei welcher Temperatur in der Einheit Kelvin  hat Wasser die höchste Dichte?

    \begin{solution}
        \SI{4}{\degreeCelsius} = \FinRes{\SI{277.15}{\K}}
    \end{solution}


\end{questions}
\end{document}