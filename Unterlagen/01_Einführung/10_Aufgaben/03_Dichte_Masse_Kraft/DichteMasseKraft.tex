% !TEX root = /Users/patricpf/Documents/repos/Bauschule-Baustoffe/Unterlagen/01_Einführung/10_Aufgaben/03_Dichte_Masse_Kraft/DichteMasseKraft.tex

% !TEX program = lualatex
\documentclass[
    %answers,
    a4paper,ngerman,12pt, addpoints]{exam}

\usepackage[utf8]{inputenc}
%\usepackage[T1]{fontenc}
%\usepackage[ngerman]{babel}


\usepackage{polyglossia}
\setdefaultlanguage[variant = swiss]{german}
\usepackage{fontspec}
\setmainfont{Aptos} % Bauschule CI Manual
\setsansfont{Aptos} % Bauschule CI Manual


\usepackage[ a4paper,
 total={165mm,250mm},
 left=25mm,
 top=25mm,
 headsep=10mm
 %footsep=12mm
 %,showframe
  ]{geometry}

\usepackage{graphicx}
\usepackage{siunitx}
\usepackage{booktabs} % schöne Tabellen
\usepackage{float}
\floatplacement{figure}{H}
\usepackage{xcolor}
\usepackage{pdfpages}
\usepackage{enumitem}
\usepackage{mdframed} % Boxen
\usepackage{amsmath,amssymb}
\usepackage{tcolorbox}
\usepackage{lastpage} % For the total number of pages
\usepackage{gensymb}
\usepackage{xspace}
\usepackage{tabularx}
\usepackage{multicol}
\usepackage[
    version=3,
    arrows=pgf-filled,
]{mhchem} % für chemische Formeln
%\usepackage{microtype}
\usepackage{subfigure}
\usepackage[hidelinks]{hyperref}
\usepackage{cleveref}
\usepackage{luacode}
\usepackage{amsmath}
\usepackage{textcomp}


\sisetup{
  locale = DE,
  inter-unit-product = \ensuremath{{\cdot}},
  detect-all,
}

% Colors
\definecolor{blau_bauschule}{RGB}{22,65,148}
\CorrectChoiceEmphasis{\color{blau_bauschule}}
\SolutionEmphasis{\color{blau_bauschule}}

\setlength{\parindent}{0em} % Verhindert einrücken
\setlength\linefillheight{0.3in}


%% COMMMANDS
\author{Patrick Pfändler}
\newcommand{\dozent}{Patrick Pfändler}
\newcommand{\fach}{Baustoffe}


\newcommand{\punkte}[1]{%
    \begin{infobox}%
        #1
    \end{infobox}}%
\newcommand{\FinRes}[1]{\underline{\underline{#1}}}

\newmdenv[linecolor=black,backgroundcolor=gray!15,frametitle={Punktverteilung},leftmargin=1cm,rightmargin=1cm]{infobox}

\newcommand{\pagebreaksol}{
    \ifprintanswers
        \clearpage
    \else
        {}
    \fi
}

\newcommand{\pagebreakexam}{
    \ifprintanswers
        {}
    \else
        \clearpage
    \fi
}

\SolutionEmphasis{\color{blau_bauschule}}
\makeatletter%
\newcommand{\solutiontable}[1]{\ifprintanswers\begingroup\Solution@Emphasis#1\if@shadedsolutions%
            {\cellcolor{SolutionColor}}%
        \else%
        \fi\endgroup\else\phantom{#1}\fi}%
\makeatother%

\newcommand{\myNmm}[1]
{
    \sisetup{per-mode=symbol}
    \SI{#1}{\newton\per\mm\squared}
}

\renewcommand{\thequestion}{\fontsize{12pt}{2pt} \selectfont  \bfseries \arabic{question}}
\sisetup{per-mode=symbol}



%% Translation

\pointpoints{Punkt}{Punkte}
\bonuspointpoints{Bonuspunkt}{Bonuspunkte}
\renewcommand{\solutiontitle}{\noindent\textbf{Lösung:}\enspace}
\chqword{Frage}
\chpgword{Seite}
\chpword{Punkte}
\chbpword{Bonus Punkte}
\chsword{Erreicht}
\chtword{Gesamt}
\hpword{Punkte:}
\hsword{Ergebnis:}
\hqword{Aufgabe:}
\htword{Summe:}


\renewcommand{\questionshook}{%
  %\setlength{\leftmargin}{0pt}% removes the indentation from the left
  \setlength{\labelwidth}{1.25cm}% adjusts label width
  \setlength{\itemindent}{0cm}% aligns the start of the item with the above
  \setlength{\labelsep}{0.25cm}% space between the label and the item text
}




%% header and footer
\pagestyle{headandfoot}
\firstpageheadrule
\runningheadrule

% Adjust the font size for the header
\firstpageheader{\fontsize{9}{11}\selectfont\fach}{}{\fontsize{9}{11}\selectfont\dozent \\ \blattname}
\runningheader{\fontsize{9}{11}\selectfont\fach}{}{\fontsize{9}{11}\selectfont\dozent \\ \blattname}

% Adjust the font size for the footer
\firstpagefooter{\includegraphics[width=2.5cm]{bauschule-logo-5cm.png}}{}{\fontsize{9}{11}\selectfont\thepage\,/\,\pageref{LastPage}}
\runningfooter{\includegraphics[width=2.5cm]{bauschule-logo-5cm.png}}{}{\fontsize{9}{11}\selectfont\thepage\,/\,\pageref{LastPage}}

\newcommand{\blattname}{Physik: Dichte, Masse, Kraft}

%% header and footer

\pagestyle{headandfoot}
\firstpageheadrule
\runningheadrule
\firstpageheader{\fach}{}{\fontsize{9pt}{2pt}\selectfont \dozent \\ \blattname}
\runningheader{\fach}{}{\fontsize{9pt}{2pt}\selectfont\dozent \\ \blattname}
\firstpagefooter{\includegraphics[width=2.5cm]{../../../../template/bauschule-logo-5cm.png}}{}{\fontsize{9pt}{2pt}\selectfont \thepage\,/\,\numpages}
\runningfooter{\includegraphics[width=2.5cm]{../../../../template/bauschule-logo-5cm.png}}{}{\fontsize{9pt}{2pt}\selectfont \thepage\,/\,\numpages}


%\printanswers

\begin{document}

{\fontsize{22pt}{2pt}\selectfont \textbf{\blattname}}
\vspace{0.3cm}

\renewcommand{\questionshook}{%
  %\setlength{\leftmargin}{0pt}% removes the indentation from the left
  \setlength{\labelwidth}{1.25cm}% adjusts label width
  \setlength{\itemindent}{0cm}% aligns the start of the item with the above
  \setlength{\labelsep}{0.25cm}% space between the label and the item text
}

\begin{questions}

\question Berechnen Sie die Masse eines Betonblocks mit den Abmessungen \SI{2}{\meter} x \SI{3}{\meter} x \SI{0.5}{\meter} und einer Dichte von \SI{2400}{\kilogram\per\cubic\meter}.
\begin{solution}
    Volumen des Blocks = \(2 \times 3 \times 0.5 = \SI{3}{\cubic\meter}\). \\
    Masse = Volumen x Dichte = \(3 \times 2400 = \SI{7200}{\kilogram}\).
\end{solution}

\question Ein Stahlträger mit einer Masse von \SI{1200}{\kilogram} wird gleichmässig von zwei Seilen gehalten. Welche Kraft wirkt auf jedes Seil?
\begin{solution}
    Gewichtskraft = Masse x Erdbeschleunigung (g = \SI{9.81}{\meter\per\square\second}). \\
    Gesamtkraft = \(1200 \times 9.81 = \SI{11772}{\newton}\). \\
    Kraft pro Seil = \(11772 / 2 = \SI{5886}{\newton}\).
\end{solution}

\question Ein Holzbalken mit einer Dichte von \SI{600}{\kilogram\per\cubic\meter} hat ein Volumen von \SI{0.8}{\cubic\meter}. Bestimmen Sie seine Masse.
\begin{solution}
    Masse = Dichte x Volumen = \(600 \times 0.8 = \SI{480}{\kilogram}\).
\end{solution}

\question Ein Bauarbeiter hält einen \SI{25}{\kilogram} schweren Ziegelstein an einem Seil. Welche Kraft muss er aufwenden, um den Stein zu halten?
\begin{solution}
    Die aufzuwendende Kraft entspricht der Gewichtskraft des Steins: \\
    Gewichtskraft = Masse x g = \(25 \times 9.81 = \SI{245.25}{\newton}\).
\end{solution}

\question Ein Betonpfeiler hat ein Volumen von \SI{1.5}{\cubic\meter} und eine Masse von \SI{3600}{\kilogram}. Berechnen Sie die Dichte des Betons.
\begin{solution}
    Dichte = Masse / Volumen = \(3600 / 1.5 = \SI{2400}{\kilogram\per\cubic\meter}\).
\end{solution}

\question Ein Kran hebt eine Last von \SI{1500}{\kilogram} mit einer Beschleunigung von \SI{0.5}{\meter\per\square\second}. Berechnen Sie die Kraft, die der Kran ausübt.
\begin{solution}
    Gesamtkraft = Masse x (g + Beschleunigung) = \(1500 \times (9.81 + 0.5) = \SI{15465}{\newton}\).
\end{solution}

\question Ein Zylinder aus Aluminium (Dichte = \SI{2700}{\kilogram\per\cubic\meter}) hat einen Radius von \SI{0.1}{\meter} und eine Höhe von \SI{0.5}{\meter}. Ermitteln Sie seine Masse.
\begin{solution}
    Volumen des Zylinders = \(\pi \times r^2 \times h = \pi \times 0.1^2 \times 0.5 \approx \SI{0.0157}{\cubic\meter}\). \\
    Masse = Dichte x Volumen = \(2700 \times 0.0157 \approx \SI{42.39}{\kilogram}\).
\end{solution}

\question Berechnen Sie die Gewichtskraft einer Granitplatte (Dichte = \SI{2750}{\kilogram\per\cubic\meter}) mit den Massen \SI{2}{\meter} x \SI{1}{\meter} x \SI{0.05}{\meter}.
\begin{solution}
    Volumen der Platte = \(2 \times 1 \times 0.05 = \SI{0.1}{\cubic\meter}\). \\
    Masse = Dichte x Volumen = \(2750 \times 0.1 = \SI{275}{\kilogram}\). \\
    Gewichtskraft = Masse x g = \(275 \times 9.81 = \SI{2697.75}{\newton}\).
\end{solution}

\question Eine Ziegelmauer hat ein Volumen von \SI{2}{\cubic\meter} und eine Masse von \SI{4000}{\kilogram}. Bestimmen Sie die Dichte der Ziegel.
\begin{solution}
    Dichte = Masse / Volumen = \(4000 / 2 = \SI{2000}{\kilogram\per\cubic\meter}\).
\end{solution}

\question Ein Stahlseil soll eine Last von \SI{2000}{\kilogram} tragen. Wenn die maximale Zugkraft des Seils \SI{5000}{\newton} beträgt, ist das Seil für diese Aufgabe geeignet? Begründen Sie Ihre Antwort.
\begin{solution}
    Gewichtskraft der Last = Masse x g = \(2000 \times 9.81 = \SI{19620}{\newton}\). \\
    Da \SI{19620}{\newton} > \SI{5000}{\newton}, ist das Seil nicht geeignet.
\end{solution}

\end{questions}
\end{document}

