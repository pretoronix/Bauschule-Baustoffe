% !TEX root = /Users/patricpf/Documents/repos/Bauschule-Baustoffe/Unterlagen/05_Beton/Mischungsbrechnung/Mischungsberechnung_Mittel.tex

% !TEX program = lualatex
\documentclass[
    %answers,
    a4paper,ngerman,12pt, addpoints]{exam}

\usepackage[utf8]{inputenc}
%\usepackage[T1]{fontenc}
%\usepackage[ngerman]{babel}


\usepackage{polyglossia}
\setdefaultlanguage[variant = swiss]{german}
\usepackage{fontspec}
\setmainfont{Aptos} % Bauschule CI Manual
\setsansfont{Aptos} % Bauschule CI Manual


\usepackage[ a4paper,
 total={165mm,250mm},
 left=25mm,
 top=25mm,
 headsep=10mm
 %footsep=12mm
 %,showframe
  ]{geometry}

\usepackage{graphicx}
\usepackage{siunitx}
\usepackage{booktabs} % schöne Tabellen
\usepackage{float}
\floatplacement{figure}{H}
\usepackage{xcolor}
\usepackage{pdfpages}
\usepackage{enumitem}
\usepackage{mdframed} % Boxen
\usepackage{amsmath,amssymb}
\usepackage{tcolorbox}
\usepackage{lastpage} % For the total number of pages
\usepackage{gensymb}
\usepackage{xspace}
\usepackage{tabularx}
\usepackage{multicol}
\usepackage[
    version=3,
    arrows=pgf-filled,
]{mhchem} % für chemische Formeln
%\usepackage{microtype}
\usepackage{subfigure}
\usepackage[hidelinks]{hyperref}
\usepackage{cleveref}
\usepackage{luacode}
\usepackage{amsmath}
\usepackage{textcomp}


\sisetup{
  locale = DE,
  inter-unit-product = \ensuremath{{\cdot}},
  detect-all,
}

% Colors
\definecolor{blau_bauschule}{RGB}{22,65,148}
\CorrectChoiceEmphasis{\color{blau_bauschule}}
\SolutionEmphasis{\color{blau_bauschule}}

\setlength{\parindent}{0em} % Verhindert einrücken
\setlength\linefillheight{0.3in}


%% COMMMANDS
\author{Patrick Pfändler}
\newcommand{\dozent}{Patrick Pfändler}
\newcommand{\fach}{Baustoffe}


\newcommand{\punkte}[1]{%
    \begin{infobox}%
        #1
    \end{infobox}}%
\newcommand{\FinRes}[1]{\underline{\underline{#1}}}

\newmdenv[linecolor=black,backgroundcolor=gray!15,frametitle={Punktverteilung},leftmargin=1cm,rightmargin=1cm]{infobox}

\newcommand{\pagebreaksol}{
    \ifprintanswers
        \clearpage
    \else
        {}
    \fi
}

\newcommand{\pagebreakexam}{
    \ifprintanswers
        {}
    \else
        \clearpage
    \fi
}

\SolutionEmphasis{\color{blau_bauschule}}
\makeatletter%
\newcommand{\solutiontable}[1]{\ifprintanswers\begingroup\Solution@Emphasis#1\if@shadedsolutions%
            {\cellcolor{SolutionColor}}%
        \else%
        \fi\endgroup\else\phantom{#1}\fi}%
\makeatother%

\newcommand{\myNmm}[1]
{
    \sisetup{per-mode=symbol}
    \SI{#1}{\newton\per\mm\squared}
}

\renewcommand{\thequestion}{\fontsize{12pt}{2pt} \selectfont  \bfseries \arabic{question}}
\sisetup{per-mode=symbol}



%% Translation

\pointpoints{Punkt}{Punkte}
\bonuspointpoints{Bonuspunkt}{Bonuspunkte}
\renewcommand{\solutiontitle}{\noindent\textbf{Lösung:}\enspace}
\chqword{Frage}
\chpgword{Seite}
\chpword{Punkte}
\chbpword{Bonus Punkte}
\chsword{Erreicht}
\chtword{Gesamt}
\hpword{Punkte:}
\hsword{Ergebnis:}
\hqword{Aufgabe:}
\htword{Summe:}


\renewcommand{\questionshook}{%
  %\setlength{\leftmargin}{0pt}% removes the indentation from the left
  \setlength{\labelwidth}{1.25cm}% adjusts label width
  \setlength{\itemindent}{0cm}% aligns the start of the item with the above
  \setlength{\labelsep}{0.25cm}% space between the label and the item text
}




%% header and footer
\pagestyle{headandfoot}
\firstpageheadrule
\runningheadrule

% Adjust the font size for the header
\firstpageheader{\fontsize{9}{11}\selectfont\fach}{}{\fontsize{9}{11}\selectfont\dozent \\ \blattname}
\runningheader{\fontsize{9}{11}\selectfont\fach}{}{\fontsize{9}{11}\selectfont\dozent \\ \blattname}

% Adjust the font size for the footer
\firstpagefooter{\includegraphics[width=2.5cm]{bauschule-logo-5cm.png}}{}{\fontsize{9}{11}\selectfont\thepage\,/\,\pageref{LastPage}}
\runningfooter{\includegraphics[width=2.5cm]{bauschule-logo-5cm.png}}{}{\fontsize{9}{11}\selectfont\thepage\,/\,\pageref{LastPage}}


\newcommand{\blattname}{Beton: Mittlere Mischungsberechnung}



%% header and footer
\pagestyle{headandfoot}
\firstpageheadrule
\runningheadrule
\firstpageheader{\fach}{}{\fontsize{9pt}{2pt}\selectfont \dozent \\ \blattname}
\runningheader{\fach}{}{\fontsize{9pt}{2pt}\selectfont\dozent \\ \blattname}
\firstpagefooter{\includegraphics[width=2.5cm]{/Users/patricpf/Documents/repos/Bauschule-Baustoffe/template/bauschule-logo-5cm.png}}{}{\fontsize{9pt}{2pt}\selectfont \thepage\,/\,\numpages}
\runningfooter{\includegraphics[width=2.5cm]{/Users/patricpf/Documents/repos/Bauschule-Baustoffe/template/bauschule-logo-5cm.png}}{}{\fontsize{9pt}{2pt}\selectfont \thepage\,/\,\numpages}

%\printanswers



\begin{document}


{\fontsize{22pt}{2pt}\selectfont \textbf{\blattname}}

\begin{questions}
    \question
    Die Anforderungen an die Mischung sind: 
\begin{itemize}[noitemsep]
    \item \SI{120}{\l} Mischung
    %\item Druckfestigkeitsklasse C20/25 
    \item Grösstkorn der Gesteinskörnung $D_{max}= \SI{32}{\mm}$
    \item $\dfrac{w}{z} \leqslant 0.5$
    %\item Konsistenzklasse F3
    \item $\rho_{\text{Zement}} = 3050 kg/m^3$ mit Mindestzementgehalt von 300 kg/m$^3$.
    \item \textbf{Materialien:} feine Gesteinskörnung 0/2, feine Gesteinskörnung 0/4, grobe Gesteinskörnung  4/8, grobe Gesteinskörnung 8/16 und grobe Gesteinskörnung 16/32 mit einer Rohdichte von je $ 2680 kg/m^3$.
\end{itemize}
\vspace{0.5cm}
Berechnen Sie die fehlenden Angaben in der Aufgabenstellung.

\question Dosieren Sie Pneumatit für die Mischung. Der Pneumatitgehalt beträgt \SI{125}{\ml} pro Kubikmeter Beton.
\end{questions}





\begin{solution}
\begin{enumerate}
    \item Der Mindestzementgehalt in diesem Fall beträgt $Z_{min} = 300 kg/m^3$.
    \item Herauslesen des maximalen Wasser-Zement-Wertes. Der maximale Wasser-Zement-Wert beträgt 0.50.
    \item Erkennen, dass \SI{120}{\l} Beton gewünscht sind.
    \item Berechnen der Zementmenge:
    \begin{equation*}
        m_{Zement} = \SI{0.120}{\m^3} \cdot \SI{300}{\kg\per\m^3} = \FinRes{\SI{36}{\kg}}
    \end{equation*}
    
    \item Berechnen der Wassermenge für \SI{120}{\l} Mischung
    
    \begin{equation*}
        m_{Wasser} = 0.5 \cdot m_{Zement} = 0.5 \cdot \SI{36}{\kg}  = \FinRes{\SI{18}{\kg}}
    \end{equation*}
    
    \item Herauslesen des Luftporengehaltes. Annahme: 1.0\%. Ergibt 10 l Luft pro Kubikmeter Beton.
    \begin{equation*}
        \SI{0.120}{\m^3} \cdot \SI{10}{\l\per\m^3} = \SI{1.2}{\l}
    \end{equation*}
    \item Bestimmen des Volumes der Gesteinskörnung [$1 m^3 = 1000 l $]:
    \begin{equation*}
        \SI{120}{\l} - \dfrac{\SI{36}{\kg}}{\SI{3.05}{\kg\per\dm^3}} - \dfrac{\SI{18}{\kg}}{\SI{1.0}{\kg\per\dm^3}}-\SI{1.2}{\l} = \FinRes{\SI{89}{\l}}
    \end{equation*}
    
    \item Umrechnen des Volumens in Kilogramm
    \begin{equation*}
        m_{\text{Gesteinskörnung}} = \SI{89}{\l} \cdot \SI{2.68}{\kg\per\dm^3} = \FinRes{\SI{238.51}{\kg}}
    \end{equation*}
    
    
    \item Bestimmen der Massen der Gesteinskörnung aus der Grafik. (Hinweis: Kurve mit der gewünschten Korngrösse verwenden.)
    
    \begin{equation*}
        \text{für 0/2: }   12\% \cdot \SI{238.51}{\kg}  = \SI{28.6213}{\kg}
    \end{equation*}
    
    \begin{equation*}
        \text{für 0/4: }   24\% \cdot \SI{238.51}{\kg}  = \SI{57.243}{\kg}
    \end{equation*}
    
    \begin{equation*}
        \text{für 4/8: }   8\% \cdot \SI{238.51}{\kg}  = \SI{19.08}{\kg}
    \end{equation*}
    
    \begin{equation*}
        \text{für 8/16: }   24\% \cdot \SI{238.51}{\kg}  = \SI{57.24}{\kg}
    \end{equation*}

    \begin{equation*}
        \text{für 16/32: }   32\% \cdot \SI{238.51}{\kg}  = \SI{76.32}{\kg}
    \end{equation*}
    

    
    \item Die Rohdichte des Frischbetons berechnen: 
    
    \begin{eqnarray*}
        \text{Rohdichte}=(m_{Zement} + m_{Wasser} + m_{\text{Gesteinskörnung}} + m_{Visco})/ \SI{0.120}{\m^3} \\
        = (18+36+238.51 +0.144 \si{\kg})/( \SI{0.120}{\m^3})  \\
        = \FinRes{\SI{2438.8}{\kg\per\m^3}}
    \end{eqnarray*}
    
    \item Pneumatit (\SI{125}{\ml} für \SI{1}{m^3} Beton)
    
    \begin{equation*}
        \text{für 120 l Beton:  } 0.120 \cdot \SI{125}{\ml} = \FinRes{\SI{15}{\ml}}
    \end{equation*}
    
    
    
\end{enumerate}
\end{solution}


\end{document}