%!TEX root = /Users/patricpf/Documents/repos/Bauschule-Baustoffe/Unterlagen/05_Beton/Mischungsbrechnung/Mischungsberechng_Folien.tex
%!TEX program = lualatex
\def\customoptions{aspectratio=169} % Exclude handout
% !TEX program = lualatex
\documentclass[aspectratio=169, 
handout,
]{beamer}


\PassOptionsToPackage{dvipsnames,svgnames}{xcolor}

%\usepackage[utf8]{inputenc}
%\usepackage[ngerman]{babel} % Schweizer Rechtschreibung

\usepackage{polyglossia}
\setdefaultlanguage[variant = swiss]{german}
\usepackage{fontspec}
\setmainfont{Times New Roman}
\setsansfont{Arial}


\usepackage{amsmath}
\usepackage{siunitx}
\sisetup{
  locale = DE,
  inter-unit-product = \ensuremath{{\cdot}},
  detect-mode,             % Use the surrounding text font mode
  detect-family,           % Use the surrounding text font family
  detect-weight,           % Use the surrounding text font weight
  mode = text,             % Ensure that numbers and units are typeset in text mode
  %negative-powers = false, % Avoid using negative powers
  per-mode = symbol        % Use the division symbol for per units
}

\usepackage{tikz}
\usepackage{enumitem}
\usepackage{graphicx}
\usepackage{booktabs}
\usepackage{calc}
\usepackage{multicol}
\usepackage{amsmath}

\usepackage{tcolorbox}
\tcbuselibrary{skins}

\usepackage[dvipsnames]{xcolor}

%\usepackage[scaled]{helvet} % Arial-ähnliche Schriftart (Helvetica)
%\renewcommand\familydefault{\sfdefault} % Setzt die Standard-Schriftart auf sans-serif


\usetheme{Madrid} % This theme is visually appealing
%\usecolortheme{whale} % A color theme with blue tones
\usecolortheme{dolphin} % A color theme with blue tones


\newcommand{\mylogo}{
    \begin{tikzpicture}[remember picture,overlay]
        \node[anchor=north east, yshift=-2mm, fill=white, inner sep=2pt] at (current page.north east) % Verschiebe das Logo um 5mm nach unten
            {\includegraphics[height=0.4cm]{/Users/patricpf/Documents/repos/Bauschule-Baustoffe/template/bauschule-logo-5cm.png}}; % Grösse nach Bedarf anpassen
    \end{tikzpicture}
}

\logo{\mylogo}

\setbeamertemplate{headline}{%
  \begin{beamercolorbox}[wd=\textwidth,ht=0.5ex,dp=1ex]{upper separation line head}
  \end{beamercolorbox}
}

\setbeamercolor{upper separation line head}{bg=blau_bauschule}


% Anpassen des Frametitels, um ihn fett zu machen
\setbeamertemplate{frametitle}{%
    \nointerlineskip%
    \begin{beamercolorbox}[sep=0.3cm,left,wd=\paperwidth]{frametitle}%
        \usebeamerfont{frametitle}\bfseries\insertframetitle%
    \end{beamercolorbox}%
}

%\usebackgroundtemplate{
%    \includegraphics[width=\paperwidth,height=\paperheight]{my_pdf_copy_of_empty_ppt_template}
%}

% Setzen Sie hier den Namen des Fachs
\newcommand{\fachname}{Baustoffe}
\newcommand{\FinRes}[1]{\underline{\underline{#1}}}

% Anpassen der Fusszeile für Abstände zum Rand
\setbeamertemplate{footline}
{
  \leavevmode%
  \hbox{%
  \begin{beamercolorbox}[wd=.33\paperwidth,ht=2.25ex,dp=1ex,left,leftskip=1em]{author in head/foot}%
    \usebeamerfont{author in head/foot}\insertshortauthor
  \end{beamercolorbox}%
  \begin{beamercolorbox}[wd=.34\paperwidth,ht=2.25ex,dp=1ex,center]{title in head/foot}%
    \usebeamerfont{title in head/foot}\fachname % Hier wird der Fachname anstelle des Titels angezeigt
  \end{beamercolorbox}%
  \begin{beamercolorbox}[wd=.33\paperwidth,ht=2.25ex,dp=1ex,right,rightskip=1em]{date in head/foot}%
    \usebeamerfont{date in head/foot}\insertframenumber{} / \inserttotalframenumber\hspace*{2ex}
  \end{beamercolorbox}}%
  \vskip0pt%
}


\newtcolorbox{Merke}{
enhanced,
boxrule=0pt,frame hidden,
borderline west={4pt}{0pt}{red!75!black},
colback=white,
sharp corners,
before upper={\textbf{Merke:}\quad},
}

\newtcolorbox{Anwendungen}{
enhanced,
boxrule=0pt,frame hidden,
borderline west={4pt}{0pt}{brown!75!black},
colback=white,
sharp corners
}




% Colors
\definecolor{blau_bauschule}{RGB}{22,65,148}
\setbeamercolor{frametitle}{fg=blau_bauschule}
\setbeamertemplate{navigation symbols}{} % Remove navigation symbols


\newtcolorbox{Definition_BS}[1]{
enhanced,boxrule=1pt,
colback=green!5!white,
colframe=green!75!black,fonttitle=\bfseries, title = #1,
%after title={\hfill\colorbox{black}{Definition}}
}



\newtcolorbox{Masseinheit}[1]{
enhanced,
boxrule=1pt,colframe=blue,
colback=white,
sharp corners, 
colframe=blue!75!black,
title = #1, 
after title={\hfill\colorbox{blue}{Masseinheit}}
}


\newtcolorbox{myLösung}{
  enhanced,
  boxrule=1pt,
  colframe=gray!75!black, % Definiert die Farbe des Rahmens als dunkelgrau
  colback=gray!20, % Definiert die Hintergrundfarbe der Box als hellgrau
  %sharp corners, % Macht die Ecken der Box scharf (nicht abgerundet)
  title = {Lösung}, % Fest eingestellter Titel der Box
  after title={}, % Fügt das Label "Masseinheit" nach dem Titel hinzu
  coltitle=white, % Farbe des Titeltexts
  fonttitle=\bfseries % Schriftart des Titels
}


% Set the title, author, and date
\title{\textbf{Beispiel Stoffraumberechnung}}
\author{Patrick Pfändler}
\date{{}}


\begin{document}
\frame{\titlepage}

\section{Theorie}
\BlueSectionSlide

\begin{frame}{Theoretische Grundlagen}
    \begin{Definition_BS}{Stoffraumrechnung}
        Die Stoffraumrechnung dient zur Ermittelung der Raumanteile der einzelnen Ausgangsstoffe an einem Kubikmeter verdichtem Frischbeton.
    \end{Definition_BS}
    

\end{frame}


\begin{frame}{Zusammensetzung eines  Betons}

    \begin{equation*}
        \text{Beton} = \text{Zement} + \text{Wasser} + \text{Gesteinskörnung} + \text{Luftporen} + \text{Zusatzstoffe}
    \end{equation*}

\end{frame}




\begin{frame}{Theoretische Grundlagen: Stoffraumrechnung}
    \begin{equation*}
        1 m^3 = \frac{z}{\rho_z} + \frac{w}{\rho_w} + \frac{g}{\rho_g} + \frac{f}{\rho_f} + p
    \end{equation*}
    , wobei 
    
    \begin{description}[leftmargin=!,labelwidth=\widthof{Abstand  },font=\normalfont]
        \item[$z, \rho_z$] Masse  bzw. Dichte des Zements 
        \item[$w, \rho_w$] Masse  bzw. Dichte des Wasssers 
        \item[$g, \rho_g$] Masse  bzw. Rohdichte der Gesteinskörnung
        \item[$f, \rho_f$] Masse  bzw. Dichte des Zussatzstoffes oder Zusatzmittels
        %\item[$f, \rho_f$] Masse  bzw. Dichte des Zussatzmittel
        \item[$p$] Porenvolumen (Lufgehalt)
    \end{description}

\end{frame}


\begin{frame}{Theoretische Grundlagen}

    \begin{equation*}
        V_\text{total} = V_{\text{Wasser}} + V_{\text{Zement}} + V_{\text{Luftporen}} + V_{\text{Gesteinskörnung}} = \SI{1}{m^3}
    \end{equation*}

\end{frame}

\begin{frame}{Warum Stoffraumrechnung?}
    \begin{itemize}
        \item Ermittlung der Mischungsverhältnisse
        \item Ermittlung der Masseanteile
        \item Ermittlung der Volumenanteile
    \end{itemize}
    \pause
    \vspace{1cm}
    Massen lasssen sich einfacher messen als Volumen.

\end{frame}







\section{Aufgabe}
\BlueSectionSlide

\begin{frame}{Aufgabe}
    \begin{table}[h]
        \centering
        \caption{Aufgabe zur Stoffraumberechnung}
        \small
        \resizebox{\textwidth}{!}{%
        \begin{tabular}{lrrr}
        \toprule
        \textbf{Betonkomponenten}       & \textbf{Gehalt [kg/m³]} & \textbf{Dichte [kg/dm³]} & \textbf{Stoffraum [l/m³]} \\ 
        \midrule
        Zement                           & 320                      & 3.1                       &    {}          \\
        Gesamtwasser                     & 157                      & 1.0                       & {}            \\
        Luftporen (2.0\%)                & -                        & -                         & {}              \\
        Zementleimvolumen                &                          &                           &                           \\
        trockene Gesteinskörnung 0–32 mm & {}                     & 2.7                       & {}             \\
        \textbf{total}                   & {}           &                           & {}             \\
        \bottomrule
        \end{tabular}
        }
        \label{tab:Stoffraumberechnung}
        \end{table}

\end{frame}

\section{Lösung}
\BlueSectionSlide

\begin{frame}{Dichte}
    Falls die Angaben zur Dichte fehlen.
    \begin{itemize}
        \item Die Dichten sind jeweils angegeben oder stehen im Skript. (z.B. Gesteinskörnung)
    \end{itemize}
\end{frame}

\begin{frame}{Stoffraum von Zement (i)}
    \begin{table}[h]
        \centering
        \caption{Aufgabe zur Stoffraumberechnung}
        \small
        \resizebox{\textwidth}{!}{%
        \begin{tabular}{lrrc}
        \toprule
        \textbf{Betonkomponenten}       & \textbf{Gehalt [kg/m³]} & \textbf{Dichte [kg/dm³]} & \textbf{Stoffraum [l/m³]} \\ 
        \midrule
        Zement                           & 320                      & 3.1                       &    \textcolor{red}{???}          \\
        Gesamtwasser                     & 157                      & 1.0                       & {}            \\
        Luftporen (2.0\%)                & -                        & -                         & {}              \\
        Zementleimvolumen                &                          &                           &                           \\
        trockene Gesteinskörnung 0–32 mm & {}                    & 2.7                       & {}             \\
        \textbf{total}                   & {}            &                           & {}             \\
        \bottomrule
        
    \end{tabular}
        }
    \label{tab:Stoffraumberechnung}
        \end{table}

\end{frame}

\begin{frame}{Stoffraum von Zement (ii)}

    \begin{equation*}
        \rho = \dfrac{m}{V}
    \end{equation*}
    \pause



    \begin{equation*}
        V_{\text{Zement}} = \dfrac{m_{\text{Zement}}}{\rho_{\text{Zement}}}
    \end{equation*}

    \pause
    \vspace{0.5cm}
    \begin{equation*}
        V_{\text{Zement}} = \dfrac{320}{3.1} = 103.23 \, \text{l/m³}
    \end{equation*}
\end{frame}



\begin{frame}{Stoffraum von Zement (i)}
    \begin{table}[h]
        \centering
        \caption{Aufgabe zur Stoffraumberechnung}
        \small
        \resizebox{\textwidth}{!}{%
        \begin{tabular}{lrrc}
        \toprule
        \textbf{Betonkomponenten}       & \textbf{Gehalt [kg/m³]} & \textbf{Dichte [kg/dm³]} & \textbf{Stoffraum [l/m³]} \\ 
        \midrule
        Zement                           & 320                      & 3.1                       &   103.23 \\
        Gesamtwasser                     & 157                      & 1.0                       & {}            \\
        Luftporen (2.0\%)                & -                        & -                         & {}              \\
        Zementleimvolumen                &                          &                           &                           \\
        trockene Gesteinskörnung 0–32 mm & {}                     & 2.7                       & {}             \\
        \textbf{total}                   & {}           &                           & {}             \\
        \bottomrule
    \end{tabular}
        }
    \label{tab:Stoffraumberechnung}
        \end{table}

\end{frame}

\begin{frame}{Wasseranteil}
    \begin{equation*}
        V_{\text{Wasser}} = \dfrac{m_{\text{Wasser}}}{\rho_{\text{Wasser}}}
    \end{equation*}

    \pause
    \vspace{0.5cm}
    \begin{equation*}
        V_{\text{Wasser}} = \dfrac{157}{1} = 157 \, \text{l/m³}
    \end{equation*}

\end{frame}

\begin{frame}{Wasser-Zement-Wert (i)}
    \begin{equation*}
        w/z = \dfrac{V_{\text{Wasser}}}{V_{\text{Zement}}}
    \end{equation*}

    \pause
    \vspace{0.5cm}
    \begin{equation*}
        w/z = \dfrac{157}{320} = 0.49
    \end{equation*}

\end{frame}

\begin{frame}{Stoffraum der Luftporen (i)}
    \begin{table}[h]
        \centering
        \caption{Aufgabe zur Stoffraumberechnung}
        \small
        \resizebox{\textwidth}{!}{%
        \begin{tabular}{lrrc}
        \toprule
        \textbf{Betonkomponenten}       & \textbf{Gehalt [kg/m³]} & \textbf{Dichte [kg/dm³]} & \textbf{Stoffraum [l/m³]} \\ 
        \midrule
        Zement                           & 320                      & 3.1                       &    103.23          \\
        Gesamtwasser                     & 157                      & 1.0                       &  157           \\
        Luftporen (2.0\%)                & -                        & -                         & \textcolor{red}{???}              \\
        Zementleimvolumen                &                          &                           &                           \\
        trockene Gesteinskörnung 0–32 mm & 1944                     & 2.7                       & {}             \\
        \textbf{total}                   & {}           &                           & {}             \\
        \bottomrule
    \end{tabular}
        }
    \label{tab:Stoffraumberechnung}
        \end{table}
\end{frame}






\begin{frame}{Luftporengehalt}
    2\% Luftporengehalt
    \vspace{1cm}
    \pause
    \begin{equation*}
        p = 2\%
    \end{equation*}
    \pause
    \begin{equation*}
        V_{\text{Luftporen}} = 0.02 \cdot 1000 = 20 \, \text{l/m³}
    \end{equation*}
    
\end{frame}

\begin{frame}{Zemeintleimvolumen (i)}

    \begin{equation*}
        V_{\text{Zemeintleimvolumen}} = V_{\text{Wasser}} + V_{\text{Zement}} + V_{\text{Luftporen}}
    \end{equation*}
    \pause
    
    \begin{equation*}
        V_{\text{Zemeintleimvolumen}} = 157 + 103.23 + 20 = 280.23 \, \text{l/m³}
    \end{equation*}

\end{frame}

\begin{frame}{Zemeintleimvolumen (ii)}
    \begin{table}[h]
        \centering
        \caption{Aufgabe zur Stoffraumberechnung}
        \small
                \resizebox{\textwidth}{!}{%
        \begin{tabular}{lrrc}
        \toprule
        \textbf{Betonkomponenten}       & \textbf{Gehalt [kg/m³]} & \textbf{Dichte [kg/dm³]} & \textbf{Stoffraum [l/m³]} \\ 
        \midrule
        Zement                           & 320                      & 3.1                       &    103.23          \\
        Gesamtwasser                     & 157                      & 1.0                       &  157            \\
        Luftporen (2.0\%)                & -                        & -                         & 20              \\
        Zementleimvolumen                &                          &                           & 280.23                          \\
        trockene Gesteinskörnung 0–32 mm & {}                     & 2.7                       & {}           \\
        \textbf{total}                   & {}           &                           & {}             \\
        \bottomrule
    \end{tabular}
    }
    \label{tab:Stoffraumberechnung}
        \end{table}

\end{frame}

\begin{frame}{Gesteinskörnung (i)}
    \begin{table}[h]
        \centering
        \caption{Aufgabe zur Stoffraumberechnung}
        \small
                \resizebox{\textwidth}{!}{%
        \begin{tabular}{lrrc}
        \toprule
        \textbf{Betonkomponenten}       & \textbf{Gehalt [kg/m³]} & \textbf{Dichte [kg/dm³]} & \textbf{Stoffraum [l/m³]} \\ 
        \midrule
        Zement                           & 320                      & 3.1                       &    103.23          \\
        Gesamtwasser                     & 157                      & 1.0                       &  157            \\
        Luftporen (2.0\%)                & -                        & -                         & 20              \\
        Zementleimvolumen                &                          &                           & 280.23                          \\
        trockene Gesteinskörnung 0–32 mm & {}                     & 2.7                       & \textcolor{red}{???}              \\
        \textbf{total}                   & {}         &                           & {}             \\
        \bottomrule
    \end{tabular}
    }
    \label{tab:Stoffraumberechnung}
        \end{table}

\end{frame}


\begin{frame}{Gesteinskörnung (ii)}

    \begin{equation*}
        V_\text{total} = V_{\text{Wasser}} + V_{\text{Zement}} + V_{\text{Luftporen}} + V_{\text{Gesteinskörnung}}
    \end{equation*}

    \pause
    \vspace{1cm}
    Einsetzen der bekannten Werte:
    \begin{equation*}
        \SI{1000}{\liter} = \SI{157}{\liter} + \SI{103.23}{\l} + \SI{20}{\liter} + V_{\text{Gesteinskörnung}}
    \end{equation*}
    \pause
    \vspace{1cm}
    Auflösen nach $V_{\text{Gesteinskörnung}}$:
    \begin{equation*}
        V_{\text{Gesteinskörnung}} = 1000 - 157 - 103.23 - 20 = 719.77 \, \text{l/m³}
    \end{equation*}

\end{frame}

\begin{frame}{Gehalt der Gesteinskörnung (iii)}

    \begin{equation*}
        m_{\text{Gesteinskörnung}} = \rho_{\text{Gesteinskörnung}} \cdot V_{\text{Gesteinskörnung}}
    \end{equation*}

    \pause
    \vspace{1cm}
    \begin{equation*}
        m_{\text{Gesteinskörnung}} = 2.7 \cdot 719.77 = 1944 \, \text{kg/m³}
    \end{equation*}

\end{frame}


\begin{frame}{Gesteinskörnung (iv)}
    \begin{table}[h]
        \centering
        \caption{Aufgabe zur Stoffraumberechnung}
        \small
                \resizebox{\textwidth}{!}{%
        \begin{tabular}{lrrc}
        \toprule
        \textbf{Betonkomponenten}       & \textbf{Gehalt [kg/m³]} & \textbf{Dichte [kg/dm³]} & \textbf{Stoffraum [l/m³]} \\ 
        \midrule
        Zement                           & 320                      & 3.1                       &    103.23          \\
        Gesamtwasser                     & 157                      & 1.0                       &  157            \\
        Luftporen (2.0\%)                & -                        & -                         & 20              \\
        Zementleimvolumen                &                          &                           & 280.23                          \\
        trockene Gesteinskörnung 0–32 mm & 1944                     & 2.7                       & 720              \\
        \textbf{total}                   & {}        &                           & {}             \\
        \bottomrule
    \end{tabular}
    }
    \label{tab:Stoffraumberechnung}
        \end{table}

\end{frame}
\begin{frame}{Dichte der Betonmischung (i)}
    \begin{table}[h]
        \centering
        \caption{Aufgabe zur Stoffraumberechnung}
        \small
                \resizebox{\textwidth}{!}{%
        \begin{tabular}{lrrc}
        \toprule
        \textbf{Betonkomponenten}       & \textbf{Gehalt [kg/m³]} & \textbf{Dichte [kg/dm³]} & \textbf{Stoffraum [l/m³]} \\ 
        \midrule
        Zement                           & 320                      & 3.1                       &    103.23          \\
        Gesamtwasser                     & 157                      & 1.0                       &  157            \\
        Luftporen (2.0\%)                & -                        & -                         & 20              \\
        Zementleimvolumen                &                          &                           & 280.23                          \\
        trockene Gesteinskörnung 0–32 mm & 1944                     & 2.7                       & 720              \\
        \textbf{total}                   & \textcolor{red}{???}         &                           & {}             \\
        \bottomrule
    \end{tabular}
    }
    \label{tab:Stoffraumberechnung}
        \end{table}

\end{frame}

\begin{frame}{Dichte der Betonmischung (ii)}

    \begin{eqnarray*}
        M_{\text{total}} &=& m_{\text{Zement}} + m_{\text{Wasser}} + m_{\text{Gesteinskörnung}} \\
    \end{eqnarray*}
    \pause
    \begin{eqnarray*}
        M_{\text{total}} &=& 320 + 157 + 1944 = 2421 \, \text{kg/m³}
    \end{eqnarray*}
    \pause
    \textit{Hinweis:} Masse der Luftporen wird hier nicht berücksichtigt.


\end{frame}

\begin{frame}{Lösung}
    \begin{table}[h]
        \centering
        \caption{Aufgabe zur Stoffraumberechnung}
        \small
        \resizebox{\textwidth}{!}{%
        \begin{tabular}{lrrc}
        \toprule
        \textbf{Betonkomponenten}       & \textbf{Gehalt [kg/m³]} & \textbf{Dichte [kg/dm³]} & \textbf{Stoffraum [l/m³]} \\ 
        \midrule
        Zement                           & 320                      & 3.1                       &    103.23          \\
        Gesamtwasser                     & 157                      & 1.0                       &  157            \\
        Luftporen (2.0\%)                & -                        & -                         & 20              \\
        Zementleimvolumen                &                          &                           & 280.23                          \\
        trockene Gesteinskörnung 0–32 mm & 1944                     & 2.7                       & 720              \\
        \textbf{total}                   & 2421         &                           & {}             \\
        \bottomrule
    \end{tabular}
        }
    \label{tab:Stoffraumberechnung}
        \end{table}

\end{frame}

\begin{frame}{Kontrolle}
    \begin{equation*}
    V_\text{total} = V_{\text{Wasser}} + V_{\text{Zement}} + V_{\text{Luftporen}} + V_{\text{Gesteinskörnung}}
    \end{equation*}
    \pause
    \vspace{1cm}
    \begin{equation*}
        157 + 103 + 20 + 720 = 1000 \, \text{l/m³}
    \end{equation*}
    \pause
    Kontrolle: i.O.


\end{frame}

\begin{frame}{Fragen?}

\end{frame}

\begin{frame}{Auftrag}
\begin{itemize}
    \item Lösen Sie Aufgabe 2 selbständig.
    \item Zeit: ca. 10 Minuten
\end{itemize}

\end{frame}







\end{document}