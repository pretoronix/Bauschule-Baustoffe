% !TEX root = /Users/patricpf/Documents/repos/Bauschule-Baustoffe/Unterlagen/05_Beton/Mischungsbrechnung/Mischungsberechnung_Einfach.tex

% !TEX program = lualatex
\documentclass[
    %answers,
    a4paper,ngerman,12pt, addpoints]{exam}

\usepackage[utf8]{inputenc}
%\usepackage[T1]{fontenc}
%\usepackage[ngerman]{babel}


\usepackage{polyglossia}
\setdefaultlanguage[variant = swiss]{german}
\usepackage{fontspec}
\setmainfont{Aptos} % Bauschule CI Manual
\setsansfont{Aptos} % Bauschule CI Manual


\usepackage[ a4paper,
 total={165mm,250mm},
 left=25mm,
 top=25mm,
 headsep=10mm
 %footsep=12mm
 %,showframe
  ]{geometry}

\usepackage{graphicx}
\usepackage{siunitx}
\usepackage{booktabs} % schöne Tabellen
\usepackage{float}
\floatplacement{figure}{H}
\usepackage{xcolor}
\usepackage{pdfpages}
\usepackage{enumitem}
\usepackage{mdframed} % Boxen
\usepackage{amsmath,amssymb}
\usepackage{tcolorbox}
\usepackage{lastpage} % For the total number of pages
\usepackage{gensymb}
\usepackage{xspace}
\usepackage{tabularx}
\usepackage{multicol}
\usepackage[
    version=3,
    arrows=pgf-filled,
]{mhchem} % für chemische Formeln
%\usepackage{microtype}
\usepackage{subfigure}
\usepackage[hidelinks]{hyperref}
\usepackage{cleveref}
\usepackage{luacode}
\usepackage{amsmath}
\usepackage{textcomp}


\sisetup{
  locale = DE,
  inter-unit-product = \ensuremath{{\cdot}},
  detect-all,
}

% Colors
\definecolor{blau_bauschule}{RGB}{22,65,148}
\CorrectChoiceEmphasis{\color{blau_bauschule}}
\SolutionEmphasis{\color{blau_bauschule}}

\setlength{\parindent}{0em} % Verhindert einrücken
\setlength\linefillheight{0.3in}


%% COMMMANDS
\author{Patrick Pfändler}
\newcommand{\dozent}{Patrick Pfändler}
\newcommand{\fach}{Baustoffe}


\newcommand{\punkte}[1]{%
    \begin{infobox}%
        #1
    \end{infobox}}%
\newcommand{\FinRes}[1]{\underline{\underline{#1}}}

\newmdenv[linecolor=black,backgroundcolor=gray!15,frametitle={Punktverteilung},leftmargin=1cm,rightmargin=1cm]{infobox}

\newcommand{\pagebreaksol}{
    \ifprintanswers
        \clearpage
    \else
        {}
    \fi
}

\newcommand{\pagebreakexam}{
    \ifprintanswers
        {}
    \else
        \clearpage
    \fi
}

\SolutionEmphasis{\color{blau_bauschule}}
\makeatletter%
\newcommand{\solutiontable}[1]{\ifprintanswers\begingroup\Solution@Emphasis#1\if@shadedsolutions%
            {\cellcolor{SolutionColor}}%
        \else%
        \fi\endgroup\else\phantom{#1}\fi}%
\makeatother%

\newcommand{\myNmm}[1]
{
    \sisetup{per-mode=symbol}
    \SI{#1}{\newton\per\mm\squared}
}

\renewcommand{\thequestion}{\fontsize{12pt}{2pt} \selectfont  \bfseries \arabic{question}}
\sisetup{per-mode=symbol}



%% Translation

\pointpoints{Punkt}{Punkte}
\bonuspointpoints{Bonuspunkt}{Bonuspunkte}
\renewcommand{\solutiontitle}{\noindent\textbf{Lösung:}\enspace}
\chqword{Frage}
\chpgword{Seite}
\chpword{Punkte}
\chbpword{Bonus Punkte}
\chsword{Erreicht}
\chtword{Gesamt}
\hpword{Punkte:}
\hsword{Ergebnis:}
\hqword{Aufgabe:}
\htword{Summe:}


\renewcommand{\questionshook}{%
  %\setlength{\leftmargin}{0pt}% removes the indentation from the left
  \setlength{\labelwidth}{1.25cm}% adjusts label width
  \setlength{\itemindent}{0cm}% aligns the start of the item with the above
  \setlength{\labelsep}{0.25cm}% space between the label and the item text
}




%% header and footer
\pagestyle{headandfoot}
\firstpageheadrule
\runningheadrule

% Adjust the font size for the header
\firstpageheader{\fontsize{9}{11}\selectfont\fach}{}{\fontsize{9}{11}\selectfont\dozent \\ \blattname}
\runningheader{\fontsize{9}{11}\selectfont\fach}{}{\fontsize{9}{11}\selectfont\dozent \\ \blattname}

% Adjust the font size for the footer
\firstpagefooter{\includegraphics[width=2.5cm]{bauschule-logo-5cm.png}}{}{\fontsize{9}{11}\selectfont\thepage\,/\,\pageref{LastPage}}
\runningfooter{\includegraphics[width=2.5cm]{bauschule-logo-5cm.png}}{}{\fontsize{9}{11}\selectfont\thepage\,/\,\pageref{LastPage}}


\newcommand{\blattname}{Beton: Einfache Mischungsberechnung}



%% header and footer
\pagestyle{headandfoot}
\firstpageheadrule
\runningheadrule
\firstpageheader{\fach}{}{\fontsize{9pt}{2pt}\selectfont \dozent \\ \blattname}
\runningheader{\fach}{}{\fontsize{9pt}{2pt}\selectfont\dozent \\ \blattname}
\firstpagefooter{\includegraphics[width=2.5cm]{/Users/patricpf/Documents/repos/Bauschule-Baustoffe/template/bauschule-logo-5cm.png}}{}{\fontsize{9pt}{2pt}\selectfont \thepage\,/\,\numpages}
\runningfooter{\includegraphics[width=2.5cm]{/Users/patricpf/Documents/repos/Bauschule-Baustoffe/template/bauschule-logo-5cm.png}}{}{\fontsize{9pt}{2pt}\selectfont \thepage\,/\,\numpages}

%\printanswers



\begin{document}

{\fontsize{22pt}{2pt}\selectfont \textbf{\blattname}}
\vspace{0.3cm}
\section*{Theoretische Grundlagen}

\begin{equation*}
    1 m^3 = \frac{z}{\rho_z} + \frac{w}{\rho_w} + \frac{g}{\rho_g} + \frac{f}{\rho_f} + p
\end{equation*}
, wobei 

\begin{description}[leftmargin=!,labelwidth=\widthof{Abstand  },font=\normalfont]
    \item[$z, \rho_z$] Masse  bzw. Dichte des Zements 
    \item[$w, \rho_w$] Masse  bzw. Dichte des Wasssers 
    \item[$g, \rho_g$] Masse  bzw. Rohdichte der Gesteinskörnung
    \item[$f, \rho_f$] Masse  bzw. Dichte des Zussatzstoffes oder Zusatzmittels
    %\item[$f, \rho_f$] Masse  bzw. Dichte des Zussatzmittel
    \item[$p$] Porenvolumen (Lufgehalt)
\end{description}


\section*{Aufgaben}
\begin{questions}
    \question Berechnen Sie die fehlenden Teile in der untenstehenden Tabelle.
    \begin{table}[h]
        \centering
        \caption{Aufgabe zur Stoffraumberechnung}
        \begin{tabular}{lrrr}
        \toprule
        \textbf{Betonkomponenten}       & \textbf{Gehalt [kg/m³]} & \textbf{Dichte [kg/dm³]} & \textbf{Stoffraum [l/m³]} \\ 
        \midrule
        Zement                           & 320                      & 3.1                       &    {}          \\
        Gesamtwasser                     & 157                      & 1.0                       & {}            \\
        Luftporen (2,0\%)                & -                        & -                         & {}              \\
        Zementleimvolumen                &                          &                           &                           \\
        trockene Gesteinskörnung 0–32 mm & {}                     & 2,7                       & {}             \\
        \textbf{total}                   & {}           &            {}               & {}             \\
        \bottomrule
        \label{tab:Stoffraumberechnung}
        \end{tabular}
        \end{table}


        \question Berechnen Sie die fehlenden Teile in der untenstehenden Tabelle.

        Nehmen Sie dabei an, dass Sie mindenstens \SI{300}{\kg} Zement und einen Wasserzementwert von 0.5 haben möchten.

    \begin{table}[h]
        \centering
        \caption{Aufgabe zur Stoffraumberechnung}
        \begin{tabular}{lrrr}
        \toprule
        \textbf{Betonkomponenten}       & \textbf{Gehalt [kg/m³]} & \textbf{Dichte [kg/dm³]} & \textbf{Stoffraum [l/m³]} \\ 
        \midrule
        Zement                           & {}                      & 3.15                       &    {}          \\
        Gesamtwasser                     & {}                      & {}                       & {}            \\
        Luftporen (2,0\%)                & -                        & -                         & {}              \\
        Zementleimvolumen                &                          &                           &                           \\
        trockene Gesteinskörnung 0–32 mm & {}                     & 2.65                       & {}             \\
        \textbf{total}                   & {}           &            {}               & {}             \\
        \bottomrule
        \label{tab:Stoffraumberechnung}
        \end{tabular}
        \end{table}

\end{questions}





\end{document}