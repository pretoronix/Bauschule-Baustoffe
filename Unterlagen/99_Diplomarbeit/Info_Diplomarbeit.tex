% !TEX program = lualatex

\documentclass[
11pt,
captions=tableheading,
smallheadings,
%headings=big,
headsepline,
footsepline, 
%chapterprefix=false			% weiss nicht was passiert
captions=tableheading,
parskip=half-,
%BCOR=10mm,
%twocolumn, 
%draft
]{scrartcl}

%\usepackage[babelshorthands]{polyglossia}
\usepackage{polyglossia}
\setdefaultlanguage[variant = swiss]{german}

%\usepackage[ngerman]{babel}

%\usepackage[ngerman]{babel} 
\usepackage[]{ scrlayer-scrpage }
\usepackage[ a4paper,
 total={165mm,244mm},
 left=25mm,
 top=25mm,
 headsep=10mm
 %footsep=12mm
 %,showframe
  ]{geometry}
 \usepackage{fontspec}
\setmainfont{Times New Roman}
\setsansfont{Arial}

\usepackage[dvipsnames]{xcolor}
\usepackage[most]{tcolorbox}
\usepackage[
version=3,
arrows=pgf-filled,
]{mhchem} % für chemische Formeln
\usepackage{microtype}
\usepackage{float}
\usepackage{enumitem}
\usepackage{multicol}
\usepackage{booktabs}
\usepackage{pgfplots}
\usepackage{tabularx}
\usepackage{longtable}
\usepackage{fontawesome5}
\usepackage{pdfpages}

\usepackage{pdflscape} % Für Querformat-Seiten


\usepackage{siunitx}
\usepackage{amsfonts}
\usepackage{tabularx}

\frenchspacing

\floatplacement{figure}{H}

% Labeling of elements
\counterwithin{figure}{section}
\counterwithin{table}{section}
\counterwithin{equation}{section}

% Colors
\definecolor{blau_bauschule}{RGB}{22,65,148}

% Titel mit Bauschule blau gemäss CI manual
\addtokomafont{section}{\color{blau_bauschule}\Huge}
\addtokomafont{subsection}{\color{blau_bauschule}\huge}
\addtokomafont{subsubsection}{\color{blau_bauschule}\Large}
\addtokomafont{paragraph}{\normalsize}
\addtokomafont{subparagraph}{\small}
% Pagestyle
\pagestyle{scrheadings}
\ihead{\fontsize{9pt}{2pt}\selectfont }
\ohead{\fontsize{9pt}{2pt}\selectfont }
\chead{\fontsize{9pt}{2pt}\selectfont \headmark}
\ifoot{\fontsize{9pt}{2pt}\selectfont Bauschule Aarau} 
\ofoot{\fontsize{9pt}{2pt}\selectfont \thepage} %Seitennummer
\cfoot{\fontsize{9pt}{2pt}\selectfont }
\setkomafont{pagehead}{\normalfont}
\setkomafont{pagefoot}{\normalfont}
\setkomafont{pagefoot}{\normalfont}
\setkomafont{pagehead}{\normalfont}
\setkomafont{pagefoot}{\normalfont}
\setcounter{topnumber}{1}
\setcounter{bottomnumber}{1}
\automark[section]{subsection}


% Bild- und Tabellenunterschriften
\renewcommand*{\figurename}{Abbildung}
\renewcommand*{\tablename}{Tabelle}

\begin{document}
\section*{Hinweise zur Diplomarbeit}

\subsection*{Phase 1: Disposition}
Bitte bedenken, dass bis zum 
29.04.2025 
die Disposition der Diplomarbeit meine \textbf{Freigabe} erhalten muss. Ihr dürft mir gerne vorab das Dokument zusenden, bevorzugt in einem editierbaren Format.

Aus meiner Sicht darf in dieser Phase der Titel der Diplomarbeit noch geändert werden.

Ich empfehle vor der Abgabe der Disposition die Bewertungskriterien (nicht die Ausführungsbestimmungen) der Diplomarbeit nochmals zu lesen.

Ziele der Dispostion: 
\begin{itemize}
    \item Die Disposition dient als „roter Faden“ für die Hauptarbeit und zeigt auf, welche zentralen Fragestellun-
    gen mit der Arbeit bearbeitet und beantwortet werden. Die Disposition wir durch die Experten begleitet,
    wobei diese vor allem darauf achten, dass der Umfang der Diplomarbeit klar definiert ist und die Leitplanken gesetzt sind.
\end{itemize}


Nicht vergessen:
\begin{itemize}
    \item Die Disposition muss von mir unterzeichnet werden.$\Rightarrow$ Unterschriftsfeld vorbereiten.
\end{itemize}



\subsection*{Phase 2: Diplomarbeit}
\subsubsection*{Formales}
\begin{itemize}
\item Ich empfehle Blocksatz.
\item Jedes Bild, jede Tabelle und jede Formel muss eine eigene Nummer haben und im Text referenziert (z.\,B. Abbildung 1) werden.
\item Ich empfehle Titel zu nummerieren und Formatvorlagen konsequent zu verwenden.
\item 3 obligatorische Besprechungen. Ich bin jeweils am Montag an der Bauschule. Im Anschluss muss jeweils ein Protokoll erstellt werden (Bestandteild der Arbeit).
\item Quellenangaben müssen vorhanden sein.
\item Schriftgrösse 12pt 
\item 160 h Arbeit für die Arbeit. $\Rightarrow $ rund 10 h pro Woche
\item Abgabe: 12.08.2025, 11:00 Uhr, Bauschule Aarau Sektretariat
\item Keine Gespräche während der Lektionen erlaubt. Online-Meetings sind erlaubt, wenn ich nicht an der Bauschule bin.
\end{itemize}

\subsubsection*{Inhaltlich}
Ich empfehle vor der Abgabe  die Bewertungskriterien der Diplomarbeit zu nochmals zu lesen.

\end{document}