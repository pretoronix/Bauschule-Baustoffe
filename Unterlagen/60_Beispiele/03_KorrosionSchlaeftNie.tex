\section{Korrosion schläft nie}
\BlueSectionSlide


\begin{frame}{Einführung}

    \begin{itemize}
        \item Was ist Korrosion?
        \item Wie entsteht Korrosion?
        \item Welche Arten von Korrosion gibt es?
        \item Relevanz von Korrosion
    \end{itemize}
\end{frame}

\begin{frame}{Lernziele (i)}
    \begin{myLernziele}
        \begin{itemize}
            \item[\textbullet] Die Lernenden können den Begriff „Korrosion“ definieren und typische Beispiele aus der Praxis nennen.
            \item[\textbullet] Die Lernenden können die wichtigsten Korrosionsarten (z. B. Rost, Lochfrass, Spaltkorrosion) benennen.
            \item[\textbullet] Sie können die Bedeutung des Systemverhaltens bei der Korrosion (Material und Umgebung) erläutern.
            \item[\textbullet] Die Lernenden können Massnahmen zum Schutz vor Korrosion (z. B. Beschichtungen, Opferanoden, Materialwahl) auf praktische Beispiele anwenden.

        \end{itemize}
    \end{myLernziele}

\end{frame}
\begin{frame}{Lernziele (ii)}
    \begin{myLernziele}
        \begin{itemize}
            \item[\textbullet] Die Lernenden können die geeigneten Umgebungen für den Einsatz von Werkstoffen wie Aluminium oder Stahl benennen.
            \item[\textbullet] Die Lernenden können Korrosionsmechanismen anhand von Schadensbildern analysieren und die Ursachen identifizieren.
            \item[\textbullet] Sie können die Auswirkungen von Korrosion auf Infrastruktur und wirtschaftliche Kosten bewerten.
        \end{itemize}
    \end{myLernziele}
\end{frame}

\begin{frame}{Lernziele (iii)}
    \begin{myLernziele}
        \begin{itemize}
            \item[\textbullet] Sie können die Rolle der Umweltbedingungen bei der Auswahl von Werkstoffen kritisch beurteilen.
            \item[\textbullet] Die Lernenden können die Effektivität verschiedener Korrosionsschutzmassnahmen (z. B. galvanischer Schutz, Schutzbeschichtungen) bewerten.
        \end{itemize}
    \end{myLernziele}
\end{frame}




\begin{frame}{Definition Begriff: Korrosion}
    \begin{Definition_BS}{Korrosion}
        Korrosion ist aus technischer Sicht die Reaktion eines Werkstoffs mit seiner Umgebung, die
        eine messbare Veränderung des Werkstoffs bewirkt. Korrosion kann zu einer Beeinträchtigung
        der Funktion eines Bauteils oder Systems führen. Eine durch Lebewesen verursachte Korrosion
        wird als Biokorrosion bezeichnet. \footnote{Quelle: Wikipedia}
    \end{Definition_BS}
\end{frame}


\begin{frame}{Auftrag während des Videos}
    \begin{itemize}
        \item Notieren Sie für sich die wichtigsten Punkte.
        \item Auf Teams befindet sich wieder ein digitale Whiteboard (siehe Teams) mit den Fragen.
    \end{itemize}
\end{frame}


\begin{frame}{Whiteboard (i)}
    \begin{Fragenblock}
        \begin{itemize}
            \item[\textbullet]  Was ist Ihnen besonders aufgefallen?
            \item[\textbullet]  Was wussten Sie noch nicht?
            \item[\textbullet]  Was hat Sie überrascht?
            \item[\textbullet]  Wo sehen Sie Anknüpfungspunkte zu anderen Themen?
            \item[\textbullet]  Was intressiert Sie noch mehr?
        \end{itemize}
    \end{Fragenblock}

\end{frame}


\begin{frame}{Whiteboard (ii)}

    \begin{itemize}
        \item[\textbullet] Auch gerne Emojis verwenden für das Bewerten oder kommentieren.
        \item[\textbullet] \href{https://schweizerischebau-my.sharepoint.com/:wb:/g/personal/pfaendler_bauschule_ch/Ef9LEc4sPxROojf2Lq036s4BVeLmop9F0z6RdOGjloxprQ?e=k0avJg}{Link zum Whiteboard}
    \end{itemize}

\end{frame}


\begin{frame}{Video}
    \begin{itemize}
        \item \href{https://www.youtube.com/watch?v=h2Ez8dJto9s}{Korrosion schläft nie}
    \end{itemize}
\end{frame}


\begin{frame}{Diskussion der Ergebnisse im Plenum}

\end{frame}


\begin{frame}{Lernzielüberprüfung}
Gehen Sie nochmals mit ihrem Banknachbarn die Lernziele durch und überprüfen Sie, ob Sie alle Lernziele erreicht haben.

Zeit: ca. 10 Minuten
\end{frame}


