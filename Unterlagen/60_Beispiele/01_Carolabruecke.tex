\section{Carolabrücke}
\BlueSectionSlide

\begin{frame}{Carolabrücke Reminder}
    \begin{itemize}
        \item \href{https://www.youtube.com/watch?v=8PvhMLrxSYA}{Video zur Carolabrücke}
    \end{itemize}
\end{frame}


\begin{frame}{Vorläufige Erkenntnisse zur Ursache und Hergang des Teileinsturzes der Carolabrücke}
    \begin{block}{Zwischenergebnisse}
        Die vorliegenden Zwischenergebnisse deuten darauf hin, dass wasserstoffinduzierte Spannungsrisskorrosion die Hauptursache für das Versagen ist.
        \footnote{\href{        \footnote{https://www.dresden.de/media/pdf/presseamt/2024_12_11_Carolabruecke_Zusammenfassung-Ergebnisse.pdf}}{www.dresden.de}}
    \end{block}

\end{frame}

\begin{frame}{Wasserstoffinduzierte Spannungsrisskorrosion (SCC)}
    \begin{Definition_BS}{Spannungsrisskorrosion}
        Wasserstoffinduzierte Spannungsrisskorrosion (SCC) ist ein Schadensmechanismus, der unter spezifischen Voraussetzungen in
        hochbelasteten Metallen, wie vergütetem Spannstahl (u.\,a. auch Hennigsdorfer Spannstahl), auftreten kann. Dabei diffundiert Wasserstoff in die innere Gefügestruktur und führt dort unter anhaltender mechanischer Spannung zu Mikrorissen, die sich fortschreitend ausbreiten und
        schlieddlich zum spröden Versagen des Stahls führen können. Mangelnder Schutz vor Feuchtigkeit, korrosive Umgebung oder
        Verarbeitungsfehler begünstigen diesen Prozess.
        \footnote{\href{        \footnote{https://www.dresden.de/media/pdf/presseamt/2024_12_11_Carolabruecke_Zusammenfassung-Ergebnisse.pdf}}{www.dresden.de}}
    \end{Definition_BS}
\end{frame}


\begin{frame}{Wichtigste Konsequenzen}
    \begin{block}{}
        \begin{itemize}
            \item[\textbullet] Die \textbf{gesamte }Brücke wird für den Verkehr gesperrt und muss abgerissen werden.
            \item[\textbullet] Wasserstrasse darf nur nach der Installation eines Schallemissionssystems wieder freigegeben werden.
        \end{itemize}
    \end{block}
    \pause
    \begin{block}{Verschulden}
        Eine umfassende Aktenlage belegt, dass
        das Bauwerk innerhalb der geltenden Regelwerke bewertet und betrieben wurde.
    \end{block}
\end{frame}


\begin{frame}{Wichtigste Erkenntnisse}
    \begin{itemize}
        \item [\textbf{→}] \textbf{Haupteinsturzursache:} Wasserstoffinduzierte Spannungsrisskorrosion
        \item [\textbf{→}] \textbf{Konsequenz:} Einsturz nicht vorhersagbar, da keine ausgeprägte Rissbildung.
        \item [\textbf{→}] \textbf{Schuldfrage:} Gesetzliche Vorgaben eingehalten, keine Versäumnisse.
        \item [\textbf{→}] \textbf{Spannstahldefekte:} Über 68 Prozent der Spannglieder in der Fahrbahnplatte von Zug C waren an der Bruchstelle stark geschädigt.
        \item [\textbf{→}] \textbf{Massnahmen:} Abriss der \textbf{gesamten} Brücke, temporäre Installation eines Schallemissionssystems. Bau einer neuen Brücke.
        \item [\textbf{→}] \textbf{Tausalze:} Sogenannte chloridinduzierte Korrosion hat an Brückenzug C stattgefunden, war jedoch nicht ursächlich für den Einsturz.
    \end{itemize}
\end{frame}

\begin{frame}{Lernziele: Einsturz der Carolabrücke}
    \begin{myLernziele}
        \begin{itemize}
            \item[\textbullet] Kenntnisse über die Ursache des Teileinsturzes der Carolabrücke
            \item[\textbullet] Verständnis für die Konsequenzen des Einsturzes
            \item[\textbullet] Wissen über die wichtigsten Erkenntnisse aus dem Zwischenbericht
        \end{itemize}
    \end{myLernziele}
\end{frame}


\begin{frame}{Wie würden Sie die Brücke abbrechen?}
    \begin{itemize}
        \item Diskussion mit Nachbarn
        \item Ziel: Kurze Skizze oder ähnliches und Überlegungen darlegen; Darf auch nur Teile der Brücke betreffen.
        \item Zeit: ca. 15 Minuten
    \end{itemize}
\end{frame}

\begin{frame}{Video zu den Abbrucharbeiten}
    \begin{itemize}
        \item \href{https://www.youtube.com/watch?v=_FJjkKMLzMc}{Video zu den Abbrucharbeiten}
    \end{itemize}
\end{frame}


