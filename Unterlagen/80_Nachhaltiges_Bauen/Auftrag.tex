\section{Nachhaltiges Bauen und Kreislaufwirtschaft}
\BlueSectionSlide



\begin{frame}{Nachhaltiges Bauen}
    \begin{block}{Was kommt euch in den Sinn?}
        
    \end{block}
\end{frame}






\begin{frame}{Lernziele}
    \begin{myLernziele}
        \begin{itemize}
            \item Kennen der Vor- und Nachteile von Wiederverwendung von Bauteilen.
            \item Kennen der Herausforderungen und Chancen von Wiederverwendung von Bauteilen. 
            \item Kennen der Risiken von Wiederverwendung von Bauteilen aus Stahlbeton oder Stahl.
        \end{itemize}
        

    \end{myLernziele}

\end{frame}


\subsection{Wiederverwendung von Bauteilen}

% \begin{frame}{Auftrag}
%     \begin{itemize}
%         \item [\textbullet] 2-er Gruppen 
%         \item [\textbullet] Diskussion: 
%             \begin{itemize}
%                 \item Was macht ihr bei eurer Firma? 
%                 \item Was könntet ihr noch tun? 
%                 \item Was sind die Herausforderungen? 
%                 \item Was sind die Chancen?
%                 \item Was sind die Risiken?
%             \end{itemize}
%         \item [\textbullet] Notizen: Mindmap, Stichworte, Skizzen, etc.
%         \item [\textbullet] Zeit: 15 Minuten
%     \end{itemize}
%     Anschliessend Diskussion der Ergebnisse im Plenum.
% \end{frame}

\begin{frame}{Wiederverwendung von Bauteilen}
    Könnt ihr euch vorstellen, dass Bauteile wiederverwendet werden können?

    \begin{itemize}
        \item [\textbullet] Welche Bauteile?
        \item [\textbullet] Welcher Einsatz? 
        \item [\textbullet] Welche Herausforderungen?
        \item [\textbullet] Welche Chancen?
        \item [\textbullet] Welche Risiken?
        \item [\textbullet] Welche Vorteile?
        \item [\textbullet] Welche Nachteile?
        \item [\textbullet] Welche Kosten?
        \item [\textbullet] Welche Nutzen?
    \end{itemize}

    Diskussion in der Gruppe für bis zu 15 Minuten.

\end{frame}


\begin{frame}{Video: Reuse und Recycling im Sinn der \newline Kreislaufwirtschaft}

    \href{https://www.youtube.com/watch?v=-osIApQK0xI}{Video: Reuse und Recycling im Sinn der Kreislaufwirtschaft}


    ab Minute 46:00
\end{frame}

% \begin{frame}{Erhaltung Bestandsbau vs Erstellung Ersatzneubau}
%     \begin{itemize}
%         \item Das, wofür es gebaut wurde.
%     \end{itemize}

% \end{frame}

\begin{frame}{Wiederverwendung von Bauteilen}
\begin{itemize}
    \item [\textbullet] Spannungs-Dehnungsdiagramm (Materiakennwerte)
    \item [\textbullet] Karbonatisierung der Elemente ($\rightarrow$ trockene Verwendung)
    \item [\textbullet] Aufwändige, grosse Tests
\end{itemize}
\end{frame}

\begin{frame}{Wo könnt ihr euch in diesem Fall \newline eine Wiederverwendung vorstellen?}
    Diskussion für maximal 10 Minuten im Plenum:
    \begin{itemize}
        \item [\textbullet] Wie abbauen? 
        \item [\textbullet] Wie heben? 
        \item [\textbullet] Wie transportieren?
    \end{itemize}
\end{frame}

\begin{frame}{Tetris}
    Sinnvoll aus eurer Sicht? 

    \begin{itemize}
        \item [\textbullet] Vorteile und Nachteile?
    \end{itemize}

\end{frame}

\begin{frame}{Neue Nutzung der Fläche}
    Die neue Nutzung der Fläche ist durch ein Park. \footnote{\href{https://www.bzbasel.ch/basel/basel-stadt/volta-nord-gruene-oase-mit-83-baeumen-und-einem-zweistoeckigen-pavillon-das-sind-die-plaene-fuer-den-lysbuechelplatz-ld.2576182}{Quelle}}

\end{frame}

\begin{frame}{Wiederverwendung von Bauteilen: Stahl}
    \begin{itemize}
        \item Wie würdet ihr Stahlbauteile aus Stahlbeton wiederverwenden?
        \item Wie abbauen?
    \end{itemize}
    Diskussion im Plenum für maximal 10 Minuten.
\end{frame}


\begin{frame}{Video}
    \href{https://www.youtube.com/watch?v=D3MBcTZ1tI8}{Die nächsten Schritte des Stahl Re-Use Konzepts von uptownBasel
    }

    Zeit: ca 10 Minuten
\end{frame}

\begin{frame}{Fragen}
    Praktikabel für euch?


\end{frame}
    

% https://www.youtube.com/watch?v=eSrmvHYrPTM
