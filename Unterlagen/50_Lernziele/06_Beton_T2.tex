\subsubsection*{Beton}
Die Studierenden kennen: 

\begin{itemize}[noitemsep]
    \item kennen die Begriffe "Beton nach Zusammensetzung" und Beton nach Eigenschaften, sowie die Druckfestigkeitsklassen von Beton.
    \item können Expositionsklassen für übliche Anwendungen bestimmen und somit auch einen geeigneten Beton selektieren.
  \item den Unterschied zwischen dem Bemessungswert der Betondruckfestigkeit und den charakteristischen Werten.
  \item Expositionsklassen von Beton und können diese für einfache Fälle anwenden.
  \item die drei Betonfrischprüfungen (Ausbreitmass, Verdichtungsmass und Setzmass) und können diese anwenden.
  \item die Verwendungszwecke von Einheitsbetonen.
  \item die Unterschiede zwischen einem Normalbeton und Pumpbeton in der Mischrezeptur.
  \item die Verfahren für das Auftragen von Spritzbeton und dessen Eigenschaften.
  \item die Unterschiede zwischen einem Normalbeton und selbstverdichtetem Beton in der Mischrezeptur und dessen Eigenschaften.
  \item die unterschiedlichen Recyclinggesteinskörnungen sowie deren Vor- und Nachteile beim Frischbeton und bei erhärtetem Beton, sowie mögliche Einsatzbereiche im Tiefbau und Hochbau.
  \item die Eigenschaften und die Zusammensetzung von Leichtbeton.
  \item die Fortschritte, welche durch Polymer-Zusatzmittel im Bauwesen erreicht werden konnten.
  \item die Unterschiede zwischen einem Normalbeton und einem hochfesten Beton in der Mischrezeptur und dessen Eigenschaften.
  \item die Eigenschaften von Faserbeton und die Einsatzmöglichkeiten von Faserbeton mit unterschiedlichen Fasertypen.
\end{itemize}