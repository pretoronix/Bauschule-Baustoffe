\subsubsection*{Bindemittel}

Die Studierenden kennen: 

\begin{itemize}[noitemsep]
	\item Definition des Begriffes "Bindemittel", Zement, sowie Kohäsion und Adhäsion.
	\item die Einteilung der Bindemittel,die Eigenschaften der unterschiedlichen Bindemittel und können diese bedarfsgerecht selektieren. 
	\item das Funktionsschema, wie Bindemittel mit weiteren Grundstoffen vermengt werden können.
	\item erste Anwendungen von Bindemitteln.
	\item kennen die (Haupt-)Bestandteile von Mischzementen und deren Eigenschaften.
	\item die Einteilung der Zemente nach SN EN 197-1 und können aus dem Zementnamen auf dessen Eigenschaften schliessen.
	\item die Hydration des Zementes und dessen Auswirkungen, sowie die beiden wichtigsten Produkte bei der Zementhydratation (CSH und Calciumhydroxid).
	\item Probleme, welche bei der Zementlagerung möglich sind und können dementsprechend reagieren.
	\item die Herstellung von Zement und den  Kalkkreislauf.
	\item die Eigenschaften, Anwendungsbereiche und Herstellung folgender weiterer Bindemittel: Hydraulischer Kalk, Baugibs, Magnesit, Schamottmörtel, Polymerbeton, bituminöse Bindemittel.
	\item Untersuchungsmethoden für bituminöse Bindemittel.
	\item ökologische und gesundheitliche Aspekte von Bindemitteln und kennen die unterschiedlichen Entsorgungsmöglichkeiten von Bindemitteln. 
\end{itemize}