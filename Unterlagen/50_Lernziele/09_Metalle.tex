\subsubsection*{Metalle}
Die Studierenden kennen: 
\begin{itemize}
  \item die wichtigsten Eckpunkte in der Geschichte zur Herstellung von Eisen bzw. Stahl.
  \item Einteilung für Metalle (Eisen- und Nichteisenmetalle, Leicht-und Schwermetalle, etc.)
  \item kennen die wichtigsten Schritte bei der Herstellung von Eisen bzw. Stahl.
  \item die Namensgebung von Gusseisen und Stählen.
  \item die Werkstoffe (Gusseisen, Grauguss, Spähroguss, Temperguss und Stahlguss), inkl. dessen Rohmaterialien, Fabrikation, wichtigste Eigenschaften und 	mögliche Anwendungsbereiche.
  \item das Spannungs-Dehnungsdiagramm und wichtige Punkte in diesem Diagramm.
  \item die Anforderungen gemäss SIA 262 an Betonstahl.
  \item Vorteile von Stahl gegenüber Roheisen.
  \item die Herstellung von Aluminium (inkl. Legierungen), sowie Anwendungsbereiche und mechanische Eigenschaften.
  \item wichtige Legierungen, Eigenschaften und Anwendungsbereiche folgender Metalle: Kupfer, Zink, Zinn, Blei.
  \item die wichtigsten Stahlbauprofile, Bleche und Rohre aus Metall.
  \item Recycling- und Entsorgungsmöglichkeiten von Metallen.
  \item die wichtigsten Normen und Empfehlungen von Metallen im Bauwesen.
\end{itemize}