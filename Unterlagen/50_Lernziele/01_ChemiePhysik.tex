\subsection*{Grundlagen der Physik und der Chemie}

Die Studierenden kennen: 

\begin{itemize}[noitemsep]
	\item die Einteilung der Naturwissenschaften.
	\item Kriterien bei der Materialselektion von Baustoffen.
	\item Eigenschaftsklassen von Baustoffen
	\item die Unterschiede zwischen Chemie und Physik.
	\item  können Beispiele zu physikalischen und chemischen Prozessen aus dem Alltag und dem Bauwesen aufzählen.
	%\item Teilgebiete der Physik und der Chemie.
	\item die Einteilung der Materie.
	\item das Periodensystem und können relevante Daten aus dem Periodensystem der Elemente lesen.
	\item den Atomaufbau (inkl. der Begriffe Elektron, Proton und Neutron). 
	\item die Begriffe Isotope, Ionen, Masseanzahl und Ordnungszahl und können diese im Kontext anwenden.
	\item die drei wichtigsten Typen von chemischen Verbindungen, sowie deren grundlegenden Eigenschaften.
	%\item die Begriffe Molekül und Verbindung und können diese Unterscheiden und Beispiele nennen.
	\item die Begriffe chemische Reaktion, Analyse und Synthese und können Beispiele nennen.
	\item die Unterschiede zwischen einer endothermen und einer exothermen Reaktion und können Beispiele (u.a. aus dem Bauwesen) nennen.
	\item die Aggregatzustände, inklusive die Eigenschaften der einzelnen Aggregatzustände.
	\item die Dichteanomalie von Wasser. (siehe auch Aufgabe zur Dichteanomalie von Wasser).
	\item die Einheit Kelvin und können einfache Berechnungen durchführen.
	\item die pH-Wert-Skala. Des Weiteren sind die Studierenden in der Lage einfache pH-Wert-Be\-rechnungen durchzuführen und Beispiele aus der Baupraxis nennen.
	\item das SI-System (inkl. Formelzeichen, Zahlenwert und zugehörige Einheit).
	\item das Dezimalsystem. 
	\item können Messdaten in Grafiken aufzeichnen. 
	\item können einfache Berechnungen mit folgenden Themen durchführen: 
	\begin{itemize}[noitemsep]
		\item Länge
		\item Zeit 
		\item Masse 
		\item Kraft 
		\item Arbeit, Energie und Leistung 
		\item Dichte (inkl. Reindichte, Rohdichte und Schüttdichte) 
		\item Druck
	\end{itemize}
	\item die Ritzhärte nach Mohs, die Härte nach Brinell und die Härte nach Rockwell.
	%\item die Begriffe offenporig und geschlossenporig.
	%\item die Luftfeuchtigkeit und die relative Luftfeuchtigkeit.
	%\item die Begriffe Dampfdruckgefälle und Diffusionswiderstand.
\end{itemize}