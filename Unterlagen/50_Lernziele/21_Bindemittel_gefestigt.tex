\subsubsection*{Mit Bindemitteln gefestigte Baustoffe}
Die Studierenden kennen: 

\begin{itemize}[noitemsep]
	\item den Begriff: Mit Bindemitteln gefestigte Baustoffe.
	\item das allgemeine Fabrikationsschema von mit Bindemitteln gefestigte Baustoffen und können dieses auf verschiedene mit Bindemitteln gefestigte Baustoffen übertragen (z.\,B. Herstellung von Kalksandstein).
	\item die Rohmaterialien, die Eigenschaften, Verwendungszwecke, Fabrikationsprozesse, mögliche Nachbearbeitungsschritte, und spezielle Eigenschaften der folgenden mit Bindemitteln gefestigte Baustoffe
	\begin{itemize}[noitemsep]
  \item Kunststeine
  \item Betonelemente
  \item Betonwaren
  \item Zementstein
  \item Splittbetonsteine, -platten
  \item Leichtbetonsteine, -platten
  \item Porenbeton
  \item Zementgebundene Holzspanplatten
  \item Leichtbauplatten
  \item Glasfaserbeton / Polymerbeton
  \item  Leichtbeton Bauplatten
  \item Faserzement
  \item Kalksandsteine
  \item Gipsbauplatten,  Gipskartonplatten, 
  \item Gipsgebundene Holzfaserplatten, gipsgebundene Holzspanplatten
\end{itemize}
	\item die Entsorgungsmöglichkeiten für mit Bindemitteln gefestigte Baustoffe und ökologische und gesundheitliche Aspekte beim Umgang mit diesen Produkten.
\end{itemize}