\subsubsection*{Kunststoffe}
Die Studierenden kennen: 

\begin{itemize}[noitemsep]
	\item die Unterschiede und Eigenschaften, Aufbau, Verarbeitungsverfahren, Vor- und Nachteile von Kunststoffen (inkl. Naturkautschuk) zu anderen Baustoffen. 
	\item die Einteilung der Baustoffe (insbesondere Kunststoffe).
	\item die häufigsten Elemente, welche bei Kunststoffen vorkommen.
	\item den Aufbau der Kunststoffe.
	\item Möglichkeiten zur Beeinflussung der Eigenschaften von Kunststoffen (Hilfsstoffe, Zusatzstoffe, etc.)
	\item Kunststoffen nach ihren thermisch-mechanisch, nach ihrem Herstellungsprozess oder Verwendungsmöglichkeiten im Bauwesen unterteilen.
	\item unterschiedliche Kunststoffgruppen.
	\item Beispiele zu den unterschiedlichen Kunststoffen.
	\item mögliche Gefahren von Kunststoffen für die Umwelt.
	\item Möglichkeiten für das Recycling und Entsorgung von Kunststoffen.
	\item Gefahren von Halogenen und Produkte, welche beim Verbrennen Halogene ausstossen können.
	%\item wichtige Normen und Empfehlungen zu den Kunststoffen.
\end{itemize}