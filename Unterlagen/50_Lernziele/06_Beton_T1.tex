\subsubsection*{Beton}
Die Studierenden: 

\begin{itemize}[noitemsep]
	%\item kennen Frisch- und Festbetonprüfungen mit den Druckfestigkeitsklassifikation beschreiben.
	\item kennen die Ausgangsstoffe von Beton und können eine Mischungsberechnung (Stoffraumrechnung) durchführen
	%\item können Ziel- und Steuergrössen festlegen 
	\item können für übliche Praxisanwendungen die entsprechende Betonfestlegung bestimmen.
	\item kennen die Geschichte des Betons. 
	\item kennen mögliche Definition von Beton (z.B. nach SIA 262), sowie dessen Bestandteile.
	\item können die Unterschiede zwischen Beton und Mörteln resp. Bindemitteln erläutern. 
	\item kennen die Vorteile von Gesteinskörungen bei der Herstellung von Beton, sowie dessen Auswirkungen bezügliche den Frisch- und Festbetoneigenschaften.
	\item kennen die Eigenschaften der Gesteinskörnung, die Wichtigkeit der Siebkurve (maximale Packung und Grössenverteilung) (inkl. der Auswirkungen auf die Eigenschaften von Frisch- und Festbeton).
	\item können Gesteinskörungen nach unterschiedlichen Eigenschaften klassifizieren (z.B. Rohdichte). 
	\item kennen die Begriffe Mehlkorngehalt, Zugabewasser und deren Auswirkungen bei der Herstellung von Beton.
	\item kennen mindestens die folgenden Zusatzmittel und deren Auswirkungen auf die Frisch- und Festbetoneigenschaften, sowie kennen deren Funktionsweise: Betonverflüssiger, Fliessmittel, Luftporenbildner, Verzögerer, Luftporenbildner, Erstarrungs- und Erhärtungsbeschleuniger, Dichtungsmittel, Stabilisierer. 
	\item kennen Einflussfaktoren, welche die Wirksamkeit von Zusatzmittel beeinflussen können, sowohl positiv wie auch negativ.
	\item kennen die Auswirkungen der Wahl des w/z-Wertes und können eigenständig Berechnungen durchführen.
	\item kennen die Entwicklung der Festigkeit von Beton über die Zeit.
	%\item kennen die Begriffe "Beton nach Zusammensetzung" und Beton nach Eigenschaften, sowie die Druckfestigkeitsklassen von Beton.
	%\item können Expositionsklassen für übliche Anwendungen bestimmen und somit auch einen geeigneten Beton selektieren.
\end{itemize}