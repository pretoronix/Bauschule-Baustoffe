\subsubsection*{Abdichtungsmaterialien und Klebstoffe}

Die Studierenden kennen: 

\begin{itemize}[noitemsep]
	\item den Zweck von Abdichtungen und unterschiedliche Abdichtungskonzepte (z.B. für Flächen oder Fugen).
	\item die Begriffe Hydrophobierung, Imprägnierung und Beschichtung.
	\item erdberührte Schutzsysteme (wie z.B. Schutzanstrich, Schutzbeschichtungen und Abdichtungen), deren Anwendungszweck und Anwendungsbereiche und die Wirkungsweise.
	\item bewitterte Schutzsysteme (wie z.B. Imprägnierungsmittel, Beschichtungen, Gussasphalt), deren Anwendungszweck und Anwendungsbereiche und die Wirkungsweise.
	\item unterschiedliche Gewässerschutzsysteme, deren Anwendungszweck und Anwendungsbereiche und die Wirkungsweise.
	\item unterschiedliche Baupapiere und Folien, deren Anwendungszweck und mögliche Anwendungsbereiche sowie die Wirkungsweise.
	\item die Inhalte der Norm SIA 270 (Anwendungsgruppen, Dichtungsklassen, technische Abkürzungen, ...)
	\item einige Fabrikationsprozesse für die Herstellung von Kunststoffdichtungsbahnen.
	\item unterschiedliche Verbindungstechniken für Kunststoffdichtungsbahnen.
	\item Anwendungsbeispiele zu Kunststoffdichtungsbahnen (u.a. SIA 281).
	\item die Begriffe Bitumenbahn, Bitumen-Dichtungsbahn, Polymerbitumen-Dichtungsbahn, AC-Be\-ständig\-keit, Elastomerbitumen, Oxidationsbitumen, Plastomerbitumen, Träger\-einlage und Ob\-erflächenausrüstung.
	\item das Bezeichnungsschema von Bitumendichtungsbahnen und können dieses Anwenden.
	\item Flüssigkunststoff-Abdichtungen, sowie die Anwendungsmöglichkeiten, Eigenschaften und zugehörige Begriffe (Haftvermittler, Solldicke, Nutzschicht).
	\item das Bezeichnungsschema von Gussasphalt und können dieses anwenden.
	\item unterschiedliche Typen von Fugen und mögliche Abdichtungssystem zu den unterschiedlichen Fugentypen.
	\item unterschiedliche Arten von Klebstoffen und mögliche Anwendungsbereiche.
\end{itemize}
