%!TEX root = /Users/patricpf/Documents/repos/Bauschule-Baustoffe/HTf-26/KW_02_Programm.tex
%!TEX program = lualatex
\def\customoptions{aspectratio=169} % Exclude handout
% !TEX program = lualatex
\documentclass[aspectratio=169, 
handout,
]{beamer}


\PassOptionsToPackage{dvipsnames,svgnames}{xcolor}

%\usepackage[utf8]{inputenc}
%\usepackage[ngerman]{babel} % Schweizer Rechtschreibung

\usepackage{polyglossia}
\setdefaultlanguage[variant = swiss]{german}
\usepackage{fontspec}
\setmainfont{Times New Roman}
\setsansfont{Arial}


\usepackage{amsmath}
\usepackage{siunitx}
\sisetup{
  locale = DE,
  inter-unit-product = \ensuremath{{\cdot}},
  detect-mode,             % Use the surrounding text font mode
  detect-family,           % Use the surrounding text font family
  detect-weight,           % Use the surrounding text font weight
  mode = text,             % Ensure that numbers and units are typeset in text mode
  %negative-powers = false, % Avoid using negative powers
  per-mode = symbol        % Use the division symbol for per units
}

\usepackage{tikz}
\usepackage{enumitem}
\usepackage{graphicx}
\usepackage{booktabs}
\usepackage{calc}
\usepackage{multicol}
\usepackage{amsmath}

\usepackage{tcolorbox}
\tcbuselibrary{skins}

\usepackage[dvipsnames]{xcolor}

%\usepackage[scaled]{helvet} % Arial-ähnliche Schriftart (Helvetica)
%\renewcommand\familydefault{\sfdefault} % Setzt die Standard-Schriftart auf sans-serif


\usetheme{Madrid} % This theme is visually appealing
%\usecolortheme{whale} % A color theme with blue tones
\usecolortheme{dolphin} % A color theme with blue tones


\newcommand{\mylogo}{
    \begin{tikzpicture}[remember picture,overlay]
        \node[anchor=north east, yshift=-2mm, fill=white, inner sep=2pt] at (current page.north east) % Verschiebe das Logo um 5mm nach unten
            {\includegraphics[height=0.4cm]{/Users/patricpf/Documents/repos/Bauschule-Baustoffe/template/bauschule-logo-5cm.png}}; % Grösse nach Bedarf anpassen
    \end{tikzpicture}
}

\logo{\mylogo}

\setbeamertemplate{headline}{%
  \begin{beamercolorbox}[wd=\textwidth,ht=0.5ex,dp=1ex]{upper separation line head}
  \end{beamercolorbox}
}

\setbeamercolor{upper separation line head}{bg=blau_bauschule}


% Anpassen des Frametitels, um ihn fett zu machen
\setbeamertemplate{frametitle}{%
    \nointerlineskip%
    \begin{beamercolorbox}[sep=0.3cm,left,wd=\paperwidth]{frametitle}%
        \usebeamerfont{frametitle}\bfseries\insertframetitle%
    \end{beamercolorbox}%
}

%\usebackgroundtemplate{
%    \includegraphics[width=\paperwidth,height=\paperheight]{my_pdf_copy_of_empty_ppt_template}
%}

% Setzen Sie hier den Namen des Fachs
\newcommand{\fachname}{Baustoffe}
\newcommand{\FinRes}[1]{\underline{\underline{#1}}}

% Anpassen der Fusszeile für Abstände zum Rand
\setbeamertemplate{footline}
{
  \leavevmode%
  \hbox{%
  \begin{beamercolorbox}[wd=.33\paperwidth,ht=2.25ex,dp=1ex,left,leftskip=1em]{author in head/foot}%
    \usebeamerfont{author in head/foot}\insertshortauthor
  \end{beamercolorbox}%
  \begin{beamercolorbox}[wd=.34\paperwidth,ht=2.25ex,dp=1ex,center]{title in head/foot}%
    \usebeamerfont{title in head/foot}\fachname % Hier wird der Fachname anstelle des Titels angezeigt
  \end{beamercolorbox}%
  \begin{beamercolorbox}[wd=.33\paperwidth,ht=2.25ex,dp=1ex,right,rightskip=1em]{date in head/foot}%
    \usebeamerfont{date in head/foot}\insertframenumber{} / \inserttotalframenumber\hspace*{2ex}
  \end{beamercolorbox}}%
  \vskip0pt%
}


\newtcolorbox{Merke}{
enhanced,
boxrule=0pt,frame hidden,
borderline west={4pt}{0pt}{red!75!black},
colback=white,
sharp corners,
before upper={\textbf{Merke:}\quad},
}

\newtcolorbox{Anwendungen}{
enhanced,
boxrule=0pt,frame hidden,
borderline west={4pt}{0pt}{brown!75!black},
colback=white,
sharp corners
}




% Colors
\definecolor{blau_bauschule}{RGB}{22,65,148}
\setbeamercolor{frametitle}{fg=blau_bauschule}
\setbeamertemplate{navigation symbols}{} % Remove navigation symbols


\newtcolorbox{Definition_BS}[1]{
enhanced,boxrule=1pt,
colback=green!5!white,
colframe=green!75!black,fonttitle=\bfseries, title = #1,
%after title={\hfill\colorbox{black}{Definition}}
}



\newtcolorbox{Masseinheit}[1]{
enhanced,
boxrule=1pt,colframe=blue,
colback=white,
sharp corners, 
colframe=blue!75!black,
title = #1, 
after title={\hfill\colorbox{blue}{Masseinheit}}
}


\newtcolorbox{myLösung}{
  enhanced,
  boxrule=1pt,
  colframe=gray!75!black, % Definiert die Farbe des Rahmens als dunkelgrau
  colback=gray!20, % Definiert die Hintergrundfarbe der Box als hellgrau
  %sharp corners, % Macht die Ecken der Box scharf (nicht abgerundet)
  title = {Lösung}, % Fest eingestellter Titel der Box
  after title={}, % Fügt das Label "Masseinheit" nach dem Titel hinzu
  coltitle=white, % Farbe des Titeltexts
  fonttitle=\bfseries % Schriftart des Titels
}


% Set the title, author, and date
\title{\textbf{Lektionsprogramm HTf-26}}
\author{Patrick Pfändler}
\date{6. Januar 2025}


\begin{document}

\frame{\titlepage}

\begin{frame}{Inhalt der Lektion}
    \tableofcontents
\end{frame}


\section{Prüfungsvorbereitung}
\BlueSectionSlide
\begin{frame}{Vorbesprechung}
    \begin{itemize}
        \item[\textbullet]  Holz- und Holzwerkstoffe: Zusammenfassung
        \item[\textbullet]  Holz- und Holzwerkstoffe: Quiz
        \item[\textbullet]  Natursteine: Zusammenfassung
    \end{itemize}
\end{frame}


\section{Korrosionsbeständige Bewehrung}

\BlueSectionSlide
\subsection{Nicht-rostender Bewehrungsstahl}
\begin{frame}{Zusammenfassung}

    \begin{itemize}
        \item[\textbullet]  Nicht-rostender Betonstahl
        \item[\textbullet]  Faserbewehrung
              \begin{itemize}
                  \item[\textbullet]  Glasfaser-Bewehrung
                  \item[\textbullet]  Carbonfaser-Bewehrung
                  \item[\textbullet]  Basaltfaser-Bewehrung
              \end{itemize}
    \end{itemize}

\end{frame}

\begin{frame}{Wahl der Korrosionswiderstandsklasse}
    Fälle, in denen die Verwendung eines Stahls mit erhöhter Korrosionsbeständigkeit in Betracht zu ziehen ist:
    \begin{itemize}
        \item[\textbullet] Das Einhalten einer normkonformen Bewehrungsüberdeckung ist nicht möglich
        \item[\textbullet] Ein hoher Chlorideintrag ist zu erwarten (z.B. bei den Arbeitsfugen von an Streusalz ausgesetzten Bauteilen)
        \item[\textbullet] Eine Instandsetzung ist mit hohem Aufwand und/oder Störung des Verkehrsflusses verbunden
        \item[\textbullet] Wegen schwierigen Bedingungen bei der Ausführung kann eine genügende Ausführungsqualität nicht sicher erreicht werden
        \item[\textbullet] Für die Tragsicherheit des Überbaus erforderliche Konsolköpfe und Leitmauern
        \item[\textbullet] Nicht kontrollierbare / nicht inspizierbare Bauteile
    \end{itemize}
\end{frame}

\begin{frame}{Mechanische Eigenschaften}
    \begin{block}{Unterschiede}
        Nichtrostenden Betonstahl hat im Vergleich zu herkömmlichen Betonstählen:
        \begin{itemize}
            \item[\textbullet] höhere Zugfestigkeit (hängt von der Legierung ab)
            \item[\textbullet] Plastische Verformung ist erhöht (hängt von der Legierung ab)
            \item[\textbullet] Wärmeausdehnungskoeffizient ist grösser (hängt von der Legierung ab)
            \item[\textbullet] Schweissneigung ist geringer. (Reduktion der Korrosionsbeständigkeit möglich) $\rightarrow$ Falls Schweissungen erforderlich, zusätzliche Anforderungen beachten.
                  %\item[\textbullet] Erhöhte Lebensdauer möglich.
        \end{itemize}

    \end{block}

\end{frame}

\begin{frame}{Ausführung (SIA 2029)}
    \textbf{Verarbeitung von nichtrostendem Betonstahl}
    \begin{itemize}
        \item[\textbullet]  \textbf{Werkzeuge} für die Bearbeitung von nichtrostendem Betonstahl müssen aus nichtrostendem Material sein.
              Um \textbf{Verunreinigungen} der nichtrostenden Betonstähle zu vermeiden, muss die Verarbeitung räumlich von der Verarbeitung von unlegiertem Betonstahl getrennt werden. Das \textbf{Abbiegen} auf der Baustelle ist verboten.
        \item[\textbullet]  \textbf{Bindedrähte} und die Drähte der Abstandshalter sollen ebenfalls aus nichtrostenden Materialien sein.
        \item[\textbullet]  Die \textbf{Distanz- und Montagestäbe} für Bewehrungen aus nichtrostenden Betonstählen sind ebenfalls mit nichtrostenden Betonstählen auszuführen.
        \item[\textbullet]  Der \textbf{Transport} und die \textbf{Lagerung} von unlegierten und nichtrostenden Betonstählen müssen räumlich getrennt erfolgen.
        \item[\textbullet]  \textbf{Mischbewehrungen} von unlegierten und nichtrostenden Betonstählen sind zulässig.
    \end{itemize}
\end{frame}

\begin{frame}{Nichtrostender Bewehrungsstahl}
    \begin{block}{Zusammenfassung}
        \begin{itemize}
            \item[\textbullet] Verhalten ähnlich wie unlegierter Betonstahl (u.a. mechanische, Verbund mit Beton, thermische)
            \item[\textbullet]Normativ abgebildet (SIA Merkblatt 2029)
            \item[\textbullet]Erhöhte dauerhaftigkeit /Erhöhte Kosten
            \item[\textbullet] $\rightarrow$ Intelligente Nutzung erforderlich.
        \end{itemize}
    \end{block}
\end{frame}

\subsection{Faserbewehrung}
\begin{frame}{Faserbewehrung}
    \begin{block}{Übersicht}
        \begin{itemize}
            \item[\textbullet]  Glasfaser-Bewehrung
            \item[\textbullet]  Carbonfaser-Bewehrung
            \item[\textbullet]  Basaltfaser-Bewehrung
        \end{itemize}
    \end{block}
\end{frame}

\begin{frame}{Bauverfahren}
    \begin{block}{Betonstahlbewehrung vs. Faserbewehrung}
        \begin{itemize}
            \item [\textbullet] Gibt es gerade (ungebogen), da nach dem Aushärten der Polymermatrix (Duroplaste) eine Formgebung nicht mehr möglich ist. \\ (Vergleich: Bewehrungsstahl ist biegbar, erlaubt komplexe Formen und gute Verankerung im Beton, Bemerkung: Biegen kann in der Fabrik erfolgen)
            \item [\textbullet]Gerade Stäbe schränken die Formgebung des Bauteils stark ein
            \item [\textbullet]Gerade Stäbe sind nicht wirtschaftlich, da schlechte Verankerung im Beton $\rightarrow$ somit kann die hohe Festigkeit von CFK (Kohlenstofffaserverstärkter Kunststoff) bei Stäben nur sehr eingeschränkt ausgenutzt werden
        \end{itemize}
    \end{block}

\end{frame}

\begin{frame}{Bauverfahren}
    \begin{block}{Textilbewehrung}
        \begin{itemize}
            \item [\textbullet]  Sinnvoller als Stäbe (besserer Verbund mit Beton), aber
                  \begin{itemize}
                      \item [\textbullet]  aufwändiger Bauprozess
                      \item [\textbullet]  Bei vertikalen und komplexen Bauteilen ist die Fixierung der Textilbewehrung beim Betonieren eine Herausforderung
                      \item[\textbullet] Feinmaschig: $\rightarrow$ Verwendung von Mörtel als Beton ($\rightarrow$ Zielkonflikt mit der Klinkerreduktion im Beton)
                  \end{itemize}
        \end{itemize}
    \end{block}
\end{frame}

\begin{frame}{Faserbewehrung: Langzeiterfahrung}
    \vspace{-0.5em} % Verringert den vertikalen Abstand unter dem Titel
    \begin{itemize}
        \item \textbf{Carbon, Glas, Basalt:} beständig.
        \item \textbf{Kunststoffmatrix (oft Epoxidharz):} viele Fragen noch ungeklärt.
        \item \textbf{Laborstudien:} weisen auf einen möglichen \textbf{Verlust an Festigkeit \& Steifigkeit} mit zunehmender Expositionsdauer (Feuchteaufnahme) hin.
        \item \textbf{Langzeiterfahrung:} nur sehr beschränkt unter reellen Expositionsbedingungen (im Vergleich zu 100+ Jahren bei Stahlbeton).
    \end{itemize}
    \vspace{1em}
    \centering \textcolor{red}{\faExclamationCircle{} Zurückhaltung angebracht bei neuen Systemen}
\end{frame}

\begin{frame}{Mögliche Einsatzbereiche für Faserbewehrung}
    \begin{itemize}
        \item[\textbullet] Spezialanwendungen: z.B. vorfabrizierte Elemente
    \end{itemize}

\end{frame}
\folieFragen
\section{Organisatorisches}
\BlueSectionSlide

\subsection{Uploads auf Teams}
\begin{frame}{Uploads auf Teams}
    \begin{itemize}
        \item[\textbullet] Terminprogramm für nach den Weihnachtsferien
        \item[\textbullet] Lernziele für die Prüfung
        \item[\textbullet] Vorbesprechung der Prüfung (Folien)
    \end{itemize}

\end{frame}

% \begin{frame}{Fragen zur Prüfung?}

% \end{frame}

\naechstePruefung{13.01.2024 }{Holz-und Holzwerkstoffe, Natursteine}



\end{document}