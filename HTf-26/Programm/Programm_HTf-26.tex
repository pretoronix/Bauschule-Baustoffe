
\documentclass[
11pt,
captions=tableheading,
%smallheadings,
headings=big,
headsepline,
footsepline, 
%chapterprefix=false			% weiss nicht was passiert
captions=tableheading,
parskip=half-,
%BCOR=10mm,
%twocolumn, 
%draft
]{scrartcl}

%\usepackage[babelshorthands]{polyglossia}
\usepackage{polyglossia}

\setdefaultlanguage[variant = swiss]{german}
%\usepackage[ngerman]{babel} 
\usepackage[]{ scrlayer-scrpage }
\usepackage[ a4paper,
 total={165mm,244mm},
 left=25mm,
 top=25mm,
 headsep=10mm
 %footsep=12mm
 %,showframe
  ]{geometry}
 \usepackage{fontspec}
\setmainfont{Arial} % Bauschule CI Manual
\setsansfont{Arial} % Bauschule CI Manual


\usepackage[dvipsnames]{xcolor}
\usepackage[most]{tcolorbox}
\usepackage[
version=3,
arrows=pgf-filled,
]{mhchem} % für chemische Formeln
%\usepackage{microtype}
\usepackage{float}
\usepackage{enumitem}
\usepackage{multicol}
\usepackage{booktabs}
\usepackage{pgfplots}
\pgfplotsset{compat=newest}

\usepackage{float}
\usepackage{booktabs}
\usepackage{xspace}
\usepackage{luacode}
%\pgfplotset{compact= 1.17}
%\usepackage{structuralanalysis}

\usepackage{siunitx}
\usepackage{smartdiagram}
\usepackage{amsfonts}
\usepackage{amssymb,amsmath}

 \usepackage{titletoc}

% order of hyperref, cleverref is important
\usepackage[hidelinks]{hyperref}
\usepackage{cleveref}


\usepackage{graphicx}

% Colors
\definecolor{blau_bauschule}{RGB}{22,65,148}


% Titel mit Bauschule blau gemäss CI manual
\addtokomafont{section}{\color{blau_bauschule}\Huge}
\addtokomafont{subsection}{\color{blau_bauschule}\huge}
\addtokomafont{subsubsection}{\color{blau_bauschule}\Large}
\addtokomafont{paragraph}{\normalsize}
\addtokomafont{subparagraph}{\small}
% Pagestyle
\pagestyle{scrheadings}
\ihead{\fontsize{9pt}{2pt}\selectfont \klasse}
\ohead{\fontsize{9pt}{2pt}\selectfont \fach}
\chead{\fontsize{9pt}{2pt}\selectfont \headmark}
\ifoot{\fontsize{9pt}{2pt}\selectfont Bauschule Aarau} 
\ofoot{\fontsize{9pt}{2pt}\selectfont \thepage} %Seitennummer
\cfoot{\fontsize{9pt}{2pt}\selectfont }
\setkomafont{pagehead}{\normalfont}
\setkomafont{pagefoot}{\normalfont}
\setkomafont{pagefoot}{\normalfont}
\setkomafont{pagehead}{\normalfont}
\setkomafont{pagefoot}{\normalfont}


% Bild- und Tabellenunterschriften
\renewcommand*{\figurename}{Abbildung}
\renewcommand*{\tablename}{Tabelle}

\newcommand{\fach}{Baustoffe\xspace}
\newcommand{\klasse}{HTf-26\xspace}


% Define a shortcut for red text
\newcommand{\red}[1]{\textcolor{red}{#1}}


% Titel
\title{\fach}
%\author{Patrick Pfändler}
\date{2024}



\begin{document}

\section*{Programm Baustoffe}

Klasse: \klasse

Zeit: 08:00 bis 10:00 Uhr (120 Minuten) im Raum 201

Anzahl Schüler: 13



Stand: \today

\vspace{0.75cm}



\begin{table}[H]
    \centering
    \begin{tabular}{llp{8.5cm}p{3.5cm}}
        \toprule
        \textbf{Datum} & \textbf{KW} & \textbf{Inhalt}                                                                & \textbf{Bemerkung}               \\
        \midrule
        22.10.2024     & 43          & Dauerhaftigkeit, Einstieg nach Ferien                                          & {}                               \\
        28.10.2024     & 44          & Quiz Metall, Holz- und Holzwerkstoffe                                          & {}                               \\
        04.11.2024     & 45          & Holz- und Holzwerkstoffe                                                       & {}                               \\
        11.11.2024     & 46          & \red{Prüfung: Wärmedämmstoffe und Metalle} und Holz- und Holzwerkstoffe        & {}                               \\
        18.11.2024     & 47          & Nachbesprechung Prüfung; Abschluss Holz- und Holzwerkstoffe; Start Natursteine & {}                               \\
        25.11.2024     & 48          & Natursteine  und Innovation im Bauwesen                                        & {}                               \\
        02.12.2024     & 49          & Innovation im Bauwesen und Nachbesprechung Natursteine                         & {}                               \\
        09.12.2024     & 50          & Betoninstandsetzung                                                            & {}                               \\
        16.12.2024     & 51          & Korrosionsbeständige Bewehrung; evtl. weiter mit Betoninstandsetzung           & Besuch Natalie Räber             \\
        \midrule
        23.12.2024     & 52          & \textcolor{blue}{Ferien}                                                       & {}                               \\
        \midrule
        %\Xhline{2\arrayrulewidth} % A thick line that is twice the default thickness
        \\ \addlinespace
        \midrule
        06.01.2025     & 2           & Betoninstandsetzung; Prüfungsvorbereitung                                      & {}                               \\
        13.01.2025     & 3           & \red{\textbf{Prüfung:} Holz- und Holzwerkstoffe, Natursteine}                  & {}                               \\
        20.01.2025     & 4           & Prüfungsnachbesprechung                                                        & {}                               \\
        \midrule
        27.01.2025     & 5           & \textcolor{blue}{Ferien}                                                       & {}                               \\
        \midrule
        03.02.2025     & 6           & Innovation im Bauwesen                                                         & {}                               \\
        10.02.2025     & 7           & Flüsterbeläge  und ökologische Beläage                                                               & {}                               \\
        17.02.2025     & 8           & Fachabschlussprüfung; Quiz über abgeschlossene Themen                    & {}                               \\
        24.02.2025     & 9           & Zusammenfassung und Feedback                                                   & letzte Lektion im Fach Baustoffe \\
        \bottomrule
    \end{tabular}
\end{table}



\end{document}