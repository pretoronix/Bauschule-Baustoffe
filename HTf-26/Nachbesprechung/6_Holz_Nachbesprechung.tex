% !TEX root = /Users/patricpf/Documents/repos/Bauschule-Baustoffe/HTf-26/Nachbesprechung/6_Holz_Nachbesprechung.tex
\def\customoptions{aspectratio=169} % Exclude handout
% !TEX program = lualatex
\documentclass[aspectratio=169, 
handout,
]{beamer}


\PassOptionsToPackage{dvipsnames,svgnames}{xcolor}

%\usepackage[utf8]{inputenc}
%\usepackage[ngerman]{babel} % Schweizer Rechtschreibung

\usepackage{polyglossia}
\setdefaultlanguage[variant = swiss]{german}
\usepackage{fontspec}
\setmainfont{Times New Roman}
\setsansfont{Arial}


\usepackage{amsmath}
\usepackage{siunitx}
\sisetup{
  locale = DE,
  inter-unit-product = \ensuremath{{\cdot}},
  detect-mode,             % Use the surrounding text font mode
  detect-family,           % Use the surrounding text font family
  detect-weight,           % Use the surrounding text font weight
  mode = text,             % Ensure that numbers and units are typeset in text mode
  %negative-powers = false, % Avoid using negative powers
  per-mode = symbol        % Use the division symbol for per units
}

\usepackage{tikz}
\usepackage{enumitem}
\usepackage{graphicx}
\usepackage{booktabs}
\usepackage{calc}
\usepackage{multicol}
\usepackage{amsmath}

\usepackage{tcolorbox}
\tcbuselibrary{skins}

\usepackage[dvipsnames]{xcolor}

%\usepackage[scaled]{helvet} % Arial-ähnliche Schriftart (Helvetica)
%\renewcommand\familydefault{\sfdefault} % Setzt die Standard-Schriftart auf sans-serif


\usetheme{Madrid} % This theme is visually appealing
%\usecolortheme{whale} % A color theme with blue tones
\usecolortheme{dolphin} % A color theme with blue tones


\newcommand{\mylogo}{
    \begin{tikzpicture}[remember picture,overlay]
        \node[anchor=north east, yshift=-2mm, fill=white, inner sep=2pt] at (current page.north east) % Verschiebe das Logo um 5mm nach unten
            {\includegraphics[height=0.4cm]{/Users/patricpf/Documents/repos/Bauschule-Baustoffe/template/bauschule-logo-5cm.png}}; % Grösse nach Bedarf anpassen
    \end{tikzpicture}
}

\logo{\mylogo}

\setbeamertemplate{headline}{%
  \begin{beamercolorbox}[wd=\textwidth,ht=0.5ex,dp=1ex]{upper separation line head}
  \end{beamercolorbox}
}

\setbeamercolor{upper separation line head}{bg=blau_bauschule}


% Anpassen des Frametitels, um ihn fett zu machen
\setbeamertemplate{frametitle}{%
    \nointerlineskip%
    \begin{beamercolorbox}[sep=0.3cm,left,wd=\paperwidth]{frametitle}%
        \usebeamerfont{frametitle}\bfseries\insertframetitle%
    \end{beamercolorbox}%
}

%\usebackgroundtemplate{
%    \includegraphics[width=\paperwidth,height=\paperheight]{my_pdf_copy_of_empty_ppt_template}
%}

% Setzen Sie hier den Namen des Fachs
\newcommand{\fachname}{Baustoffe}
\newcommand{\FinRes}[1]{\underline{\underline{#1}}}

% Anpassen der Fusszeile für Abstände zum Rand
\setbeamertemplate{footline}
{
  \leavevmode%
  \hbox{%
  \begin{beamercolorbox}[wd=.33\paperwidth,ht=2.25ex,dp=1ex,left,leftskip=1em]{author in head/foot}%
    \usebeamerfont{author in head/foot}\insertshortauthor
  \end{beamercolorbox}%
  \begin{beamercolorbox}[wd=.34\paperwidth,ht=2.25ex,dp=1ex,center]{title in head/foot}%
    \usebeamerfont{title in head/foot}\fachname % Hier wird der Fachname anstelle des Titels angezeigt
  \end{beamercolorbox}%
  \begin{beamercolorbox}[wd=.33\paperwidth,ht=2.25ex,dp=1ex,right,rightskip=1em]{date in head/foot}%
    \usebeamerfont{date in head/foot}\insertframenumber{} / \inserttotalframenumber\hspace*{2ex}
  \end{beamercolorbox}}%
  \vskip0pt%
}


\newtcolorbox{Merke}{
enhanced,
boxrule=0pt,frame hidden,
borderline west={4pt}{0pt}{red!75!black},
colback=white,
sharp corners,
before upper={\textbf{Merke:}\quad},
}

\newtcolorbox{Anwendungen}{
enhanced,
boxrule=0pt,frame hidden,
borderline west={4pt}{0pt}{brown!75!black},
colback=white,
sharp corners
}




% Colors
\definecolor{blau_bauschule}{RGB}{22,65,148}
\setbeamercolor{frametitle}{fg=blau_bauschule}
\setbeamertemplate{navigation symbols}{} % Remove navigation symbols


\newtcolorbox{Definition_BS}[1]{
enhanced,boxrule=1pt,
colback=green!5!white,
colframe=green!75!black,fonttitle=\bfseries, title = #1,
%after title={\hfill\colorbox{black}{Definition}}
}



\newtcolorbox{Masseinheit}[1]{
enhanced,
boxrule=1pt,colframe=blue,
colback=white,
sharp corners, 
colframe=blue!75!black,
title = #1, 
after title={\hfill\colorbox{blue}{Masseinheit}}
}


\newtcolorbox{myLösung}{
  enhanced,
  boxrule=1pt,
  colframe=gray!75!black, % Definiert die Farbe des Rahmens als dunkelgrau
  colback=gray!20, % Definiert die Hintergrundfarbe der Box als hellgrau
  %sharp corners, % Macht die Ecken der Box scharf (nicht abgerundet)
  title = {Lösung}, % Fest eingestellter Titel der Box
  after title={}, % Fügt das Label "Masseinheit" nach dem Titel hinzu
  coltitle=white, % Farbe des Titeltexts
  fonttitle=\bfseries % Schriftart des Titels
}


\usepackage{fontawesome}  % For check and cross symbols


% Set the title, author, and date
\title{\textbf{Nachbesprechung Prüfung: HTf-26: Holz und Holzwerkstoffe, Natursteine 
}}
\author{Patrick Pfändler}
\date{20. Januar 2025}


\begin{document}
%Start of the slides
\frame{\titlepage}

\begin{frame}{Inhalsverzeichnis}
    \tableofcontents
\end{frame}

\section{Notenübersicht}
\BlueSectionSlide

\notenverteilung{
    4.3 %min
}{
    5.1%median
}{
    5.1%mean
}{
    5.6 %max
}{
    0.36%std
}
\notenformel{
    8 %note 1
}
{
    32%note 6
}
{
    35.36%max points
}


\section{Nachbesprechung der Prüfung}
\BlueSectionSlide

\begin{frame}{Trocknung von Holz: Berechnung}
Wie lange dauert die Trocknung von Holz auf ca. 15 bis 20\% Wassergehalt eines 5 cm dicken Brettes aus Laubholz.
\pause
\begin{myLösung}
Die Trocknungsdauer des Brettes beträgt ungefähr 5 Jahre.
\end{myLösung}

Bitte jeweils ausrechnen um die vollte Punktzahl zu erhalten.

\end{frame}

\begin{frame}{Wahr / Falsch}

\begin{Fragenblock}
    Mittels natürlicher Trocknung können sämtliche Holzfeuchten erreicht werden?
\end{Fragenblock}
\pause

\begin{myLösung}
Wahr.
\end{myLösung}

\end{frame}
\begin{frame}{Wahr / Falsch}

\begin{Fragenblock}
    Faserplatten können ohne Klebstoff hergestellt werden.
\end{Fragenblock}
\pause

\begin{myLösung}
Wahr.
\end{myLösung}
\end{frame}

\begin{frame}{Holzfaserplatte vs Spanplatte}
    $\rightarrow$ Frage kam in der Prüfung vor und hätte noch etwas besser beantwortet werden können.
\end{frame}

\begin{frame}{Gesteine erkennen}
    Frage  ist relativ schwierig, daher auch Reduktion der benötigten Punktzahl für die Maximalnote.

\end{frame}

\begin{frame}{Gesteine}
    \begin{table}[h!]
        \centering
        \begin{tabular}{ll}
            \toprule
            \textbf{Gesteinsart}                & {}                 \\
            \midrule
            Magmatit                            & Erstarrungsgestein \\
            Sedimentgestein                     & Ablagerungsgestein \\
            \textcolor{red}{Umwandlungsgestein} & Metamorphit        \\
            \bottomrule
        \end{tabular}
        \caption{Übersicht der Gesteinsarten}
        \label{tab:gesteinsarten}
    \end{table}

\end{frame}


\begin{frame}{Defintion erkennen}

\begin{Fragenblock}
    Verwitterung und Abtragung älterer Gesteine an der Erdoberfläche beschreibt die Herkunft des Material.
    Nenne die Einteilung der Gesteine:
\end{Fragenblock}
\pause

\begin{myLösung}
Sedimentgestein; Ziel wäre es gewesen hier nicht die gesamte Unterteilung zu schreiben. Dies Frage wurde aber häufig richtig beantwortet.
\end{myLösung}
\end{frame}

\begin{frame}{Markiere das Brett ohne Mark}
\begin{figure}
    \centering
    \includegraphics[width=0.5\textwidth]{/Users/patricpf/Documents/repos/Bauschule-Baustoffe/HTf-26/Nachbesprechung/Bilder/MarkfreiesBrett.png}
    \caption{Markiere das Brett ohne Mark.}
    \label{fig:markierebrett}
\end{figure}

\begin{myLösung}
Ziel wäre es gewesen die Markierung auf das Brett zu setzen. Nicht nur auf die Seite.
\end{myLösung}

\end{frame}

\begin{frame}{Künstliche Trocknung}
\begin{Fragenblock}
    In der geschlossenen Kammer wird mittels Wärmepumpe die Luft abgekühlt, das aufgenommene Wasser kondensiert. Die erwärmte Luft durchströmt dann wieder den Holzstapel und nimmt dort wieder Feuchtigkeit auf.
\end{Fragenblock}
\pause
\begin{myLösung}
Kondensationstrocknung. (Trockenkammer alleine hat nicht ausgreicht.)
\end{myLösung}

\end{frame}

\begin{frame}{Holzschutz}
\begin{columns}
% Linke Spalte mit der Abbildung
\column{0.5\textwidth}
\begin{figure}
    \includegraphics[height=0.8\textheight]{/Users/patricpf/Documents/repos/Bauschule-Baustoffe/HTf-26/Nachbesprechung/Bilder/Stützenfuss.png}
\end{figure}

% Rechte Spalte für weiteren Inhalt
\column{0.5\textwidth}
\begin{myLösung}
Konstruktiver Holzschutz
\end{myLösung}
\end{columns}
\end{frame}

\begin{frame}{Fazit}
    Ich bin zufrieden mit eurer Leistung.

\end{frame}

\end{document}