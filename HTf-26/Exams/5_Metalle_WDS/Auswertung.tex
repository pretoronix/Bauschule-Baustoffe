% !TEX root = /Users/patricpf/Documents/repos/Bauschule-Baustoffe/HTf-26/Exams/5_Metalle_WDS/Auswertung.tex
% !TEX program = lualatex
\documentclass[aspectratio=169, 
handout,
]{beamer}


\PassOptionsToPackage{dvipsnames,svgnames}{xcolor}

%\usepackage[utf8]{inputenc}
%\usepackage[ngerman]{babel} % Schweizer Rechtschreibung

\usepackage{polyglossia}
\setdefaultlanguage[variant = swiss]{german}
\usepackage{fontspec}
\setmainfont{Times New Roman}
\setsansfont{Arial}


\usepackage{amsmath}
\usepackage{siunitx}
\sisetup{
  locale = DE,
  inter-unit-product = \ensuremath{{\cdot}},
  detect-mode,             % Use the surrounding text font mode
  detect-family,           % Use the surrounding text font family
  detect-weight,           % Use the surrounding text font weight
  mode = text,             % Ensure that numbers and units are typeset in text mode
  %negative-powers = false, % Avoid using negative powers
  per-mode = symbol        % Use the division symbol for per units
}

\usepackage{tikz}
\usepackage{enumitem}
\usepackage{graphicx}
\usepackage{booktabs}
\usepackage{calc}
\usepackage{multicol}
\usepackage{amsmath}

\usepackage{tcolorbox}
\tcbuselibrary{skins}

\usepackage[dvipsnames]{xcolor}

%\usepackage[scaled]{helvet} % Arial-ähnliche Schriftart (Helvetica)
%\renewcommand\familydefault{\sfdefault} % Setzt die Standard-Schriftart auf sans-serif


\usetheme{Madrid} % This theme is visually appealing
%\usecolortheme{whale} % A color theme with blue tones
\usecolortheme{dolphin} % A color theme with blue tones


\newcommand{\mylogo}{
    \begin{tikzpicture}[remember picture,overlay]
        \node[anchor=north east, yshift=-2mm, fill=white, inner sep=2pt] at (current page.north east) % Verschiebe das Logo um 5mm nach unten
            {\includegraphics[height=0.4cm]{/Users/patricpf/Documents/repos/Bauschule-Baustoffe/template/bauschule-logo-5cm.png}}; % Grösse nach Bedarf anpassen
    \end{tikzpicture}
}

\logo{\mylogo}

\setbeamertemplate{headline}{%
  \begin{beamercolorbox}[wd=\textwidth,ht=0.5ex,dp=1ex]{upper separation line head}
  \end{beamercolorbox}
}

\setbeamercolor{upper separation line head}{bg=blau_bauschule}


% Anpassen des Frametitels, um ihn fett zu machen
\setbeamertemplate{frametitle}{%
    \nointerlineskip%
    \begin{beamercolorbox}[sep=0.3cm,left,wd=\paperwidth]{frametitle}%
        \usebeamerfont{frametitle}\bfseries\insertframetitle%
    \end{beamercolorbox}%
}

%\usebackgroundtemplate{
%    \includegraphics[width=\paperwidth,height=\paperheight]{my_pdf_copy_of_empty_ppt_template}
%}

% Setzen Sie hier den Namen des Fachs
\newcommand{\fachname}{Baustoffe}
\newcommand{\FinRes}[1]{\underline{\underline{#1}}}

% Anpassen der Fusszeile für Abstände zum Rand
\setbeamertemplate{footline}
{
  \leavevmode%
  \hbox{%
  \begin{beamercolorbox}[wd=.33\paperwidth,ht=2.25ex,dp=1ex,left,leftskip=1em]{author in head/foot}%
    \usebeamerfont{author in head/foot}\insertshortauthor
  \end{beamercolorbox}%
  \begin{beamercolorbox}[wd=.34\paperwidth,ht=2.25ex,dp=1ex,center]{title in head/foot}%
    \usebeamerfont{title in head/foot}\fachname % Hier wird der Fachname anstelle des Titels angezeigt
  \end{beamercolorbox}%
  \begin{beamercolorbox}[wd=.33\paperwidth,ht=2.25ex,dp=1ex,right,rightskip=1em]{date in head/foot}%
    \usebeamerfont{date in head/foot}\insertframenumber{} / \inserttotalframenumber\hspace*{2ex}
  \end{beamercolorbox}}%
  \vskip0pt%
}


\newtcolorbox{Merke}{
enhanced,
boxrule=0pt,frame hidden,
borderline west={4pt}{0pt}{red!75!black},
colback=white,
sharp corners,
before upper={\textbf{Merke:}\quad},
}

\newtcolorbox{Anwendungen}{
enhanced,
boxrule=0pt,frame hidden,
borderline west={4pt}{0pt}{brown!75!black},
colback=white,
sharp corners
}




% Colors
\definecolor{blau_bauschule}{RGB}{22,65,148}
\setbeamercolor{frametitle}{fg=blau_bauschule}
\setbeamertemplate{navigation symbols}{} % Remove navigation symbols


\newtcolorbox{Definition_BS}[1]{
enhanced,boxrule=1pt,
colback=green!5!white,
colframe=green!75!black,fonttitle=\bfseries, title = #1,
%after title={\hfill\colorbox{black}{Definition}}
}



\newtcolorbox{Masseinheit}[1]{
enhanced,
boxrule=1pt,colframe=blue,
colback=white,
sharp corners, 
colframe=blue!75!black,
title = #1, 
after title={\hfill\colorbox{blue}{Masseinheit}}
}


\newtcolorbox{myLösung}{
  enhanced,
  boxrule=1pt,
  colframe=gray!75!black, % Definiert die Farbe des Rahmens als dunkelgrau
  colback=gray!20, % Definiert die Hintergrundfarbe der Box als hellgrau
  %sharp corners, % Macht die Ecken der Box scharf (nicht abgerundet)
  title = {Lösung}, % Fest eingestellter Titel der Box
  after title={}, % Fügt das Label "Masseinheit" nach dem Titel hinzu
  coltitle=white, % Farbe des Titeltexts
  fonttitle=\bfseries % Schriftart des Titels
}


\usepackage{fontawesome}  % For check and cross symbols


% Set the title, author, and date
\title{\textbf{Nachbesprechung Prüfung: HTf-26: Metalle und Wärmedämmstoffe}}
\author{Patrick Pfändler}



\begin{document}
%Farben 
\definecolor{orangish}{wave}{620}


%Start of the slides
\frame{\titlepage}

\begin{frame}{Inhalt der Lektion}
    \tableofcontents
\end{frame}


\section{Notenübersicht}
\begin{frame}{Notenverteilung}
    \begin{table}[h!]
        \centering
        %\caption{Statistische Kennwerte der Messdaten}
        \label{tab:statistische_werte}
        \begin{tabular}{ll}
            \hline
            \textbf{{}} & \textbf{Note} \\ \hline
            Minimum & 3.50 \\
            Median & 4.70 \\
            Mittelwert & 4.55 \\
            Maximum & 5.50 \\
            Standardabweichung (Std) & 0.64 \\ \hline
        \end{tabular}
    \end{table}
\end{frame}


\begin{frame}{Formel für die Notenberechnung}
    \begin{itemize}
        \item[\textbullet] Note 1 mit 4 Punkten
        \item[\textbullet] Note 6 mit 35 (von 38.3) Punkten  
    \end{itemize}
\end{frame}

\section{Nachbesprechung der Prüfung}
\begin{frame}{Wahr und Falsch}

    Metalle sind feuerbeständig? $\Rightarrow$ FALSCH! 

    \vspace{\baselineskip} 

    Frischen des Roheisens senkt den \textbf{Sauerstoffgehalt}?
    $\Rightarrow$ FALSCH!

    \vspace{\baselineskip} 
    
    Metalle lassen sich häufig aufgrund ihres typischen Glanzes erkennen? 
    $\Rightarrow$ Richtig!
\end{frame}

\begin{frame}{Anordnung der Kohlenstoffatome}
    Beim Grauguss ordnen sich die Kohlenstoffe lamellenförmig an.

    \vspace{\baselineskip} 

    $\Rightarrow$ Richtig! (Fehler in meiner Lösung.)

    \vspace{\baselineskip} 

    \textit{Dies wurde bei der Korrektur berücksichtig!}


\end{frame}


\begin{frame}{Was ist legieren?}
    Mischen eines Nichtmetalles und eines Metalles?

    $\Rightarrow$ Richtig! 

    \textcolor{orangish}{Bestes Beispiel ist Stahl.}
\end{frame}


\begin{frame}{Erkennen des Metalles im Ankerkopf}
    \textcolor{orangish}{Ich hätte gerne Spannstahl gehört. (Stahl allein reichte nicht aus.)}

    \vspace{\baselineskip} 

    Ich habe die Frage nun leicht modifiziert.
\end{frame}


\begin{frame}{Deponie und Metalle}
    Für 1.5 Punkte musste einige Punkte erwähnt werden.

    \vspace{\baselineskip} 
    \begin{itemize}
        \item[\textbullet] Recycling möglich
        \item[\textbullet] Recyling teilw. günstiger als Abbau
        \item[\textbullet] "gibt Geld"
    \end{itemize}
\end{frame}


\begin{frame}{Spannungs-Dehnungsdiagramm von Stahl}
    \begin{figure}[h!]
        \centering
        \includegraphics[width=0.61\textwidth]{spannungsdehnungsdiagrammLsg.png}
        \caption{Spannungs-Dehnungsdiagramm von Stahl: GLEICH WIE in der Übung.}
        \label{fig:spannungs_dehnungsdiagramm}
    \end{figure}
\end{frame}

\begin{frame}{U-Wert-Berechnung}

    \textcolor{orangish}{Wäre aus meiner Sicht keine schwierige Frage gewesen; Aber zu wenig Punkte wurden geholt.}    

\end{frame}

\begin{frame}{Dämmstoffe: Wahr Falsch}
    \begin{table}[ht]
        \renewcommand{\arraystretch}{1.5} % Row height adjustment
        \setlength{\tabcolsep}{8pt}      % Column padding
        \centering
        \begin{tabular}{|p{0.6\textwidth}|c|c|}
            \hline
            \textbf{Beschreibung} & \textbf{Ja} & \textbf{Nein} \\ \hline
            Aerogel oder flexibler Hochleistungsdämmstoff auf Aerogelbasis ist gesundheitlich unbedenklich. &
            \textcolor{green!60!black}{\faCheckCircle} 67\% &
            \textcolor{red}{\faTimesCircle} 33\% \\ \hline
            Aus Altpapier lässt sich Dämmstoff herstellen. &
            \textcolor{green!60!black}{\faCheckCircle} 85\% &
            \textcolor{red}{\faTimesCircle} 15\% \\ \hline
        \end{tabular}
    \end{table}
\end{frame}

\begin{frame}{Dämmstoffe erkennen}
    \begin{itemize}
        \item Unterscheidung: Kokosfasern und Hanfdämmstoffe war schwierig.
        \item EPS vs. XPS: Unterscheidungsmerkmal die "Kügelchen"
    \end{itemize}

\end{frame}


\begin{frame}{Dämmstoffe: Wahr oder Falsch}
    \begin{table}[ht]
        \renewcommand{\arraystretch}{1.5} % Row height adjustment
        \setlength{\tabcolsep}{6pt}      % Column padding
        \centering
        \begin{tabular}{|p{0.5\textwidth}|c|c|}
            \hline
            \textbf{Beschreibung} & \textbf{Wahr} & \textbf{Falsch} \\ \hline
            Enthält mineralisches Bindemittel & 
            \textcolor{green!60!black}{\faCheckCircle} 92\% & 
            \textcolor{red}{\faTimesCircle} 8\% \\ \hline
            Kann Portlandzement enthalten & 
            \textcolor{green!60!black}{\faCheckCircle} 100\% & 
            \\ \hline
            Ist ein Holzfaserwerkstoff & 
            \textcolor{red}{\faTimesCircle} 83\% & 
            \textcolor{green!60!black}{\faCheckCircle} 17\% \\ \hline
            Wird nur in Innenräumen verbaut & 
            \textcolor{red}{\faTimesCircle} 50\% & 
            \textcolor{green!60!black}{\faCheckCircle} 50\% \\ \hline
            Ist asbestfrei & 
            \textcolor{green!60!black}{\faCheckCircle} 100\% & 
            \\ \hline
            Beständig gegen Verrottung & 
            \textcolor{green!60!black}{\faCheckCircle} 50\% & 
            \textcolor{red}{\faTimesCircle} 50\% \\ \hline
        \end{tabular}
    \end{table}
\end{frame}


\begin{frame}{Dämmen vs. Abdichten?}
    
    Was ist der Unterschied zwischen Dämmen und Abdichten?

    \vspace{\baselineskip} 

    Kurze Diskussion als Zweiergruppe (ca. 2 min).


\end{frame}

\begin{frame}{Berechne die Zugfestigkeit des Stahls}
    Aus dem Zugversuch mit einem Durchmesser 16 mm. 

    \vspace{\baselineskip} 

    GLEICH WIE in der Übung.

    \vspace{\baselineskip} 

    Hier wurden zu wenig Punkte geholt.
    
\end{frame}

\begin{frame}{Berechne die Zugfestigkeit des Stahls}

\begin{figure}[h!]
    \centering
    \includegraphics[height=0.72\textheight]{Spannung_Dehnung_Lektion.jpeg}
    \caption{Skizee aus der Lektion!}
    \label{fig:zugfestigkeit}
\end{figure}

\end{frame}

\begin{frame}{Feedback oder Fragen zur Prüfung?}
    
\end{frame}





\folieFragen
\end{document}