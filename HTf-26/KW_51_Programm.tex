%!TEX root = /Users/patricpf/Documents/repos/Bauschule-Baustoffe/HTf-26/KW_51_Programm.tex
%!TEX program = lualatex
\def\customoptions{aspectratio=169} % Exclude handout
% !TEX program = lualatex
\documentclass[aspectratio=169, 
handout,
]{beamer}


\PassOptionsToPackage{dvipsnames,svgnames}{xcolor}

%\usepackage[utf8]{inputenc}
%\usepackage[ngerman]{babel} % Schweizer Rechtschreibung

\usepackage{polyglossia}
\setdefaultlanguage[variant = swiss]{german}
\usepackage{fontspec}
\setmainfont{Times New Roman}
\setsansfont{Arial}


\usepackage{amsmath}
\usepackage{siunitx}
\sisetup{
  locale = DE,
  inter-unit-product = \ensuremath{{\cdot}},
  detect-mode,             % Use the surrounding text font mode
  detect-family,           % Use the surrounding text font family
  detect-weight,           % Use the surrounding text font weight
  mode = text,             % Ensure that numbers and units are typeset in text mode
  %negative-powers = false, % Avoid using negative powers
  per-mode = symbol        % Use the division symbol for per units
}

\usepackage{tikz}
\usepackage{enumitem}
\usepackage{graphicx}
\usepackage{booktabs}
\usepackage{calc}
\usepackage{multicol}
\usepackage{amsmath}

\usepackage{tcolorbox}
\tcbuselibrary{skins}

\usepackage[dvipsnames]{xcolor}

%\usepackage[scaled]{helvet} % Arial-ähnliche Schriftart (Helvetica)
%\renewcommand\familydefault{\sfdefault} % Setzt die Standard-Schriftart auf sans-serif


\usetheme{Madrid} % This theme is visually appealing
%\usecolortheme{whale} % A color theme with blue tones
\usecolortheme{dolphin} % A color theme with blue tones


\newcommand{\mylogo}{
    \begin{tikzpicture}[remember picture,overlay]
        \node[anchor=north east, yshift=-2mm, fill=white, inner sep=2pt] at (current page.north east) % Verschiebe das Logo um 5mm nach unten
            {\includegraphics[height=0.4cm]{/Users/patricpf/Documents/repos/Bauschule-Baustoffe/template/bauschule-logo-5cm.png}}; % Grösse nach Bedarf anpassen
    \end{tikzpicture}
}

\logo{\mylogo}

\setbeamertemplate{headline}{%
  \begin{beamercolorbox}[wd=\textwidth,ht=0.5ex,dp=1ex]{upper separation line head}
  \end{beamercolorbox}
}

\setbeamercolor{upper separation line head}{bg=blau_bauschule}


% Anpassen des Frametitels, um ihn fett zu machen
\setbeamertemplate{frametitle}{%
    \nointerlineskip%
    \begin{beamercolorbox}[sep=0.3cm,left,wd=\paperwidth]{frametitle}%
        \usebeamerfont{frametitle}\bfseries\insertframetitle%
    \end{beamercolorbox}%
}

%\usebackgroundtemplate{
%    \includegraphics[width=\paperwidth,height=\paperheight]{my_pdf_copy_of_empty_ppt_template}
%}

% Setzen Sie hier den Namen des Fachs
\newcommand{\fachname}{Baustoffe}
\newcommand{\FinRes}[1]{\underline{\underline{#1}}}

% Anpassen der Fusszeile für Abstände zum Rand
\setbeamertemplate{footline}
{
  \leavevmode%
  \hbox{%
  \begin{beamercolorbox}[wd=.33\paperwidth,ht=2.25ex,dp=1ex,left,leftskip=1em]{author in head/foot}%
    \usebeamerfont{author in head/foot}\insertshortauthor
  \end{beamercolorbox}%
  \begin{beamercolorbox}[wd=.34\paperwidth,ht=2.25ex,dp=1ex,center]{title in head/foot}%
    \usebeamerfont{title in head/foot}\fachname % Hier wird der Fachname anstelle des Titels angezeigt
  \end{beamercolorbox}%
  \begin{beamercolorbox}[wd=.33\paperwidth,ht=2.25ex,dp=1ex,right,rightskip=1em]{date in head/foot}%
    \usebeamerfont{date in head/foot}\insertframenumber{} / \inserttotalframenumber\hspace*{2ex}
  \end{beamercolorbox}}%
  \vskip0pt%
}


\newtcolorbox{Merke}{
enhanced,
boxrule=0pt,frame hidden,
borderline west={4pt}{0pt}{red!75!black},
colback=white,
sharp corners,
before upper={\textbf{Merke:}\quad},
}

\newtcolorbox{Anwendungen}{
enhanced,
boxrule=0pt,frame hidden,
borderline west={4pt}{0pt}{brown!75!black},
colback=white,
sharp corners
}




% Colors
\definecolor{blau_bauschule}{RGB}{22,65,148}
\setbeamercolor{frametitle}{fg=blau_bauschule}
\setbeamertemplate{navigation symbols}{} % Remove navigation symbols


\newtcolorbox{Definition_BS}[1]{
enhanced,boxrule=1pt,
colback=green!5!white,
colframe=green!75!black,fonttitle=\bfseries, title = #1,
%after title={\hfill\colorbox{black}{Definition}}
}



\newtcolorbox{Masseinheit}[1]{
enhanced,
boxrule=1pt,colframe=blue,
colback=white,
sharp corners, 
colframe=blue!75!black,
title = #1, 
after title={\hfill\colorbox{blue}{Masseinheit}}
}


\newtcolorbox{myLösung}{
  enhanced,
  boxrule=1pt,
  colframe=gray!75!black, % Definiert die Farbe des Rahmens als dunkelgrau
  colback=gray!20, % Definiert die Hintergrundfarbe der Box als hellgrau
  %sharp corners, % Macht die Ecken der Box scharf (nicht abgerundet)
  title = {Lösung}, % Fest eingestellter Titel der Box
  after title={}, % Fügt das Label "Masseinheit" nach dem Titel hinzu
  coltitle=white, % Farbe des Titeltexts
  fonttitle=\bfseries % Schriftart des Titels
}


% Set the title, author, and date
\title{\textbf{Lektionsprogramm HTf-26}}
\author{Patrick Pfändler}
\date{16. Dezember 2024}


\begin{document}

\frame{\titlepage}

\begin{frame}{Inhalt der Lektion}
	\tableofcontents
\end{frame}

\section{Carolabrücke}
\begin{frame}{Vorläufige Erkenntnisse zur Ursache und Hergang des Teileinsturzes der Carolabrücke}
    \begin{block}{Zwischenergebnisse}
        Die vorliegenden Zwischenergebnisse deuten darauf hin, dass wasserstoffinduzierte Spannungsrisskorrosion die Hauptursache für das Versagen ist.
        \footnote{\href{        \footnote{https://www.dresden.de/media/pdf/presseamt/2024_12_11_Carolabruecke_Zusammenfassung-Ergebnisse.pdf}}{www.dresden.de}}
    \end{block}

\end{frame}

\begin{frame}{Wasserstoffinduzierte Spannungsrisskorrosion (SCC)}
    \begin{Definition_BS}{Spannungsrisskorrosion}
        Wasserstoffinduzierte Spannungsrisskorrosion (SCC) ist ein Schadensmechanismus, der unter spezifischen Voraussetzungen in
        hochbelasteten Metallen, wie vergütetem Spannstahl (u.\,a. auch Hennigsdorfer Spannstahl), auftreten kann. Dabei diffundiert Wasserstoff in die innere Gefügestruktur und führt dort unter anhaltender mechanischer Spannung zu Mikrorissen, die sich fortschreitend ausbreiten und
        schlieddlich zum spröden Versagen des Stahls führen können. Mangelnder Schutz vor Feuchtigkeit, korrosive Umgebung oder
        Verarbeitungsfehler begünstigen diesen Prozess.
        \footnote{\href{        \footnote{https://www.dresden.de/media/pdf/presseamt/2024_12_11_Carolabruecke_Zusammenfassung-Ergebnisse.pdf}}{www.dresden.de}}
    \end{Definition_BS}
\end{frame}


\begin{frame}{Video zur Carolabrücke}
    \begin{block}{}
        \begin{itemize}
            \item[\textbullet] \href{https://www.youtube.com/watch?v=5J9Z1K1Z9f4}{Link zum Video}
        \end{itemize}
    \end{block}

\end{frame}

\begin{frame}{Wichtigste Konsequenzen}
    \begin{block}{}
        \begin{itemize}
            \item[\textbullet] Die \textbf{gesamte }Brücke wird für den Verkehr gesperrt und muss abgerissen werden.
            \item[\textbullet] Wasserstrasse darf nur nach der Installation eines Schallemissionssystems wieder freigegeben werden.
        \end{itemize}
    \end{block}
    \pause
    \begin{block}{Verschulden}
        Eine umfassende Aktenlage belegt, dass
        das Bauwerk innerhalb der geltenden Regelwerke bewertet und betrieben wurde.
    \end{block}
\end{frame}

\begin{frame}{Wichtigste Erkenntnisse}
    \begin{itemize}
        \item [\textbf{→}] \textbf{Haupteinsturzursache:} Wasserstoffinduzierte Spannungsrisskorrosion
        \item [\textbf{→}] \textbf{Konsequenz:} Einsturz nicht vorhersagbar, da keine ausgeprägte Rissbildung.
        \item [\textbf{→}] \textbf{Schuldfrage:} Gesetzliche Vorgaben eingehalten, keine Versäumnisse.
        \item [\textbf{→}] \textbf{Spannstahldefekte:} Über 68 Prozent der Spannglieder in der Fahrbahnplatte von Zug C waren an der Bruchstelle stark geschädigt. 
        \item [\textbf{→}] \textbf{Massnahmen:} Abriss der \textbf{gesamten} Brücke, temporäre Installation eines Schallemissionssystems. Bau einer neuen Brücke.
        \item [\textbf{→}] \textbf{Tausalze:} Sogenannte chloridinduzierte Korrosion hat an Brückenzug C stattgefunden, war jedoch nicht ursächlich für den Einsturz. 
    \end{itemize}


\end{frame}



\section{Betoninstandsetzung}
\subsection{Repetition}
\begin{frame}{Repetition: Von letzter Woche}
	\begin{block}{Mögliche Schäden an Betonbauwerken: Beton}
		\begin{itemize}
			\item[\textbullet] Mechanisch
			\item[\textbullet] Chemisch
			\item[\textbullet] Physikalisch
		\end{itemize}
	\end{block}
\end{frame}
\begin{frame}{Repetition: Von letzter Woche}
	\begin{block}{Mögliche Schäden an Betonbauwerken: Bewehrung}
		\begin{itemize}
			\item[\textbullet] Karbonatisierung
			\item[\textbullet] Korrosionsfördernde Verunreinigungen
			\item[\textbullet] Streuströme
		\end{itemize}
	\end{block}
\end{frame}

\subsection{Beispiele aus eurer Praxis?}
\begin{frame}{Beispiele von eurer Arbeit?}
	\begin{block}{}
		\begin{itemize}
			\item[\textbullet] Welche Schäden an Betonbauwerken habt ihr schon gesehen?
			\item[\textbullet] Wie wurden diese behoben?
			\item[\textbullet] Wie könnt ihr es in der Zukunft vermeiden resp. verbessern?
		\end{itemize}
	\end{block}
\end{frame}

\begin{frame}{Definition Begriff: Korrosion}
	\begin{Definition_BS}{Korrosion}
		Korrosion ist aus technischer Sicht die Reaktion eines Werkstoffs mit seiner Umgebung, die
		eine messbare Veränderung des Werkstoffs bewirkt. Korrosion kann zu einer Beeinträchtigung
		der Funktion eines Bauteils oder Systems führen. Eine durch Lebewesen verursachte Korrosion
		wird als Biokorrosion bezeichnet. \footnote{Quelle: Wikipedia}
	\end{Definition_BS}
\end{frame}

\subsection{Korrosionsbeständige Bewehrung}

\begin{frame}{Lernziele: Korrosionsbeständige Bewehrung}
	\begin{myLernziele}
		\begin{itemize}
			\item[\textbullet] Kenntnisse über die Möglichkeiten korrosionsbeständiger Bewehrungsmaterialien
			      \begin{itemize}
			      	\item Nicht-rostender Betonstahl
			      	\item Faserbewehrung
			      	      \begin{itemize}
			      	      	\item Glasfaser-Bewehrung
			      	      	\item Carbonfaser-Bewehrung
			      	      	\item Basalfaser-Bewehrung
			      	      \end{itemize}
			      \end{itemize}
		\end{itemize}
	\end{myLernziele}
\end{frame}

\begin{frame}{Hauptursache für Schädigung}
    \begin{Fragenblock}
        Was ist die Hauptursache für die Schädigung von Betonbauwerken?
        
        \begin{itemize}
            \item[\faSquare] Chloride
            \item[\faSquare] Karbonatisierung
            \item[\faSquare] Frost-Tausalz
            \item[\faSquare] Kombination aus anderen Schädigungsmechanismen
        \end{itemize}
        
    \end{Fragenblock}
\end{frame}
    
\begin{frame}{Hauptursache für Schädigung}
    \begin{Fragenblock}
        Was ist die Hauptursache für die Schädigung von Betonbauwerken?
        
        \begin{itemize}
            \item[\textcolor{green!70!black}{\faCheckSquare}] Chloride
            \item[\faSquare] Karbonatisierung
            \item[\faSquare] Frost-Tausalz
            \item[\faSquare] Kombination aus anderen Schädigungsmechanismen
        \end{itemize}
        
    \end{Fragenblock}
\end{frame}

\begin{frame}{Lebensdauermodell}
    \begin{Fragenblock}
        Welcher Stahl hatte im gezeigten Schema die längere Lebensdauer (rote Linie)?
        
        \begin{itemize}
            \item[\faSquare] Unlegierter Betonstahl
            \item[\faSquare] Nichtrostender Betonstahl
        \end{itemize}
        
    \end{Fragenblock}
\end{frame}
    
\begin{frame}{Lebensdauermodell}
    \begin{Fragenblock}
        Welcher Stahl hatte im gezeigten Schema die längere Lebensdauer (rote Linie)?
        
        \begin{itemize}
            \item[\faSquare] Unlegierter Betonstahl
            \item[\textcolor{green!70!black}{\faCheckSquare}] Nichtrostender Betonstahl
        \end{itemize}
        
    \end{Fragenblock}
\end{frame}

\begin{frame}{SIA Merkblatt 2029}
    \begin{Definition_BS}{Nichtrostender Betonstahl}
        Die Gruppe der nichtrostenden Betonstähle umfasst Stahlsorten mit einem Chromgehalt von mindestens 10.5 Massen-Prozent.
    \end{Definition_BS}
\end{frame}

\begin{frame}{Klassifizierung Korrosionswiderstand}
    \begin{Definition_BS}{Wirksumme (PREN)}
        Die Wirksumme (PREN) ist ein Näherungsmaß für den Widerstand gegen Lochkorrosion. Sie wird nach folgender Formel berechnet:
        \begin{equation*}
            \text{PREN} = \text{Cr} + 3.3 \cdot \text{Mo} + 16 \cdot \text{N}
        \end{equation*}
    \end{Definition_BS}
    

    \textbf{Klassifizierung der Stahlsorten:}
    \begin{itemize}
        \item [\textbullet] \textbf{Ferritische Stahlsorten:} \( n = 0 \)
        \item [\textbullet] \textbf{Duplex-Stahlsorten:} \( n = 16 \)
        \item [\textbullet] \textbf{Austenitische Stahlsorten:} \( n = 30 \)
    \end{itemize}
\end{frame}



\begin{frame}{Beispiel für die Berechnung von PREN}
    \textbf{Gegeben ist ein Stahl mit:}
    \begin{itemize}
        \item Chrom (Cr): \( 18\,\% \)
        \item Molybdän (Mo): \( 2\,\% \)
        \item Stickstoff (N): \( 0.15\,\% \)
    \end{itemize}
    \hspace{20pt}
    \begin{Fragenblock}
        Berechnen Sie den PREN-Wert für diesen Stahl.
    \end{Fragenblock}
\end{frame}
    
\begin{frame}{Lösung}
  \begin{myLösung}
        \textbf{Berechnung:}
        \begin{align*}
            \text{PREN} &= \text{Cr} + 3.3 \cdot \text{Mo} + 16 \cdot \text{N} \\
            &= 18 + 3.3 \cdot 2 + 16 \cdot 0.15 \\
            &= 18 + 6.6 + 2.4 \\
            &= 27
        \end{align*}

        \pause
        \textbf{Interpretation:} Mit einem PREN-Wert von \( 27 \) zeigt diese Stahlsorte einen moderaten Widerstand gegen Lochkorrosion und fällt in die Kategorie \textbf{Duplex-Stahlsorten}.
    \end{myLösung}
\end{frame}

        


\begin{frame}{Klassifizierung Korrosionswiderstand}
    \begin{Definition_BS}{Korrosionswiderstandsklassen (KWK)}
        Die Einteilung eines (nichtrostenden) Betonstahls in die KorrosionswiderstandSklassen KWK (0 - 4) wird aufgrund seiner Wirksumme vorgenommen.
    \end{Definition_BS}
\end{frame}

\begin{frame}{Korrosionswiderstandsklassen (KWK)}
    \begin{table}[ht]
        \centering
        \renewcommand{\arraystretch}{1.5} % Erhöht den Zeilenabstand für bessere Lesbarkeit
        \begin{tabular}{|c|c|p{10cm}|}
            \hline
            \rowcolor{blue!20} \textbf{KWK} & \textbf{Wirksumme} & \textbf{Bemerkungen / typische Vertreter} \\
            \hline
            0 & 0--9 & Unlegierter oder niedrig legierter Betonstahl \\
            \hline
            1 & 10--16 & Chromstähle \\
            \hline
            2 & 17--22 & Chromnickelstähle \\
            \hline
            3 & 23--30 & Chromnickelstähle mit Molybdän \\
            \hline
            4 & $\geq 31$ & Stahlstorten mit erhöhtem Gehalt an Chrom und/oder Molybdän \\
            \hline
        \end{tabular}
        \caption{Quelle: SIA Merkblatt 2029, Tabelle 1}
        \label{tab:kwk}
    \end{table}
\end{frame}


\begin{frame}{Korrosionswiderstandsklassen (KWK) - Werkstoffe}
    \begin{table}[ht]
        \centering
        \renewcommand{\arraystretch}{1.2} % Erhöht den Zeilenabstand für bessere Lesbarkeit
        \begin{tabular}{|c|c|l|c|c|c|c|}
            \hline
            \rowcolor{blue!20} \textbf{KWK} & \textbf{Werkstoff-Nr.} & \textbf{Kurzbeschreibung} & \textbf{Cr, M\%} & \textbf{Mo, M\%} & \textbf{N, M\%} & \textbf{WS} \\
            \hline
            1 & 1.4003 & X2CrNi 12 / X2Cr 11 & 10.5 & - & - & 11 \\
            \hline
            1 & Top 12 (1.4003) & X2CrNi 12 / X2Cr 11 & 12.1 & 0.5 & - & 13 \\
            \hline
            2 & 1.4301 & X5CrNi 18-10 & 17 & - & - & 17 \\
            \hline
            3 & 1.4401 & X5CrNiMo 17-12-2 & 16.5 & 2 & - & 23 \\
            \hline
            3 & 1.4429 & X2CrNiMoN 17-13-3 & 16.5 & 2.5 & 0.12 & 27 \\
            \hline
            4 & 1.4462 & X2CrNiMoN 22-5-3 & 21 & 2.5 & 0.10 & 31 \\
            \hline
            4 & 1.4529 & X1NiCrMoCuN 25-20-7 & 19 & 6 & 0.15 & 41 \\
            \hline
        \end{tabular}
        \caption{Quelle: SIA Merkblatt 2029, Tabelle 2; Steeltec-group, Top 12 Technical Datasheet}
        \label{tab:kwk_werkstoffe}
    \end{table}
\end{frame}

\begin{frame}{Beispiel zur Einteilung in die KWK}
    \begin{Fragenblock}
        Welche Korrosionswiderstandsklasse (KWK) hat die Legierung von vorher? (Wirksumme = 27)
    \end{Fragenblock}


    \pause
    \begin{myLösung}
        \textbf{Lösung:}
        \begin{itemize}
            \item[\faCheckSquare] KWK 3
        \end{itemize}
    \end{myLösung}
\end{frame}

\begin{frame}{Vorteile von nichtrostendem Betonstahl}
\begin{Fragenblock}
    Welches sind Vorteile von nichtrostendem Betonstahl bei Betonbauwerken? (Hinweis: Denke an die Exposition)
\end{Fragenblock}

\pause

\begin{myLösung}
    \textbf{Vorteile:}
    \begin{itemize}
        \item[\faCheckSquare] Geringere Überdeckung bei gleicher Lebensdauer möglich. $\Rightarrow$ schlankere Bauteile, weniger Betonverbrauch möglich
    \end{itemize}
\end{myLösung}
\end{frame}

\begin{frame}{Wahl der Korrosionswiderstandsklasse}

    \begin{figure}[h!bt]
        \centering
        \includegraphics[width=1\linewidth]{/Users/patricpf/Documents/repos/Bauschule-Baustoffe/HTf-26/Bilder/Einteilung_Wirksumme.png}
        \caption{Quelle: SIA Merkblatt 2029, Tabelle 3}
    \end{figure}
    
\end{frame}


\begin{frame}{Korrosionsbeständige Bewehrung}
	\begin{block}{Video}
		\begin{itemize}
			\item [\textbullet] Weiter ab 15 min 
		\end{itemize}
	\end{block}
\end{frame}






\begin{frame}{Uploads auf Teams}
	\begin{itemize}
		\item[\textbullet] keine
	\end{itemize}
	        
\end{frame}
    
    
    
    
    

% \begin{frame}{Fragen zur Prüfung?}
        
% \end{frame}
    
\naechstePruefung{13.01.2024 }{Holz-und Holzwerkstoffe, Natursteine}
\folieFragen


\end{document}