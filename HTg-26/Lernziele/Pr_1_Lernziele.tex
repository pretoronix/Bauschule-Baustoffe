% !TEX root = /Users/patricpf/Documents/repos/Bauschule-Baustoffe/HTg-26/Lernziele/Pr_1_Lernziele.tex
% !TEX program = lualatex
% !TEX program = lualatex
\documentclass[
    %answers,
    a4paper,ngerman,12pt, addpoints]{exam}

\usepackage[utf8]{inputenc}
%\usepackage[T1]{fontenc}
%\usepackage[ngerman]{babel}


\usepackage{polyglossia}
\setdefaultlanguage[variant = swiss]{german}
\usepackage{fontspec}
\setmainfont{Aptos} % Bauschule CI Manual
\setsansfont{Aptos} % Bauschule CI Manual


\usepackage[ a4paper,
 total={165mm,250mm},
 left=25mm,
 top=25mm,
 headsep=10mm
 %footsep=12mm
 %,showframe
  ]{geometry}

\usepackage{graphicx}
\usepackage{siunitx}
\usepackage{booktabs} % schöne Tabellen
\usepackage{float}
\floatplacement{figure}{H}
\usepackage{xcolor}
\usepackage{pdfpages}
\usepackage{enumitem}
\usepackage{mdframed} % Boxen
\usepackage{amsmath,amssymb}
\usepackage{tcolorbox}
\usepackage{lastpage} % For the total number of pages
\usepackage{gensymb}
\usepackage{xspace}
\usepackage{tabularx}
\usepackage{multicol}
\usepackage[
    version=3,
    arrows=pgf-filled,
]{mhchem} % für chemische Formeln
%\usepackage{microtype}
\usepackage{subfigure}
\usepackage[hidelinks]{hyperref}
\usepackage{cleveref}
\usepackage{luacode}
\usepackage{amsmath}
\usepackage{textcomp}


\sisetup{
  locale = DE,
  inter-unit-product = \ensuremath{{\cdot}},
  detect-all,
}

% Colors
\definecolor{blau_bauschule}{RGB}{22,65,148}
\CorrectChoiceEmphasis{\color{blau_bauschule}}
\SolutionEmphasis{\color{blau_bauschule}}

\setlength{\parindent}{0em} % Verhindert einrücken
\setlength\linefillheight{0.3in}


%% COMMMANDS
\author{Patrick Pfändler}
\newcommand{\dozent}{Patrick Pfändler}
\newcommand{\fach}{Baustoffe}


\newcommand{\punkte}[1]{%
    \begin{infobox}%
        #1
    \end{infobox}}%
\newcommand{\FinRes}[1]{\underline{\underline{#1}}}

\newmdenv[linecolor=black,backgroundcolor=gray!15,frametitle={Punktverteilung},leftmargin=1cm,rightmargin=1cm]{infobox}

\newcommand{\pagebreaksol}{
    \ifprintanswers
        \clearpage
    \else
        {}
    \fi
}

\newcommand{\pagebreakexam}{
    \ifprintanswers
        {}
    \else
        \clearpage
    \fi
}

\SolutionEmphasis{\color{blau_bauschule}}
\makeatletter%
\newcommand{\solutiontable}[1]{\ifprintanswers\begingroup\Solution@Emphasis#1\if@shadedsolutions%
            {\cellcolor{SolutionColor}}%
        \else%
        \fi\endgroup\else\phantom{#1}\fi}%
\makeatother%

\newcommand{\myNmm}[1]
{
    \sisetup{per-mode=symbol}
    \SI{#1}{\newton\per\mm\squared}
}

\renewcommand{\thequestion}{\fontsize{12pt}{2pt} \selectfont  \bfseries \arabic{question}}
\sisetup{per-mode=symbol}



%% Translation

\pointpoints{Punkt}{Punkte}
\bonuspointpoints{Bonuspunkt}{Bonuspunkte}
\renewcommand{\solutiontitle}{\noindent\textbf{Lösung:}\enspace}
\chqword{Frage}
\chpgword{Seite}
\chpword{Punkte}
\chbpword{Bonus Punkte}
\chsword{Erreicht}
\chtword{Gesamt}
\hpword{Punkte:}
\hsword{Ergebnis:}
\hqword{Aufgabe:}
\htword{Summe:}


\renewcommand{\questionshook}{%
  %\setlength{\leftmargin}{0pt}% removes the indentation from the left
  \setlength{\labelwidth}{1.25cm}% adjusts label width
  \setlength{\itemindent}{0cm}% aligns the start of the item with the above
  \setlength{\labelsep}{0.25cm}% space between the label and the item text
}




%% header and footer
\pagestyle{headandfoot}
\firstpageheadrule
\runningheadrule

% Adjust the font size for the header
\firstpageheader{\fontsize{9}{11}\selectfont\fach}{}{\fontsize{9}{11}\selectfont\dozent \\ \blattname}
\runningheader{\fontsize{9}{11}\selectfont\fach}{}{\fontsize{9}{11}\selectfont\dozent \\ \blattname}

% Adjust the font size for the footer
\firstpagefooter{\includegraphics[width=2.5cm]{bauschule-logo-5cm.png}}{}{\fontsize{9}{11}\selectfont\thepage\,/\,\pageref{LastPage}}
\runningfooter{\includegraphics[width=2.5cm]{bauschule-logo-5cm.png}}{}{\fontsize{9}{11}\selectfont\thepage\,/\,\pageref{LastPage}}



\newcommand{\blattname}{Prov. Lernziele: Prüfung 1: Physik, Chemie und Bindemittel}
\newcommand{\nrPruefung}{ersten}
\newcommand{\Added}[1]{\textcolor{blue}{#1}}


%\printanswers
\begin{document}
\section*{\blattname}
\subsection*{Tipps zur Prüfungsvorbereitung}
Hyperlinks funktionieren nicht in Teams nicht immer zuverlässig. $\rightarrow$ Dokument herunterladen.
\begin{itemize}
	\item Durchgehen der Aufgabenblätter (inkl. Lösungen)
% 	\begin{itemize}[]

% 		\item \href{https://forms.office.com/Pages/ResponsePage.aspx?id=HsbbSHAOrE6HJuK4duaJwdUERlKbBqZKkaxTc87ge2NUNkdGRTVJS0NaVkRRMTBMMkoxNFpCUVhIRy4u}{Dämmstoffe Teil 2 (Forms)}
% 		\item \href{https://forms.office.com/Pages/ResponsePage.aspx?id=HsbbSHAOrE6HJuK4duaJwdUERlKbBqZKkaxTc87ge2NUOVc0UVJWTlREMTFJRlI4TE1YMzI2WEYxTC4u}{Metalle Teil 1 (Forms)}
% 		\item \href{https://forms.office.com/Pages/ResponsePage.aspx?id=HsbbSHAOrE6HJuK4duaJwdUERlKbBqZKkaxTc87ge2NURjBRS1VLNUxSM0M2NUxLNThGT0ZFMzNOTy4u}{Metalle Teil 2 (Forms)}
% 	\end{itemize}
 	\item Folien und/oder Skript nochmals durchgehen
 	\item Lösen der \href{https://www.classtime.com/code/CXR262}{Musterprüfung}
\end{itemize}

\subsection*{Informationen zur Prüfung}

\begin{description}[leftmargin=!,labelwidth=\widthof{Hinweise zur Bearbeitung...},font=\normalfont]
%\item [Prüfungsmodus] Sämtliche Unterlagen (gedruckt und/oder digital), Internet, keine Kommunikation zu anderen Personen (wie an der Fachabschlussprüfung)
\item [Prüfungsmodus] Online-Prüfung, Open-Book-Prüfung mit Classtime
\item  [Prüfungsdauer] ca. 40 min
\item [Empfohlene Hilfsmittel] Taschenrechner
\item [Anzahl Punkte] Die Maximalpunktzahl der Prüfung sind ca. 40.  Fürs Zeitmanagement, es sollte ca. 1 Punkt pro Minute erreicht werden. Für die Maximalnote werden i.d.R. nicht sämtliche Punkte benötigt.
\item [Bewertung] Die Prüfung wird halb-automatisch ausgewertet. Die Prüfungen werden nach der Prüfung korrigiert und die Resultate werden in Teams-Chat als PDF versendet.
\item [Hinweise zur Bearbeitung] Geben Sie zumindest beim Schlussresultat eine resp. die verlangte Einheit an. Ohne Angabe von Einheiten kann i.d.R. nicht maximale Punktzahl der Aufgabe erreicht werden. \\ Bei Multiple-Choice-Aufgaben führen falsche Kreuze nicht zu Punktabzug. Bei grob falschen Antworten kann ein Punktabzug erfolgen. \\ Die Bearbeitungszeit der Prüfung ist i.d.R. äusserst knapp bemessen! $\Rightarrow$ Lösen Sie zuerst, was Sie direkt wissen und kommen Sie später auf die schwierigeren Fragen zurück.
\end{description}


\pagebreak
\subsection*{Lernziele}
Diese Lernziele geben einen groben Überblick über den Stoffumfang der \nrPruefung\xspace Prüfung im Fach Baustoffe.



\subsection*{Grundlagen der Physik und der Chemie}

Die Studierenden kennen: 

\begin{itemize}[noitemsep]
	\item die Einteilung der Naturwissenschaften.
	\item Kriterien bei der Materialselektion von Baustoffen.
	\item Eigenschaftsklassen von Baustoffen
	\item die Unterschiede zwischen Chemie und Physik.
	\item  können Beispiele zu physikalischen und chemischen Prozessen aus dem Alltag und dem Bauwesen aufzählen.
	%\item Teilgebiete der Physik und der Chemie.
	\item die Einteilung der Materie.
	\item das Periodensystem und können relevante Daten aus dem Periodensystem der Elemente lesen.
	\item den Atomaufbau (inkl. der Begriffe Elektron, Proton und Neutron). 
	\item die Begriffe Isotope, Ionen, Masseanzahl und Ordnungszahl und können diese im Kontext anwenden.
	\item die drei wichtigsten Typen von chemischen Verbindungen, sowie deren grundlegenden Eigenschaften.
	%\item die Begriffe Molekül und Verbindung und können diese Unterscheiden und Beispiele nennen.
	\item die Begriffe chemische Reaktion, Analyse und Synthese und können Beispiele nennen.
	\item die Unterschiede zwischen einer endothermen und einer exothermen Reaktion und können Beispiele (u.a. aus dem Bauwesen) nennen.
	\item die Aggregatzustände, inklusive die Eigenschaften der einzelnen Aggregatzustände.
	\item die Dichteanomalie von Wasser. (siehe auch Aufgabe zur Dichteanomalie von Wasser).
	\item die Einheit Kelvin und können einfache Berechnungen durchführen.
	\item die pH-Wert-Skala. Des Weiteren sind die Studierenden in der Lage einfache pH-Wert-Be\-rechnungen durchzuführen und Beispiele aus der Baupraxis nennen.
	\item das SI-System (inkl. Formelzeichen, Zahlenwert und zugehörige Einheit).
	\item das Dezimalsystem. 
	\item können Messdaten in Grafiken aufzeichnen. 
	\item können einfache Berechnungen mit folgenden Themen durchführen: 
	\begin{itemize}[noitemsep]
		\item Länge
		\item Zeit 
		\item Masse 
		\item Kraft 
		\item Arbeit, Energie und Leistung 
		\item Dichte (inkl. Reindichte, Rohdichte und Schüttdichte) 
		\item Druck
	\end{itemize}
	\item die Ritzhärte nach Mohs, die Härte nach Brinell und die Härte nach Rockwell.
	%\item die Begriffe offenporig und geschlossenporig.
	%\item die Luftfeuchtigkeit und die relative Luftfeuchtigkeit.
	%\item die Begriffe Dampfdruckgefälle und Diffusionswiderstand.
\end{itemize}
\subsubsection*{Bindemittel}

Die Studierenden kennen: 

\begin{itemize}[noitemsep]
	\item Definition des Begriffes "Bindemittel", Zement, sowie Kohäsion und Adhäsion.
	\item die Einteilung der Bindemittel,die Eigenschaften der unterschiedlichen Bindemittel und können diese bedarfsgerecht selektieren. 
	\item das Funktionsschema, wie Bindemittel mit weiteren Grundstoffen vermengt werden können.
	\item erste Anwendungen von Bindemitteln.
	\item kennen die (Haupt-)Bestandteile von Mischzementen und deren Eigenschaften.
	\item die Einteilung der Zemente nach SN EN 197-1 und können aus dem Zementnamen auf dessen Eigenschaften schliessen.
	\item die Hydration des Zementes und dessen Auswirkungen, sowie die beiden wichtigsten Produkte bei der Zementhydratation (CSH und Calciumhydroxid).
	\item Probleme, welche bei der Zementlagerung möglich sind und können dementsprechend reagieren.
	\item die Herstellung von Zement und den  Kalkkreislauf.
	\item die Eigenschaften, Anwendungsbereiche und Herstellung folgender weiterer Bindemittel: Hydraulischer Kalk, Baugibs, Magnesit, Schamottmörtel, Polymerbeton, bituminöse Bindemittel.
	\item Untersuchungsmethoden für bituminöse Bindemittel.
	\item ökologische und gesundheitliche Aspekte von Bindemitteln und kennen die unterschiedlichen Entsorgungsmöglichkeiten von Bindemitteln. 
\end{itemize}
%\subsubsection*{Kunststoffe}
Die Studierenden kennen: 

\begin{itemize}[noitemsep]
	\item die Unterschiede und Eigenschaften, Aufbau, Verarbeitungsverfahren, Vor- und Nachteile von Kunststoffen (inkl. Naturkautschuk) zu anderen Baustoffen. 
	\item die Einteilung der Baustoffe (insbesondere Kunststoffe).
	\item die häufigsten Elemente, welche bei Kunststoffen vorkommen.
	\item den Aufbau der Kunststoffe.
	\item Möglichkeiten zur Beeinflussung der Eigenschaften von Kunststoffen (Hilfsstoffe, Zusatzstoffe, etc.)
	\item Kunststoffen nach ihren thermisch-mechanisch, nach ihrem Herstellungsprozess oder Verwendungsmöglichkeiten im Bauwesen unterteilen.
	\item unterschiedliche Kunststoffgruppen.
	\item Beispiele zu den unterschiedlichen Kunststoffen.
	\item mögliche Gefahren von Kunststoffen für die Umwelt.
	\item Möglichkeiten für das Recycling und Entsorgung von Kunststoffen.
	\item Gefahren von Halogenen und Produkte, welche beim Verbrennen Halogene ausstossen können.
	%\item wichtige Normen und Empfehlungen zu den Kunststoffen.
\end{itemize}
%\subsubsection*{Abdichtungsmaterialien und Klebstoffe}

Die Studierenden kennen: 

\begin{itemize}[noitemsep]
	\item den Zweck von Abdichtungen und unterschiedliche Abdichtungskonzepte (z.B. für Flächen oder Fugen).
	\item die Begriffe Hydrophobierung, Imprägnierung und Beschichtung.
	\item erdberührte Schutzsysteme (wie z.B. Schutzanstrich, Schutzbeschichtungen und Abdichtungen), deren Anwendungszweck und Anwendungsbereiche und die Wirkungsweise.
	\item bewitterte Schutzsysteme (wie z.B. Imprägnierungsmittel, Beschichtungen, Gussasphalt), deren Anwendungszweck und Anwendungsbereiche und die Wirkungsweise.
	\item unterschiedliche Gewässerschutzsysteme, deren Anwendungszweck und Anwendungsbereiche und die Wirkungsweise.
	\item unterschiedliche Baupapiere und Folien, deren Anwendungszweck und mögliche Anwendungsbereiche sowie die Wirkungsweise.
	\item die Inhalte der Norm SIA 270 (Anwendungsgruppen, Dichtungsklassen, technische Abkürzungen, ...)
	\item einige Fabrikationsprozesse für die Herstellung von Kunststoffdichtungsbahnen.
	\item unterschiedliche Verbindungstechniken für Kunststoffdichtungsbahnen.
	\item Anwendungsbeispiele zu Kunststoffdichtungsbahnen (u.a. SIA 281).
	\item die Begriffe Bitumenbahn, Bitumen-Dichtungsbahn, Polymerbitumen-Dichtungsbahn, AC-Be\-ständig\-keit, Elastomerbitumen, Oxidationsbitumen, Plastomerbitumen, Träger\-einlage und Ob\-erflächenausrüstung.
	\item das Bezeichnungsschema von Bitumendichtungsbahnen und können dieses Anwenden.
	\item Flüssigkunststoff-Abdichtungen, sowie die Anwendungsmöglichkeiten, Eigenschaften und zugehörige Begriffe (Haftvermittler, Solldicke, Nutzschicht).
	\item das Bezeichnungsschema von Gussasphalt und können dieses anwenden.
	\item unterschiedliche Typen von Fugen und mögliche Abdichtungssystem zu den unterschiedlichen Fugentypen.
	\item unterschiedliche Arten von Klebstoffen und mögliche Anwendungsbereiche.
\end{itemize}







\end{document}
