
% !TEX program = lualatex
% !TEX root = /Users/patricpf/Documents/repos/Bauschule-Baustoffe/HTg-26/Programm/KW_33_Programm.tex

% !TEX program = lualatex
\documentclass[aspectratio=169, 
handout,
]{beamer}


\PassOptionsToPackage{dvipsnames,svgnames}{xcolor}

%\usepackage[utf8]{inputenc}
%\usepackage[ngerman]{babel} % Schweizer Rechtschreibung

\usepackage{polyglossia}
\setdefaultlanguage[variant = swiss]{german}
\usepackage{fontspec}
\setmainfont{Times New Roman}
\setsansfont{Arial}


\usepackage{amsmath}
\usepackage{siunitx}
\sisetup{
  locale = DE,
  inter-unit-product = \ensuremath{{\cdot}},
  detect-mode,             % Use the surrounding text font mode
  detect-family,           % Use the surrounding text font family
  detect-weight,           % Use the surrounding text font weight
  mode = text,             % Ensure that numbers and units are typeset in text mode
  %negative-powers = false, % Avoid using negative powers
  per-mode = symbol        % Use the division symbol for per units
}

\usepackage{tikz}
\usepackage{enumitem}
\usepackage{graphicx}
\usepackage{booktabs}
\usepackage{calc}
\usepackage{multicol}
\usepackage{amsmath}

\usepackage{tcolorbox}
\tcbuselibrary{skins}

\usepackage[dvipsnames]{xcolor}

%\usepackage[scaled]{helvet} % Arial-ähnliche Schriftart (Helvetica)
%\renewcommand\familydefault{\sfdefault} % Setzt die Standard-Schriftart auf sans-serif


\usetheme{Madrid} % This theme is visually appealing
%\usecolortheme{whale} % A color theme with blue tones
\usecolortheme{dolphin} % A color theme with blue tones


\newcommand{\mylogo}{
    \begin{tikzpicture}[remember picture,overlay]
        \node[anchor=north east, yshift=-2mm, fill=white, inner sep=2pt] at (current page.north east) % Verschiebe das Logo um 5mm nach unten
            {\includegraphics[height=0.4cm]{/Users/patricpf/Documents/repos/Bauschule-Baustoffe/template/bauschule-logo-5cm.png}}; % Grösse nach Bedarf anpassen
    \end{tikzpicture}
}

\logo{\mylogo}

\setbeamertemplate{headline}{%
  \begin{beamercolorbox}[wd=\textwidth,ht=0.5ex,dp=1ex]{upper separation line head}
  \end{beamercolorbox}
}

\setbeamercolor{upper separation line head}{bg=blau_bauschule}


% Anpassen des Frametitels, um ihn fett zu machen
\setbeamertemplate{frametitle}{%
    \nointerlineskip%
    \begin{beamercolorbox}[sep=0.3cm,left,wd=\paperwidth]{frametitle}%
        \usebeamerfont{frametitle}\bfseries\insertframetitle%
    \end{beamercolorbox}%
}

%\usebackgroundtemplate{
%    \includegraphics[width=\paperwidth,height=\paperheight]{my_pdf_copy_of_empty_ppt_template}
%}

% Setzen Sie hier den Namen des Fachs
\newcommand{\fachname}{Baustoffe}
\newcommand{\FinRes}[1]{\underline{\underline{#1}}}

% Anpassen der Fusszeile für Abstände zum Rand
\setbeamertemplate{footline}
{
  \leavevmode%
  \hbox{%
  \begin{beamercolorbox}[wd=.33\paperwidth,ht=2.25ex,dp=1ex,left,leftskip=1em]{author in head/foot}%
    \usebeamerfont{author in head/foot}\insertshortauthor
  \end{beamercolorbox}%
  \begin{beamercolorbox}[wd=.34\paperwidth,ht=2.25ex,dp=1ex,center]{title in head/foot}%
    \usebeamerfont{title in head/foot}\fachname % Hier wird der Fachname anstelle des Titels angezeigt
  \end{beamercolorbox}%
  \begin{beamercolorbox}[wd=.33\paperwidth,ht=2.25ex,dp=1ex,right,rightskip=1em]{date in head/foot}%
    \usebeamerfont{date in head/foot}\insertframenumber{} / \inserttotalframenumber\hspace*{2ex}
  \end{beamercolorbox}}%
  \vskip0pt%
}


\newtcolorbox{Merke}{
enhanced,
boxrule=0pt,frame hidden,
borderline west={4pt}{0pt}{red!75!black},
colback=white,
sharp corners,
before upper={\textbf{Merke:}\quad},
}

\newtcolorbox{Anwendungen}{
enhanced,
boxrule=0pt,frame hidden,
borderline west={4pt}{0pt}{brown!75!black},
colback=white,
sharp corners
}




% Colors
\definecolor{blau_bauschule}{RGB}{22,65,148}
\setbeamercolor{frametitle}{fg=blau_bauschule}
\setbeamertemplate{navigation symbols}{} % Remove navigation symbols


\newtcolorbox{Definition_BS}[1]{
enhanced,boxrule=1pt,
colback=green!5!white,
colframe=green!75!black,fonttitle=\bfseries, title = #1,
%after title={\hfill\colorbox{black}{Definition}}
}



\newtcolorbox{Masseinheit}[1]{
enhanced,
boxrule=1pt,colframe=blue,
colback=white,
sharp corners, 
colframe=blue!75!black,
title = #1, 
after title={\hfill\colorbox{blue}{Masseinheit}}
}


\newtcolorbox{myLösung}{
  enhanced,
  boxrule=1pt,
  colframe=gray!75!black, % Definiert die Farbe des Rahmens als dunkelgrau
  colback=gray!20, % Definiert die Hintergrundfarbe der Box als hellgrau
  %sharp corners, % Macht die Ecken der Box scharf (nicht abgerundet)
  title = {Lösung}, % Fest eingestellter Titel der Box
  after title={}, % Fügt das Label "Masseinheit" nach dem Titel hinzu
  coltitle=white, % Farbe des Titeltexts
  fonttitle=\bfseries % Schriftart des Titels
}



% Set the title, author, and date
\title{\textbf{Lektionsprogramm HTg-26}}
\author{Patrick Pfändler}
\date{11.08.2025}


\begin{document}

\frame{\titlepage}


%\folieFragen



\begin{frame}{Inhalt der heutigen Lektion}
    \tableofcontents
\end{frame}


\section{Organisatorisches}
\BlueSectionSlide

\subsection{Uploads auf Teams}
\begin{frame}{Uploads auf Teams}
    \begin{itemize}
        %\item[\textbullet] Auswertung Prüfung
        %\item[\textbullet] Slides Beton Teil 1 (neue Version)
              %\item[\textbullet] Aktualisierter Terminplan
              %\item[\textbullet] Arbeitsauftrag Schmutziger Baustoff
        %\item[\textbullet] Mischungsberechnung: Aufgabenblatt und Folien (einfach und mittel)
        %\item[\textbullet] Lernziele Prüfung
        %\item[\textbullet] Zusammenfassung Prüfungsstoff
        %\item[\textbullet] Unterlagen zu den Vorträgen
        %\item[\textbullet] Aufgabe zu Verdichtungsmass
        %\item[\textbullet] 06 Kunststoffe: Folien
        %\item[\textbullet] Lernziele Prüfung 3: Beil 1 \& 2 ton Te
        %\item[\textbullet] Vortrag Yanis
        %\item[\textbullet] Keine neuen Uploads diese Woche
        \item[\textbullet] Folien zu Nachbesprechung des Quiz über Kunststoffe
    \end{itemize}

\end{frame}


\section{Rückblick auf die "Sommerferien"}
\BlueSectionSlide

\begin{frame}{Rückblick auf die "Sommerferien"}
    \begin{itemize}
        \item[\textbullet] Was habt ihr in den Sommerferien gemacht?
        \item[\textbullet] Wo gab es auf den Baustellen Probleme?
    \end{itemize}

\end{frame}

\begin{frame}{"Meine Ferien"}
    \begin{itemize}
        \item[\textbullet] Ich habe Job gewechselt und war daher nicht in den Ferien.
        \item[\textbullet] Moon and Stars in Locarno.
    \end{itemize}

\end{frame}


\section{Innovation im Bauwesen}
\BlueSectionSlide

\begin{frame}{Nutzung von KI}

    \begin{itemize}
        \item Hier ist ein Link zu einer interessanten Website: \href{https://schweizerischebau-my.sharepoint.com/:wb:/g/personal/pfaendler_bauschule_ch/EaA6FFoUtUVCvGoMXfZtrMoBPsF2c1g-e0zOir5MbWxZ_Q?e=VzYIgs}{Nutzung von KI}
    \end{itemize}
\vspace{1em}
$\Rightarrow$ Link funktioniert nur mit dem Bauschule-Account.

\end{frame}


\begin{frame}{Nutzung von KI -- Auftrag}
    \begin{itemize}
        \item[\textbullet] Wo setzt ihr in eurem Alltag KI ein?
        \item[\textbullet] Wo setzt ihr in eurem Job KI ein?
        \item[\textbullet] Wo hättet ihr noch gerne Unterstützung durch die KI in eurem Job?
    \end{itemize}
\vspace{1em}
Ca. 10 Minuten Brainstorming individuell.
\end{frame}


\begin{frame}{Diskussion im Plenum}

\end{frame}


\begin{frame}{Construix}

\href{https://www.baumeister.ch/construix-start/}{Construix}

\begin{itemize}
    \item Registrieren als Student (kostenlose Nutzung bis Ende 2025)
    \item Datum für die Kündigung im Kalender eintragen
\end{itemize}

\end{frame}


\begin{frame}{Construix -- Auftrag}
    \begin{itemize}
        \item[\textbullet] 2-er Gruppen
        \item[\textbullet] 10 Minuten Zeit
        \item[\textbullet] Prompten im Bot und Fehler oder besonders Interessante Antworten per Screenshot auf dem Whiteboard festhalten.
    \end{itemize}
\end{frame}

\begin{frame}{Diskussion im Plenum}

\end{frame}


\begin{frame}{Pause}
    \begin{itemize}
        \item[\textbullet] ca. 5 Minuten Pause
        \item[\textbullet] Danach geht es weiter mit den Vorträgen
    \end{itemize}

\end{frame}


\section{Vorträge}
\BlueSectionSlide

\begin{frame}{Vorträge}
    \begin{itemize}
        \item[\textbullet] Vorträge von den Schülern
        \item[\textbullet] 7 bis 15 Minuten pro Vortrag
        \item[\textbullet] 5 Minuten Fragen
        \item[\textbullet] Folien sind auf Teams hochgeladen
    \end{itemize}
\end{frame}

\subsection{Brugger Cedric}
\begin{frame}{Minergie Standards}
    von Cedric Brugger
\end{frame}



\section{Abschluss Postenlauf}
\BlueSectionSlide
\section{Postenlauf:Innovation im Bauwesen}
\begin{frame}{Postenlauf: Innovation im Bauwesen}
    ca. 90 min Länge, Bearbeitung in der Lektion (keine SLE resp. Hausaufgabe)
    \begin{itemize}
        \item [\textbullet] Wähle 3 der 5 Texte aus. Die Texte sind auf Teams in den Ordner \textit{Innovation im Bauwesen} hochgeladen. Arbeitsform: Einzelarbeit.
        \begin{itemize}
            \item [\textbullet] Erstelle zu einem Text ein Mindmap. 
            \item [\textbullet] Erstelle zu einem Text eine schriftliche Zusammenfassung von ca. 150 Wörtern.
            \item [\textbullet] Erstelle zu einem Text mindestens 5 Kontrollfragen (inkl. Lösung) für deinen Tischnachbarn.
        \end{itemize}
        \item [\textbullet] Die Dokumente müssen in Teams hochgeladen  werden in den jeweiligen Aufgaben und werden mit bestanden / nicht bestanden bewertet.
    \end{itemize}
    \end{frame}

\begin{frame}{Übergeordnete Lernziele}
    Die übergeordnete Lernziele für diese Unterrichtseinheit sind: 
    \begin{itemize}
        \item Bauführer und Bauführerinnen informieren sich über neue Methoden und Technologien und den Einsatz von
        multifunktionalen und intelligenten Baustoffen in ihrem Arbeitsbereich.
        \begin{itemize}
            \item Sie informieren sich aus Fachpresse und Messen über Innovationen. \textbf{(K2)}
            \item Sie betreiben ein firmeninternes Wissensmanagement zukunftsorientiert. \textbf{(K4)}
            \item Sie erarbeiten Dokumentationen zur Einführung von kreislauffähigen Materialien und Baumethoden
            in ihrem Bereich. \textbf{(K4)}
        \end{itemize}
        \item Bauführer und Bauführerinnen leiten den fachgerechten und vorschriftsmässigen Einsatz neuer Methoden, Technologien und Baustoffe.
        \begin{itemize}
            \item Sie wenden neue Methoden, Technologien und Baustoffe bei Bauarbeiten an. \textbf{(K3)}
            \item Sie instruieren die Mitarbeitenden in neuen Bauabläufen. \textbf{(K3)}
            \item Sie führen Evaluationen zum Einsatz von neuen Baustoffen durch. \textbf{(K4)}
        \end{itemize}
    \end{itemize}
\end{frame}


\begin{frame}{Abschluss Postenlauf}
    \begin{itemize}
        \item[\textbullet] Zeit: ca. 10 Minuten
        \item[\textbullet] Ziel: Abschluss oder Repetition der Themen des Postenlaufs
    \end{itemize}

\end{frame}


% \section{Padlet}
% \BlueSectionSlide
% \begin{frame}{Padlet}
%     \begin{itemize}
%         \item[\textbullet] Ziel wäre es, Fragen zu den Themen zu sammeln, diese können dann in den nächsten Lektionen beantwortet oder als Quiz zur Prüfungsvorbereitung verwendet werden.
%         \item[\textbullet] Link: \url{https://padlet.com/pfaendler/Fragen}
%         \item[\textbullet] Ergänzung der Organisation nach Thema (in den Titeln)
%     \end{itemize}
% \end{frame}


% \begin{frame}{Padlet ausfüllen}

%     \begin{itemize}
%         \item Zeit: ca. 5 Minuten
%     \end{itemize}

% \end{frame}

\section{Kunststoffe}
\BlueSectionSlide
\begin{frame}{Kunststoffe}
$\Rightarrow$ Nachbesprechung des Quiz über Kunststoffe
\end{frame}






\section{Organisatorisches}
\BlueSectionSlide



% \begin{frame}{Fragen zur Prüfung?}

% \end{frame}



\begin{frame}{Ausblick auf die nächste Lektion}
\begin{itemize}
    \item Wärmedämmstoffe 
\end{itemize}
\end{frame}

\naechstePruefung{08.09.2025}{Abdichtungen, Bindemittel gefestigt, Kunststoffe, WDS}


\end{document}