
\documentclass[
11pt,
captions=tableheading,
%smallheadings,
headings=big,
headsepline,
footsepline, 
%chapterprefix=false			% weiss nicht was passiert
captions=tableheading,
parskip=half-,
%BCOR=10mm,
%twocolumn, 
%draft
]{scrartcl}

%\usepackage[babelshorthands]{polyglossia}
\usepackage{polyglossia}

\setdefaultlanguage[variant = swiss]{german}
%\usepackage[ngerman]{babel} 
\usepackage[]{ scrlayer-scrpage }
\usepackage[ a4paper,
 total={165mm,250mm},
 left=25mm,
 top=25mm,
 headsep=10mm
 %footsep=12mm
 %,showframe
  ]{geometry}
 \usepackage{fontspec}
\setmainfont{Arial} % Bauschule CI Manual
\setsansfont{Arial} % Bauschule CI Manual


\usepackage[dvipsnames]{xcolor}
\usepackage[most]{tcolorbox}
\usepackage[
version=3,
arrows=pgf-filled,
]{mhchem} % für chemische Formeln
%\usepackage{microtype}
\usepackage{float}
\usepackage{enumitem}
\usepackage{multicol}
\usepackage{booktabs}
\usepackage{pgfplots}
\pgfplotsset{compat=newest}

\usepackage{float}
\usepackage{booktabs}
\usepackage{xspace}
\usepackage{luacode}
%\pgfplotset{compact= 1.17}
%\usepackage{structuralanalysis}

\usepackage{siunitx}
\usepackage{smartdiagram}
\usepackage{amsfonts}
\usepackage{amssymb,amsmath}

 \usepackage{titletoc}

% order of hyperref, cleverref is important
\usepackage[hidelinks]{hyperref}
\usepackage{cleveref}


\usepackage{graphicx}

% Colors
\definecolor{blau_bauschule}{RGB}{22,65,148}


% Titel mit Bauschule blau gemäss CI manual
\addtokomafont{section}{\color{blau_bauschule}\Huge}
\addtokomafont{subsection}{\color{blau_bauschule}\huge}
\addtokomafont{subsubsection}{\color{blau_bauschule}\Large}
\addtokomafont{paragraph}{\normalsize}
\addtokomafont{subparagraph}{\small}
% Pagestyle
\pagestyle{scrheadings}
\ihead{\fontsize{9pt}{2pt}\selectfont \klasse}
\ohead{\fontsize{9pt}{2pt}\selectfont \fach}
\chead{\fontsize{9pt}{2pt}\selectfont \headmark}
\ifoot{\fontsize{9pt}{2pt}\selectfont Bauschule Aarau} 
\ofoot{\fontsize{9pt}{2pt}\selectfont \thepage} %Seitennummer
\cfoot{\fontsize{9pt}{2pt}\selectfont }
\setkomafont{pagehead}{\normalfont}
\setkomafont{pagefoot}{\normalfont}
\setkomafont{pagefoot}{\normalfont}
\setkomafont{pagehead}{\normalfont}
\setkomafont{pagefoot}{\normalfont}


% Bild- und Tabellenunterschriften
\renewcommand*{\figurename}{Abbildung}
\renewcommand*{\tablename}{Tabelle}

\newcommand{\fach}{Baustoffe\xspace}
\newcommand{\klasse}{HTg-26\xspace}


% Define a shortcut for red text
\newcommand{\red}[1]{\textcolor{red}{#1}}


% Titel
\title{\fach}
%\author{Patrick Pfändler}
\date{2024}



\begin{document}

\section*{Programm Baustoffe}

Klasse: \klasse

Zeit: 15:15 bis 17:00 Uhr (105 Minuten) im Raum 214

Schüler: 14



Stand: \today

\vspace{1cm}



\begin{table}[H]
    \centering
    \begin{tabular}{llp{8cm}p{4cm}}
        \toprule
        \textbf{Datum} & \textbf{KW} & \textbf{Inhalt}                                          & \textbf{Bemerkung}  \\
        \midrule
        22.10.2024     & 43          & Einführung                                               & {}                  \\
        28.10.2024     & 44          & Physik                                                   & Wärmeausdehnung     \\
        04.11.2024     & 45          & Physik                                                   & {}                  \\
        11.11.2024     & 46          & Physik                                                   & {}                  \\
        18.11.2024     & 47          & Physik und Chemie                                        & Chemie F 9          \\
        25.11.2024     & 48          & Chemie                                                   & Chemie fertig       \\
        02.12.2024     & 49          & Bindemittel                                              & {}                  \\
        09.12.2024     & 50          & Bindemittel und Mörtel                                   & Musterprüfung lösen \\
        16.12.2024     & 51          & Mörtel und \red{Prüfung: Physik, Chemie und Bindemittel} & Nur Prüfung         \\
        \midrule
        23.12.2024     & 52          & \textcolor{blue}{Ferien}                                                       \\
        \midrule
        06.01.2025     & 2           & Nachbesprechung Prüfung, Bindemittel                     & {}                  \\
        13.01.2025     & 3           & Mörtel                                                   & {}                  \\
        20.01.2025     & 4           & Mörtel                                                    & {}                  \\
        \midrule
        27.01.2025     & 5           & \textcolor{blue}{Ferien}                                 & {}                  \\
        \midrule
        03.02.2025     & 6           & Beton                                                    & Besuch N. Raeber (Didaktikkurs)                  \\
        10.02.2025     & 7           & Beton                                                    & {}                  \\
        17.02.2025     & 8           & Beton                                                    & {}                  \\
        24.02.2025     & 9           & \red{Prüfung: Mörtel, Beton}                             &                     \\
        \bottomrule
    \end{tabular}
\end{table}



\end{document}