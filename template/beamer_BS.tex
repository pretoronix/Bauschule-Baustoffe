% !TEX program = lualatex
\ifdefined\customoptions
    \documentclass[\customoptions]{beamer}
\else
    \documentclass[aspectratio=169,handout,12pt]{beamer}
\fi


\PassOptionsToPackage{dvipsnames,svgnames}{xcolor}

%\usepackage[utf8]{inputenc}
%\usepackage[ngerman]{babel} % Schweizer Rechtschreibung

\usepackage{polyglossia}
\setdefaultlanguage[variant = swiss]{german}
\usepackage{fontspec}
\setmainfont{Times New Roman}
\setsansfont{Arial}


\usepackage{amsmath}
\usepackage{siunitx}
\sisetup{
    locale = DE,
    inter-unit-product = \ensuremath{{\cdot}},
    detect-mode,             % Use the surrounding text font mode
    detect-family,           % Use the surrounding text font family
    detect-weight,           % Use the surrounding text font weight
    mode = text,             % Ensure that numbers and units are typeset in text mode
    %negative-powers = false, % Avoid using negative powers
    per-mode = symbol        % Use the division symbol for per units
}

\usepackage{tikz}
\usepackage{enumitem}
\usepackage{graphicx}
\usepackage{booktabs}
\usepackage{calc}
\usepackage{multicol}
\usepackage{amsmath}
\usepackage{fontawesome}
\usepackage{colortbl}


\usepackage{tcolorbox}
\tcbuselibrary{skins}

\usepackage[dvipsnames]{xcolor}

%\usepackage[scaled]{helvet} % Arial-ähnliche Schriftart (Helvetica)
%\renewcommand\familydefault{\sfdefault} % Setzt die Standard-Schriftart auf sans-serif


\usetheme{Madrid} % This theme is visually appealing
%\usecolortheme{whale} % A color theme with blue tones
\usecolortheme{dolphin} % A color theme with blue tones
\setbeamerfont{frametitle}{size=\Large}
\setbeamercolor{frametitle}{fg=blau_bauschule}
\setbeamercolor{toc}{fg=blau_bauschule}
\setbeamercolor{section in toc}{fg=blau_bauschule}
\setbeamercolor{subsection in toc}{fg=blau_bauschule}

\newcommand{\mylogo}{
    \begin{tikzpicture}[remember picture,overlay]
        \node[anchor=north east, yshift=-2mm, fill=white, inner sep=2pt] at (current page.north east) % Verschiebe das Logo um 5mm nach unten
        {\includegraphics[height=0.4cm]{/Users/patricpf/Documents/repos/Bauschule-Baustoffe/template/bauschule-logo-5cm.png}}; % Grösse nach Bedarf anpassen
    \end{tikzpicture}
}


\newcommand{\myNmm}[1]
{
    \sisetup{per-mode=symbol}
    \SI{#1}{\newton\per\mm\squared}
}

\logo{\mylogo}

\setbeamertemplate{headline}{%
    \begin{beamercolorbox}[wd=\textwidth,ht=0.5ex,dp=1ex]{upper separation line head}
    \end{beamercolorbox}
}

\setbeamercolor{upper separation line head}{bg=blau_bauschule}


% Anpassen des Frametitels, um ihn fett zu machen
\setbeamertemplate{frametitle}{%
    \nointerlineskip%
    \begin{beamercolorbox}[sep=0.3cm,left,wd=\paperwidth]{frametitle}%
        \usebeamerfont{frametitle}\bfseries\insertframetitle%
    \end{beamercolorbox}%
}

\setbeamertemplate{section in toc}[square]
\setbeamertemplate{subsection in toc}[square]

\newcommand{\BlueSectionSlide}{
    {
            \setbeamercolor{background canvas}{bg=blau_bauschule} % Temporarily set the background color
            \begin{frame}[plain] % Plain frame without navigation
                \begin{center}
                    {\Huge \color{white} \textbf{\insertsection}} % Section title in the center
                \end{center}
            \end{frame}
            \setbeamercolor{background canvas}{bg=} % Reset background color
        }
}

%\usebackgroundtemplate{
%    \includegraphics[width=\paperwidth,height=\paperheight]{my_pdf_copy_of_empty_ppt_template}
%}

% Setzen Sie hier den Namen des Fachs
\newcommand{\fachname}{Baustoffe}
\newcommand{\FinRes}[1]{\underline{\underline{#1}}}

\setbeamertemplate{footline}{%
    \begin{tikzpicture}[remember picture,overlay]
        % Dünner Farbverlauf
        \shade[left color=blau_bauschule, right color=blau_bauschule]
        ([yshift=0cm]current page.south west) rectangle ([yshift=0.5cm]current page.south east);
        % Autor linksbündig
        \node[anchor=south west, xshift=0.5cm, yshift=0.15cm] at (current page.south west) {%
            \textcolor{white}{\insertshortauthor}};
        % Titel zentriert
        \node[anchor=south, yshift=0.15cm] at (current page.south) {%
            \textcolor{white}{\fachname}};
        % Foliennummer rechtsbündig
        \node[anchor=south east, xshift=-0.5cm, yshift=0.15cm] at (current page.south east) {%
            \textcolor{white}{\insertframenumber{} / \inserttotalframenumber}};
    \end{tikzpicture}
}

%\setbeamertemplate{footline}{%
%  \hspace*{0.5cm}\textbf{\fachname}\hfill \insertframenumber / \inserttotalframenumber\hspace*{0.5cm}}

\setbeamerfont{block title}{size=\normalsize,series=\bfseries,family=\sffamily}
%\setbeamerfont{frametitle}{size=\LARGE,series=\bfseries,family=\sffamily}


\newtcolorbox{Merke}{
    enhanced,
    boxrule=0pt,frame hidden,
    borderline west={4pt}{0pt}{red!75!black},
    colback=white,
    sharp corners,
    before upper={\textbf{Merke:}\quad},
}

\newtcolorbox{Anwendungen}{
    enhanced,
    boxrule=0pt,frame hidden,
    borderline west={4pt}{0pt}{brown!75!black},
    colback=white,
    sharp corners
}




% Colors
\definecolor{blau_bauschule}{RGB}{22,65,148}
\setbeamercolor{frametitle}{fg=blau_bauschule}
\setbeamertemplate{navigation symbols}{} % Remove navigation symbols




\newtcolorbox{Definition_BS}[1]{
    enhanced,boxrule=1pt,
    colback=green!5!white,
    colframe=green!75!black,fonttitle=\bfseries, title = #1,
    %after title={\hfill\colorbox{black}{Definition}}
}



\newtcolorbox{Masseinheit}[1]{
    enhanced,
    boxrule=1pt,colframe=blue,
    colback=white,
    sharp corners,
    colframe=blue!75!black,
    title = #1,
    after title={\hfill\colorbox{blue}{Masseinheit}}
}


\newtcolorbox{myLösung}{
    enhanced,
    boxrule=1pt,
    colframe=gray!75!black, % Definiert die Farbe des Rahmens als dunkelgrau
    colback=gray!20, % Definiert die Hintergrundfarbe der Box als hellgrau
    %sharp corners, % Macht die Ecken der Box scharf (nicht abgerundet)
    title = {Lösung}, % Fest eingestellter Titel der Box
    after title={}, % Fügt das Label "Masseinheit" nach dem Titel hinzu
    coltitle=white, % Farbe des Titeltexts
    fonttitle=\bfseries % Schriftart des Titels
}


\newtcolorbox{myLernziele}{
    enhanced,
    colframe=cyan!75!black, % Rahmenfarbe (Türkis)
    colback=cyan!10, % Hintergrundfarbe (Helles Türkis)
    coltitle=black, % Titeltextfarbe (Schwarz)
    boxrule=1pt, % Rahmenstärke
    title={Lernziele}, % Fester Titel
    fonttitle=\bfseries, % Schriftart des Titels (Fett)
    left=2mm, % Innenabstand links
    right=2mm, % Innenabstand rechts
    top=1mm, % Innenabstand oben
    bottom=1mm % Innenabstand unten
}


\newtcolorbox{Fragenblock}{
    enhanced,
    colframe=orange!75!black, % Rahmenfarbe (Orange)
    colback=orange!10, % Hintergrundfarbe (Helles Orange)
    coltitle=black, % Titeltextfarbe
    boxrule=1pt, % Rahmenstärke
    title={Frage}, % Fester Titel
    fonttitle=\bfseries, % Schriftart des Titels (Fett)
    left=2mm, % Innenabstand links
    right=2mm, % Innenabstand rechts
    top=1mm, % Innenabstand oben
    bottom=1mm % Innenabstand unten
}

\newcommand{\folieFragen}{%
    \subsection{Fragen zur letzten Lektion}
    \begin{frame}{Fragen zur letzten Lektion}
        \begin{itemize}
            \item Haben Sie Fragen zur letzten Lektion?
        \end{itemize}

    \end{frame}
}

\newcommand{\naechstePruefung}[2]{%
    \subsection{Nächste Prüfung}
    \begin{frame}{Nächste Prüfung}
        \begin{itemize}
            \item[\textbullet] {\textbf{#1:}\hspace{5pt}}\textcolor{red}{\textbf{Prüfung:} #2}
        \end{itemize}
    \end{frame}
}

\newcommand{\pruefung}[3]{%
    \section{Prüfung}
    \BlueSectionSlide
    \begin{frame}{\textcolor{red}{Prüfung: #1}}
        \begin{itemize}
            \item[\textbullet] Bitte Namen eintragen:  \textit{Nachname Vorname}
            \item[\textbullet] Zeit: #2 Minuten
            \item[\textbullet] \href{#3}{\textcolor{blue}{Link zur Prüfung}}
        \end{itemize}
    \end{frame}
}

\newcommand{\Lernziele}[1]{%
    \begin{frame}{Lernziele}
        \begin{itemize}
            #1
        \end{itemize}
    \end{frame}
}


\newcommand{\notenverteilung}[5]{
    \begin{frame}{Notenverteilung}

        \begin{table}[h!]
            \centering
            \begin{tabular}{ll}
                \toprule
                \textbf{{}}              & \textbf{Note} \\ \midrule
                Minimum                  & #1            \\
                Median                   & #2            \\
                Mittelwert               & #3            \\
                Maximum                  & #4            \\
                Standardabweichung (Std) & #5            \\ \bottomrule
            \end{tabular}
        \end{table}
    \end{frame}
}

\newcommand{\notenformel}[3]{%
    \begin{frame}{Formel für die Notenberechnung}
        \begin{itemize}
            \item[\textbullet] Note 1 mit #1 Punkten
            \item[\textbullet] Note 6 mit #2 Punkten von maximal #3 Punkten
        \end{itemize}
    \end{frame}
}