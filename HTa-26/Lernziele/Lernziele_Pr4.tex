% !TEX root = /Users/patricpf/Documents/repos/Bauschule-Baustoffe/Hta-26/Lernziele/Lernziele_Pr4.tex
% !TEX program = lualatex
% !TEX program = lualatex
\documentclass[
    %answers,
    a4paper,ngerman,12pt, addpoints]{exam}

\usepackage[utf8]{inputenc}
%\usepackage[T1]{fontenc}
%\usepackage[ngerman]{babel}


\usepackage{polyglossia}
\setdefaultlanguage[variant = swiss]{german}
\usepackage{fontspec}
\setmainfont{Aptos} % Bauschule CI Manual
\setsansfont{Aptos} % Bauschule CI Manual


\usepackage[ a4paper,
 total={165mm,250mm},
 left=25mm,
 top=25mm,
 headsep=10mm
 %footsep=12mm
 %,showframe
  ]{geometry}

\usepackage{graphicx}
\usepackage{siunitx}
\usepackage{booktabs} % schöne Tabellen
\usepackage{float}
\floatplacement{figure}{H}
\usepackage{xcolor}
\usepackage{pdfpages}
\usepackage{enumitem}
\usepackage{mdframed} % Boxen
\usepackage{amsmath,amssymb}
\usepackage{tcolorbox}
\usepackage{lastpage} % For the total number of pages
\usepackage{gensymb}
\usepackage{xspace}
\usepackage{tabularx}
\usepackage{multicol}
\usepackage[
    version=3,
    arrows=pgf-filled,
]{mhchem} % für chemische Formeln
%\usepackage{microtype}
\usepackage{subfigure}
\usepackage[hidelinks]{hyperref}
\usepackage{cleveref}
\usepackage{luacode}
\usepackage{amsmath}
\usepackage{textcomp}


\sisetup{
  locale = DE,
  inter-unit-product = \ensuremath{{\cdot}},
  detect-all,
}

% Colors
\definecolor{blau_bauschule}{RGB}{22,65,148}
\CorrectChoiceEmphasis{\color{blau_bauschule}}
\SolutionEmphasis{\color{blau_bauschule}}

\setlength{\parindent}{0em} % Verhindert einrücken
\setlength\linefillheight{0.3in}


%% COMMMANDS
\author{Patrick Pfändler}
\newcommand{\dozent}{Patrick Pfändler}
\newcommand{\fach}{Baustoffe}


\newcommand{\punkte}[1]{%
    \begin{infobox}%
        #1
    \end{infobox}}%
\newcommand{\FinRes}[1]{\underline{\underline{#1}}}

\newmdenv[linecolor=black,backgroundcolor=gray!15,frametitle={Punktverteilung},leftmargin=1cm,rightmargin=1cm]{infobox}

\newcommand{\pagebreaksol}{
    \ifprintanswers
        \clearpage
    \else
        {}
    \fi
}

\newcommand{\pagebreakexam}{
    \ifprintanswers
        {}
    \else
        \clearpage
    \fi
}

\SolutionEmphasis{\color{blau_bauschule}}
\makeatletter%
\newcommand{\solutiontable}[1]{\ifprintanswers\begingroup\Solution@Emphasis#1\if@shadedsolutions%
            {\cellcolor{SolutionColor}}%
        \else%
        \fi\endgroup\else\phantom{#1}\fi}%
\makeatother%

\newcommand{\myNmm}[1]
{
    \sisetup{per-mode=symbol}
    \SI{#1}{\newton\per\mm\squared}
}

\renewcommand{\thequestion}{\fontsize{12pt}{2pt} \selectfont  \bfseries \arabic{question}}
\sisetup{per-mode=symbol}



%% Translation

\pointpoints{Punkt}{Punkte}
\bonuspointpoints{Bonuspunkt}{Bonuspunkte}
\renewcommand{\solutiontitle}{\noindent\textbf{Lösung:}\enspace}
\chqword{Frage}
\chpgword{Seite}
\chpword{Punkte}
\chbpword{Bonus Punkte}
\chsword{Erreicht}
\chtword{Gesamt}
\hpword{Punkte:}
\hsword{Ergebnis:}
\hqword{Aufgabe:}
\htword{Summe:}


\renewcommand{\questionshook}{%
  %\setlength{\leftmargin}{0pt}% removes the indentation from the left
  \setlength{\labelwidth}{1.25cm}% adjusts label width
  \setlength{\itemindent}{0cm}% aligns the start of the item with the above
  \setlength{\labelsep}{0.25cm}% space between the label and the item text
}




%% header and footer
\pagestyle{headandfoot}
\firstpageheadrule
\runningheadrule

% Adjust the font size for the header
\firstpageheader{\fontsize{9}{11}\selectfont\fach}{}{\fontsize{9}{11}\selectfont\dozent \\ \blattname}
\runningheader{\fontsize{9}{11}\selectfont\fach}{}{\fontsize{9}{11}\selectfont\dozent \\ \blattname}

% Adjust the font size for the footer
\firstpagefooter{\includegraphics[width=2.5cm]{bauschule-logo-5cm.png}}{}{\fontsize{9}{11}\selectfont\thepage\,/\,\pageref{LastPage}}
\runningfooter{\includegraphics[width=2.5cm]{bauschule-logo-5cm.png}}{}{\fontsize{9}{11}\selectfont\thepage\,/\,\pageref{LastPage}}



\newcommand{\blattname}{Prov. Lernziele: Prüfung 4: Metalle, Wärmedämmstoffe, Innovation im Bauwesen}
\newcommand{\Added}[1]{\textcolor{blue}{#1}}


%\printanswers
\begin{document}
\section*{\blattname}
\subsection*{Tipps zur Prüfungsvorbereitung}
Hyperlinks funktionieren nicht in Teams nicht immer zuverlässig. $\rightarrow$ Dokument herunterladen wird empfohlen.
\begin{itemize}
	\item Repetition der Quiz zu den einzelnen Lektionen. (Mit einem anderen Name können Sie die Quiz auf Classtime nochmals durchführen.)
	\item Aufgaben zum Zugversuch nochmals durchgehen
% 	\begin{itemize}[]

% 		\item \href{https://forms.office.com/Pages/ResponsePage.aspx?id=HsbbSHAOrE6HJuK4duaJwdUERlKbBqZKkaxTc87ge2NUNkdGRTVJS0NaVkRRMTBMMkoxNFpCUVhIRy4u}{Dämmstoffe Teil 2 (Forms)}
% 		\item \href{https://forms.office.com/Pages/ResponsePage.aspx?id=HsbbSHAOrE6HJuK4duaJwdUERlKbBqZKkaxTc87ge2NUOVc0UVJWTlREMTFJRlI4TE1YMzI2WEYxTC4u}{Metalle Teil 1 (Forms)}
% 		\item \href{https://forms.office.com/Pages/ResponsePage.aspx?id=HsbbSHAOrE6HJuK4duaJwdUERlKbBqZKkaxTc87ge2NURjBRS1VLNUxSM0M2NUxLNThGT0ZFMzNOTy4u}{Metalle Teil 2 (Forms)}
% 	\end{itemize}
% 	\item Folien und/oder Skript nochmals durchgehen
\end{itemize}

\subsection*{Informationen zur Prüfung}

\begin{description}[leftmargin=!,labelwidth=\widthof{Hinweise zur Bearbeitung...},font=\normalfont]
%\item [Prüfungsmodus] Sämtliche Unterlagen (gedruckt und/oder digital), Internet, keine Kommunikation zu anderen Personen (wie an der Fachabschlussprüfung)
\item [Prüfungsmodus] Online-Prüfung, Open-Book-Prüfung mit Classtime
\item  [Prüfungsdauer] ca. 40 min
\item [Empfohlene Hilfsmittel] Taschenrechner, gedruckte Zusammenfassung
\item [Anzahl Punkte] Die Maximalpunktzahl der Prüfung sind ca. 40.  Fürs Zeitmanagement, es sollte ca. 1 Punkt pro Minute erreicht werden. Für die Maximalnote werden i.d.R. nicht sämtliche Punkte benötigt.
\item [Bewertung] Die Prüfung wird halb-automatisch ausgewertet. Die Prüfungen werden nach der Prüfung korrigiert und die Resultate werden in Teams-Chat als PDF versendet. Die Note der Prüfung erhalten Sie ebenfalls über Teams zugestellt. \\ Die Notenskala berücksichtigt, dass Multiple-Choice-Fragen vorhanden sind.
\item [Hinweise zur Bearbeitung] Geben Sie zumindest beim Schlussresultat eine resp. die verlangte Einheit an. Ohne Angabe von Einheiten kann i.d.R. nicht maximale Punktzahl der Aufgabe erreicht werden. \\ Bei Multiple-Choice-Aufgaben führen falsche Kreuze nicht zu Punktabzug. Bei grob falschen Antworten kann ein Punktabzug erfolgen. \\ Die Bearbeitungszeit der Prüfung ist i.d.R. äusserst knapp bemessen! $\Rightarrow$ Lösen Sie zuerst, was Sie direkt wissen und kommen Sie später auf die schwierigeren Fragen zurück.\\
Lesen Sie die Frage und die Hinweise der Aufgabenstellung genau.
\end{description}


\pagebreak
\subsection*{Lernziele}
Diese Lernziele geben einen groben Überblick über den Stoffumfang der vierten Prüfung im Fach Baustoffe.



%\subsubsection*{Mit Bindemitteln gefestigte Baustoffe}
Die Studierenden kennen: 

\begin{itemize}[noitemsep]
	\item den Begriff: Mit Bindemitteln gefestigte Baustoffe.
	\item das allgemeine Fabrikationsschema von mit Bindemitteln gefestigte Baustoffen und können dieses auf verschiedene mit Bindemitteln gefestigte Baustoffen übertragen (z.\,B. Herstellung von Kalksandstein).
	\item die Rohmaterialien, die Eigenschaften, Verwendungszwecke, Fabrikationsprozesse, mögliche Nachbearbeitungsschritte, und spezielle Eigenschaften der folgenden mit Bindemitteln gefestigte Baustoffe
	\begin{itemize}[noitemsep]
  \item Kunststeine
  \item Betonelemente
  \item Betonwaren
  \item Zementstein
  \item Splittbetonsteine, -platten
  \item Leichtbetonsteine, -platten
  \item Porenbeton
  \item Zementgebundene Holzspanplatten
  \item Leichtbauplatten
  \item Glasfaserbeton / Polymerbeton
  \item  Leichtbeton Bauplatten
  \item Faserzement
  \item Kalksandsteine
  \item Gipsbauplatten,  Gipskartonplatten, 
  \item Gipsgebundene Holzfaserplatten, gipsgebundene Holzspanplatten
\end{itemize}
	\item die Entsorgungsmöglichkeiten für mit Bindemitteln gefestigte Baustoffe und ökologische und gesundheitliche Aspekte beim Umgang mit diesen Produkten.
\end{itemize}
%\subsubsection*{Kunststoffe}
Die Studierenden kennen: 

\begin{itemize}[noitemsep]
	\item die Unterschiede und Eigenschaften, Aufbau, Verarbeitungsverfahren, Vor- und Nachteile von Kunststoffen (inkl. Naturkautschuk) zu anderen Baustoffen. 
	\item die Einteilung der Baustoffe (insbesondere Kunststoffe).
	\item die häufigsten Elemente, welche bei Kunststoffen vorkommen.
	\item den Aufbau der Kunststoffe.
	\item Möglichkeiten zur Beeinflussung der Eigenschaften von Kunststoffen (Hilfsstoffe, Zusatzstoffe, etc.)
	\item Kunststoffen nach ihren thermisch-mechanisch, nach ihrem Herstellungsprozess oder Verwendungsmöglichkeiten im Bauwesen unterteilen.
	\item unterschiedliche Kunststoffgruppen.
	\item Beispiele zu den unterschiedlichen Kunststoffen.
	\item mögliche Gefahren von Kunststoffen für die Umwelt.
	\item Möglichkeiten für das Recycling und Entsorgung von Kunststoffen.
	\item Gefahren von Halogenen und Produkte, welche beim Verbrennen Halogene ausstossen können.
	%\item wichtige Normen und Empfehlungen zu den Kunststoffen.
\end{itemize}
%\subsubsection*{Abdichtungsmaterialien und Klebstoffe}

Die Studierenden kennen: 

\begin{itemize}[noitemsep]
	\item den Zweck von Abdichtungen und unterschiedliche Abdichtungskonzepte (z.B. für Flächen oder Fugen).
	\item die Begriffe Hydrophobierung, Imprägnierung und Beschichtung.
	\item erdberührte Schutzsysteme (wie z.B. Schutzanstrich, Schutzbeschichtungen und Abdichtungen), deren Anwendungszweck und Anwendungsbereiche und die Wirkungsweise.
	\item bewitterte Schutzsysteme (wie z.B. Imprägnierungsmittel, Beschichtungen, Gussasphalt), deren Anwendungszweck und Anwendungsbereiche und die Wirkungsweise.
	\item unterschiedliche Gewässerschutzsysteme, deren Anwendungszweck und Anwendungsbereiche und die Wirkungsweise.
	\item unterschiedliche Baupapiere und Folien, deren Anwendungszweck und mögliche Anwendungsbereiche sowie die Wirkungsweise.
	\item die Inhalte der Norm SIA 270 (Anwendungsgruppen, Dichtungsklassen, technische Abkürzungen, ...)
	\item einige Fabrikationsprozesse für die Herstellung von Kunststoffdichtungsbahnen.
	\item unterschiedliche Verbindungstechniken für Kunststoffdichtungsbahnen.
	\item Anwendungsbeispiele zu Kunststoffdichtungsbahnen (u.a. SIA 281).
	\item die Begriffe Bitumenbahn, Bitumen-Dichtungsbahn, Polymerbitumen-Dichtungsbahn, AC-Be\-ständig\-keit, Elastomerbitumen, Oxidationsbitumen, Plastomerbitumen, Träger\-einlage und Ob\-erflächenausrüstung.
	\item das Bezeichnungsschema von Bitumendichtungsbahnen und können dieses Anwenden.
	\item Flüssigkunststoff-Abdichtungen, sowie die Anwendungsmöglichkeiten, Eigenschaften und zugehörige Begriffe (Haftvermittler, Solldicke, Nutzschicht).
	\item das Bezeichnungsschema von Gussasphalt und können dieses anwenden.
	\item unterschiedliche Typen von Fugen und mögliche Abdichtungssystem zu den unterschiedlichen Fugentypen.
	\item unterschiedliche Arten von Klebstoffen und mögliche Anwendungsbereiche.
\end{itemize}


\subsubsection*{Dämmstoffe}

Die Studierenden kennen: 

\begin{itemize}[noitemsep]
	\item die Einteilung der Wärmedämmstoffe.
	\item die U-Wert-Berechnung.
	\item Zusammensetzung, Fabrikation, Produktnamen, Eigenschaften und wichtige technische Daten von den wichtigsten Wärmedämmstoffen (vgl. Skript).
	\item ökologische Kriterien bei der Selektion von Wärmedämmstoffen.
	\item die Funktion der Dampfbremse bzw. Dampfsperre und dessen Position im Dach- oder Fassadenaufbau.
	\item  unterschiedliche Arten von Schall.
	\item brennbare und nicht-brennbare Wärmedämmstoffe.
	\item Zweck und Beispiele zu Schwingungsdämpfern.
	\item  die wichtigsten Normen und Empfehlungen von Dämmstoffen im Bauwesen.

\end{itemize}
\input{/Users/patricpf/Documents/repos/Bauschule-Baustoffe/Unterlagen/50_Lernziele/09_Metalle.tex}
\subsubsection*{Innovation im Bauwesen}
Die Lernziele für die Lektionen zum Thema Innovation im Bauwesen:
\begin{itemize}[noitemsep]
	\item Sie kennen mindestens drei innovative Materialien, Baumethoden oder Kombinationen davon.
	\item Sie können basierend auf Projekten den Einsatz von innovativen Methoden vorschlagen und dabei Vor- und Nachteile im gegenwärtigen Projekt aufzeigen.
	\item Sie können innovative Baumethoden für das interne Wissensmanagement dokumentieren und strukturieren.
	\item Sie können relevante Informationen aus Fachartikeln extrahieren und bewerten.
\end{itemize}






\end{document}
