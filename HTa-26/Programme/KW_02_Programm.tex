% !TEX TS-program = lualatex

% !TEX root = /Users/patricpf/Documents/repos/Bauschule-Baustoffe/HTa-26/Programme/KW_02_Programm.tex
% !TEX TS-program = lualatex

% !TEX program = lualatex
\documentclass[aspectratio=169, 
handout,
]{beamer}


\PassOptionsToPackage{dvipsnames,svgnames}{xcolor}

%\usepackage[utf8]{inputenc}
%\usepackage[ngerman]{babel} % Schweizer Rechtschreibung

\usepackage{polyglossia}
\setdefaultlanguage[variant = swiss]{german}
\usepackage{fontspec}
\setmainfont{Times New Roman}
\setsansfont{Arial}


\usepackage{amsmath}
\usepackage{siunitx}
\sisetup{
  locale = DE,
  inter-unit-product = \ensuremath{{\cdot}},
  detect-mode,             % Use the surrounding text font mode
  detect-family,           % Use the surrounding text font family
  detect-weight,           % Use the surrounding text font weight
  mode = text,             % Ensure that numbers and units are typeset in text mode
  %negative-powers = false, % Avoid using negative powers
  per-mode = symbol        % Use the division symbol for per units
}

\usepackage{tikz}
\usepackage{enumitem}
\usepackage{graphicx}
\usepackage{booktabs}
\usepackage{calc}
\usepackage{multicol}
\usepackage{amsmath}

\usepackage{tcolorbox}
\tcbuselibrary{skins}

\usepackage[dvipsnames]{xcolor}

%\usepackage[scaled]{helvet} % Arial-ähnliche Schriftart (Helvetica)
%\renewcommand\familydefault{\sfdefault} % Setzt die Standard-Schriftart auf sans-serif


\usetheme{Madrid} % This theme is visually appealing
%\usecolortheme{whale} % A color theme with blue tones
\usecolortheme{dolphin} % A color theme with blue tones


\newcommand{\mylogo}{
    \begin{tikzpicture}[remember picture,overlay]
        \node[anchor=north east, yshift=-2mm, fill=white, inner sep=2pt] at (current page.north east) % Verschiebe das Logo um 5mm nach unten
            {\includegraphics[height=0.4cm]{/Users/patricpf/Documents/repos/Bauschule-Baustoffe/template/bauschule-logo-5cm.png}}; % Grösse nach Bedarf anpassen
    \end{tikzpicture}
}

\logo{\mylogo}

\setbeamertemplate{headline}{%
  \begin{beamercolorbox}[wd=\textwidth,ht=0.5ex,dp=1ex]{upper separation line head}
  \end{beamercolorbox}
}

\setbeamercolor{upper separation line head}{bg=blau_bauschule}


% Anpassen des Frametitels, um ihn fett zu machen
\setbeamertemplate{frametitle}{%
    \nointerlineskip%
    \begin{beamercolorbox}[sep=0.3cm,left,wd=\paperwidth]{frametitle}%
        \usebeamerfont{frametitle}\bfseries\insertframetitle%
    \end{beamercolorbox}%
}

%\usebackgroundtemplate{
%    \includegraphics[width=\paperwidth,height=\paperheight]{my_pdf_copy_of_empty_ppt_template}
%}

% Setzen Sie hier den Namen des Fachs
\newcommand{\fachname}{Baustoffe}
\newcommand{\FinRes}[1]{\underline{\underline{#1}}}

% Anpassen der Fusszeile für Abstände zum Rand
\setbeamertemplate{footline}
{
  \leavevmode%
  \hbox{%
  \begin{beamercolorbox}[wd=.33\paperwidth,ht=2.25ex,dp=1ex,left,leftskip=1em]{author in head/foot}%
    \usebeamerfont{author in head/foot}\insertshortauthor
  \end{beamercolorbox}%
  \begin{beamercolorbox}[wd=.34\paperwidth,ht=2.25ex,dp=1ex,center]{title in head/foot}%
    \usebeamerfont{title in head/foot}\fachname % Hier wird der Fachname anstelle des Titels angezeigt
  \end{beamercolorbox}%
  \begin{beamercolorbox}[wd=.33\paperwidth,ht=2.25ex,dp=1ex,right,rightskip=1em]{date in head/foot}%
    \usebeamerfont{date in head/foot}\insertframenumber{} / \inserttotalframenumber\hspace*{2ex}
  \end{beamercolorbox}}%
  \vskip0pt%
}


\newtcolorbox{Merke}{
enhanced,
boxrule=0pt,frame hidden,
borderline west={4pt}{0pt}{red!75!black},
colback=white,
sharp corners,
before upper={\textbf{Merke:}\quad},
}

\newtcolorbox{Anwendungen}{
enhanced,
boxrule=0pt,frame hidden,
borderline west={4pt}{0pt}{brown!75!black},
colback=white,
sharp corners
}




% Colors
\definecolor{blau_bauschule}{RGB}{22,65,148}
\setbeamercolor{frametitle}{fg=blau_bauschule}
\setbeamertemplate{navigation symbols}{} % Remove navigation symbols


\newtcolorbox{Definition_BS}[1]{
enhanced,boxrule=1pt,
colback=green!5!white,
colframe=green!75!black,fonttitle=\bfseries, title = #1,
%after title={\hfill\colorbox{black}{Definition}}
}



\newtcolorbox{Masseinheit}[1]{
enhanced,
boxrule=1pt,colframe=blue,
colback=white,
sharp corners, 
colframe=blue!75!black,
title = #1, 
after title={\hfill\colorbox{blue}{Masseinheit}}
}


\newtcolorbox{myLösung}{
  enhanced,
  boxrule=1pt,
  colframe=gray!75!black, % Definiert die Farbe des Rahmens als dunkelgrau
  colback=gray!20, % Definiert die Hintergrundfarbe der Box als hellgrau
  %sharp corners, % Macht die Ecken der Box scharf (nicht abgerundet)
  title = {Lösung}, % Fest eingestellter Titel der Box
  after title={}, % Fügt das Label "Masseinheit" nach dem Titel hinzu
  coltitle=white, % Farbe des Titeltexts
  fonttitle=\bfseries % Schriftart des Titels
}





% Set the title, author, and date
\title{\textbf{Lektionsprogramm HTa-26}}
\author{Patrick Pfändler}
\date{6. Januar 2025}


\begin{document}

\frame{\titlepage}

\begin{frame}{Inhalt der Lektion}
	\tableofcontents
\end{frame}

%\pruefung{Abdichtungen, mit Bindemitteln gefestigte Baustoffe, Kunststoffe}{45}{https://www.classtime.com/code/RAWHBA}

\section{Metalle}
\BlueSectionSlide
\begin{frame}{Start mit Metallen}
\end{frame}

\subsection{Rückblick Zugversuch}
\begin{frame}{Rückblick: Zugversuch}
	\centering
	\includegraphics[height=\textheight]{../../Unterlagen/09_Metalle/Bilder/Zugversuch.jpeg}
\end{frame}


\begin{frame}{Zugversuch}
	\begin{Fragenblock}
		\begin{itemize}
			\item Was lässt sich alles aus einem Zugversuch ableiten?
			\item Welche Kennwerte lassen sich aus einem Zugversuch ableiten?
			\item Welche Einheiten haben die Kennwerte?
			\item Was ist die Dehnung?
			\item Was ist die Spannung?
			\item Was ist die Streckgrenze?
			\item Was ist die Zugfestigkeit?
			\item Was ist das Elastizitätsmodul?
			\item Was ist die Bruchdehnung?
		\end{itemize}
	\end{Fragenblock}
\end{frame}





%\section{Postenlauf:Innovation im Bauwesen}
\begin{frame}{Postenlauf: Innovation im Bauwesen}
    ca. 90 min Länge, Bearbeitung in der Lektion (keine SLE resp. Hausaufgabe)
    \begin{itemize}
        \item [\textbullet] Wähle 3 der 5 Texte aus. Die Texte sind auf Teams in den Ordner \textit{Innovation im Bauwesen} hochgeladen. Arbeitsform: Einzelarbeit.
        \begin{itemize}
            \item [\textbullet] Erstelle zu einem Text ein Mindmap. 
            \item [\textbullet] Erstelle zu einem Text eine schriftliche Zusammenfassung von ca. 150 Wörtern.
            \item [\textbullet] Erstelle zu einem Text mindestens 5 Kontrollfragen (inkl. Lösung) für deinen Tischnachbarn.
        \end{itemize}
        \item [\textbullet] Die Dokumente müssen in Teams hochgeladen  werden in den jeweiligen Aufgaben und werden mit bestanden / nicht bestanden bewertet.
    \end{itemize}
    \end{frame}

\begin{frame}{Übergeordnete Lernziele}
    Die übergeordnete Lernziele für diese Unterrichtseinheit sind: 
    \begin{itemize}
        \item Bauführer und Bauführerinnen informieren sich über neue Methoden und Technologien und den Einsatz von
        multifunktionalen und intelligenten Baustoffen in ihrem Arbeitsbereich.
        \begin{itemize}
            \item Sie informieren sich aus Fachpresse und Messen über Innovationen. \textbf{(K2)}
            \item Sie betreiben ein firmeninternes Wissensmanagement zukunftsorientiert. \textbf{(K4)}
            \item Sie erarbeiten Dokumentationen zur Einführung von kreislauffähigen Materialien und Baumethoden
            in ihrem Bereich. \textbf{(K4)}
        \end{itemize}
        \item Bauführer und Bauführerinnen leiten den fachgerechten und vorschriftsmässigen Einsatz neuer Methoden, Technologien und Baustoffe.
        \begin{itemize}
            \item Sie wenden neue Methoden, Technologien und Baustoffe bei Bauarbeiten an. \textbf{(K3)}
            \item Sie instruieren die Mitarbeitenden in neuen Bauabläufen. \textbf{(K3)}
            \item Sie führen Evaluationen zum Einsatz von neuen Baustoffen durch. \textbf{(K4)}
        \end{itemize}
    \end{itemize}
\end{frame}



\section{Organisatorisches}
\BlueSectionSlide

\subsection{Uploads auf Teams}
\begin{frame}{Uploads auf Teams}
	\begin{itemize}
		\item[\textbullet] keine
	\end{itemize}

\end{frame}

% \begin{frame}{Fragen zur Prüfung?}

% \end{frame}

\naechstePruefung{17.02.2024 }{Metalle, Wärmedämmstoffe, Innovation im Bauwesen}
\folieFragen

\end{document}