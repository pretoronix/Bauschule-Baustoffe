% !TEX TS-program = lualatex

% !TEX root = /Users/patricpf/Documents/repos/Bauschule-Baustoffe/HTa-26/Programme/KW_08_Programm.tex
% !TEX TS-program = lualatex

\def\customoptions{aspectratio=169} % Exclude handout
% !TEX program = lualatex
\documentclass[aspectratio=169, 
handout,
]{beamer}


\PassOptionsToPackage{dvipsnames,svgnames}{xcolor}

%\usepackage[utf8]{inputenc}
%\usepackage[ngerman]{babel} % Schweizer Rechtschreibung

\usepackage{polyglossia}
\setdefaultlanguage[variant = swiss]{german}
\usepackage{fontspec}
\setmainfont{Times New Roman}
\setsansfont{Arial}


\usepackage{amsmath}
\usepackage{siunitx}
\sisetup{
  locale = DE,
  inter-unit-product = \ensuremath{{\cdot}},
  detect-mode,             % Use the surrounding text font mode
  detect-family,           % Use the surrounding text font family
  detect-weight,           % Use the surrounding text font weight
  mode = text,             % Ensure that numbers and units are typeset in text mode
  %negative-powers = false, % Avoid using negative powers
  per-mode = symbol        % Use the division symbol for per units
}

\usepackage{tikz}
\usepackage{enumitem}
\usepackage{graphicx}
\usepackage{booktabs}
\usepackage{calc}
\usepackage{multicol}
\usepackage{amsmath}

\usepackage{tcolorbox}
\tcbuselibrary{skins}

\usepackage[dvipsnames]{xcolor}

%\usepackage[scaled]{helvet} % Arial-ähnliche Schriftart (Helvetica)
%\renewcommand\familydefault{\sfdefault} % Setzt die Standard-Schriftart auf sans-serif


\usetheme{Madrid} % This theme is visually appealing
%\usecolortheme{whale} % A color theme with blue tones
\usecolortheme{dolphin} % A color theme with blue tones


\newcommand{\mylogo}{
    \begin{tikzpicture}[remember picture,overlay]
        \node[anchor=north east, yshift=-2mm, fill=white, inner sep=2pt] at (current page.north east) % Verschiebe das Logo um 5mm nach unten
            {\includegraphics[height=0.4cm]{/Users/patricpf/Documents/repos/Bauschule-Baustoffe/template/bauschule-logo-5cm.png}}; % Grösse nach Bedarf anpassen
    \end{tikzpicture}
}

\logo{\mylogo}

\setbeamertemplate{headline}{%
  \begin{beamercolorbox}[wd=\textwidth,ht=0.5ex,dp=1ex]{upper separation line head}
  \end{beamercolorbox}
}

\setbeamercolor{upper separation line head}{bg=blau_bauschule}


% Anpassen des Frametitels, um ihn fett zu machen
\setbeamertemplate{frametitle}{%
    \nointerlineskip%
    \begin{beamercolorbox}[sep=0.3cm,left,wd=\paperwidth]{frametitle}%
        \usebeamerfont{frametitle}\bfseries\insertframetitle%
    \end{beamercolorbox}%
}

%\usebackgroundtemplate{
%    \includegraphics[width=\paperwidth,height=\paperheight]{my_pdf_copy_of_empty_ppt_template}
%}

% Setzen Sie hier den Namen des Fachs
\newcommand{\fachname}{Baustoffe}
\newcommand{\FinRes}[1]{\underline{\underline{#1}}}

% Anpassen der Fusszeile für Abstände zum Rand
\setbeamertemplate{footline}
{
  \leavevmode%
  \hbox{%
  \begin{beamercolorbox}[wd=.33\paperwidth,ht=2.25ex,dp=1ex,left,leftskip=1em]{author in head/foot}%
    \usebeamerfont{author in head/foot}\insertshortauthor
  \end{beamercolorbox}%
  \begin{beamercolorbox}[wd=.34\paperwidth,ht=2.25ex,dp=1ex,center]{title in head/foot}%
    \usebeamerfont{title in head/foot}\fachname % Hier wird der Fachname anstelle des Titels angezeigt
  \end{beamercolorbox}%
  \begin{beamercolorbox}[wd=.33\paperwidth,ht=2.25ex,dp=1ex,right,rightskip=1em]{date in head/foot}%
    \usebeamerfont{date in head/foot}\insertframenumber{} / \inserttotalframenumber\hspace*{2ex}
  \end{beamercolorbox}}%
  \vskip0pt%
}


\newtcolorbox{Merke}{
enhanced,
boxrule=0pt,frame hidden,
borderline west={4pt}{0pt}{red!75!black},
colback=white,
sharp corners,
before upper={\textbf{Merke:}\quad},
}

\newtcolorbox{Anwendungen}{
enhanced,
boxrule=0pt,frame hidden,
borderline west={4pt}{0pt}{brown!75!black},
colback=white,
sharp corners
}




% Colors
\definecolor{blau_bauschule}{RGB}{22,65,148}
\setbeamercolor{frametitle}{fg=blau_bauschule}
\setbeamertemplate{navigation symbols}{} % Remove navigation symbols


\newtcolorbox{Definition_BS}[1]{
enhanced,boxrule=1pt,
colback=green!5!white,
colframe=green!75!black,fonttitle=\bfseries, title = #1,
%after title={\hfill\colorbox{black}{Definition}}
}



\newtcolorbox{Masseinheit}[1]{
enhanced,
boxrule=1pt,colframe=blue,
colback=white,
sharp corners, 
colframe=blue!75!black,
title = #1, 
after title={\hfill\colorbox{blue}{Masseinheit}}
}


\newtcolorbox{myLösung}{
  enhanced,
  boxrule=1pt,
  colframe=gray!75!black, % Definiert die Farbe des Rahmens als dunkelgrau
  colback=gray!20, % Definiert die Hintergrundfarbe der Box als hellgrau
  %sharp corners, % Macht die Ecken der Box scharf (nicht abgerundet)
  title = {Lösung}, % Fest eingestellter Titel der Box
  after title={}, % Fügt das Label "Masseinheit" nach dem Titel hinzu
  coltitle=white, % Farbe des Titeltexts
  fonttitle=\bfseries % Schriftart des Titels
}




% Set the title, author, and date
\title{\textbf{Lektionsprogramm HTa-26}}
\author{Patrick Pfändler}
\date{17. Februar 2025}


\begin{document}
\frame{\titlepage}
% \begin{frame}{Zeitplan für heute}
% 	\begin{itemize}
% 		\item[\textbullet] 11:15 - 12:15: \textcolor{red}{Prüfung}; anschliessend ca. 10 min Pause
% 		\item[\textbullet] 12:15 - 12:45: Abschluss Innovation im Bauwesen (selbständig, obligatorisch)
% 		\item[\textbullet] 12:45 - 13:30: \textcolor{blue}{Mittagspause}
% 		\item[\textbullet] 13:30 - 15:15: Holz- und Holzwerkstoffe
% 	\end{itemize}
% \end{frame}


\begin{frame}{Inhalt der Lektion}
	\tableofcontents
\end{frame}

\subsection{Uploads auf Teams}
\begin{frame}{Uploads auf Teams}
	\begin{itemize}
		\item[\textbullet] Programm KW 08
		\item[\textbullet] Blatt: Benennung von Holz-und Holzwerkstoffen
	\end{itemize}

\end{frame}

\begin{frame}{Nachprüfungen}
	\begin{itemize}
		\item Keine!
	\end{itemize}

\end{frame}



\section{Nachbesprechung der Prüfung}
\BlueSectionSlide

\notenformel{4}{44}{49}
\notenverteilung{3.20}{5.20}{4.74}{5.50}{0.80}

\begin{frame}{Nachbesprechung der Fragen}

\end{frame}

\begin{frame}{Dämmung Innen vs. Aussen
	}
	\begin{Fragenblock}
			
	Verglichen mit dem U-Wert einer innenseitig gedämmten Wand ist der U-Wert der aussenseitig
	gedämmten Wand gleich. (Annahme: selbes Dämmmaterial und selbe Dämmmaterialdicke)
	
	\begin{equation*}
		U  = \frac{1}{R_{I} + \dfrac{d}{\lambda} + R_{}}
	\end{equation*}
	
	\end{Fragenblock}
	\pause
	\vspace{1cm}

	$\Rightarrow$ Gleich Gross, die Position der Dämmung geht nicht in die Berechnung des U-Wertes ein.

\end{frame}

\begin{frame}{Material erkennen}

	\begin{Fragenblock}
		Nenne die Abkürzung des Materials aus dem Periodensystem:

Hinweis: Es handelt sich um eine Knetlegierung.
	\end{Fragenblock}
	\pause
	\vspace{1cm}
	$\Rightarrow$ Knetlegierung $ \rightarrow $ Al

\end{frame}

\begin{frame}{Herstellung von Stahl}
	\begin{Fragenblock}
		Welche Hauptverfahren werden zur Stahlherstellung verwendet?
	\end{Fragenblock}
	\pause
	\vspace{1cm}
	$\Rightarrow$ LD-Verfahren, Elektrostahlverfahren

\end{frame}


\begin{frame}{Bezeichnung von Metallen}
	\begin{Fragenblock}
		Was können Sie alles aus dieser Bezeichnung ableiten: 

		EN-GJS-500-14
		
		Gehen Sie auf die einzelnen Bestandteile ein und wo möglich mit Einheit.

	\end{Fragenblock}
	\pause
	\vspace{1cm}
	$\Rightarrow$ EN = Europäische Norm, GJS = Gusseisen mit Kugelgraphit, 500 = Zugfestigkeit in MPa, 14 = Bruchdehnung in \%

\end{frame}


\begin{frame}{Berechnung der Zugfestigkeit des Stahls}
	\begin{itemize}
		\item [\textbullet] $F_{\text{max}} = 125 \, \text{kN}$
		\item [\textbullet] Durchmesser $d = 16 \, \text{mm}$
		\item [\textbullet] $A = \frac{\pi}{4} \cdot d^2$
		\item [\textbullet] $\sigma = \frac{F_{\text{max}}}{A}$
		\item [\textbullet] $\sigma =  \frac{125 \, \text{kN}}{\frac{\pi}{4} \cdot (16 \, \text{mm})^2}$
		\item [\textbullet] $\sigma =  \frac{125 \, \text{kN}}{\frac{\pi}{4} \cdot 256 \, \text{mm}^2}$ 
		\item [\textbullet] $\sigma =  \frac{125 \, \text{kN}}{201.06 \, \text{mm}^2} = 621.8 \, \text{N/mm}^2 = 621.8 \, \text{MPa}$
	\end{itemize}

\end{frame}


\begin{frame}{Was ist die Streckgrenze dieses Metalls?}
	\begin{Fragenblock}
		Bitte Einheit angeben.

		S235J0
	\end{Fragenblock}

	\pause
	\vspace{1cm}
	\textcolor{red}{Bitte Einheit angeben.}
	$\Rightarrow$ 235 N/mm$^2$ resp. 235 MPa

\end{frame}


\begin{frame}{Einsatz von Innovativen im Bauwesen
	}
	\begin{Fragenblock}
		Welche innovativen Lösungen könnten Sie für ein zukünftiges Projekt in Ihrem bisherigen oder neuen Betrieb vorschlagen? Beziehen Sie sich dabei auf die Inhalte der gelesenen Texte. Argumentieren Sie mit Vor- und Nachteilen der jeweiligen Methode, um die Lösung überzeugend zu präsentieren. Schreiben Sie etwa 3 bis 4 prägnante Sätze mit dem Ziel, Ihren Vorgesetzten von der vorgeschlagenen Innovation zu überzeugen.
	\end{Fragenblock}
	\pause
	\vspace{1cm}
	$\Rightarrow$  \textcolor{red}{Beziehen Sie sich dabei auf die Inhalte der gelesenen Texte.}

\end{frame}

\begin{frame}{Fragen zur Prüfung?}
	\begin{itemize}
		\item[\textbullet] Fragen zur Prüfung?
		\item[\textbullet] Feedback zur Prüfung?
	\end{itemize}

\end{frame}


\subsection{Noten}
\begin{frame}{Noten}
	\begin{itemize}
		\item[\textbullet] Anschauen der Noten bei Mir.
	\end{itemize}
\end{frame}



%\pruefung{Metalle, Dämmstoffe, Innovation im Bauwesen}{50}{https://www.classtime.com/code/W8CX6D}
%\pruefung{Abdichtungen, mit Bindemitteln gefestigte Baustoffe, Kunststoffe}{45}{https://www.classtime.com/code/RAWHBA}
%\section{Postenlauf:Innovation im Bauwesen}
\begin{frame}{Postenlauf: Innovation im Bauwesen}
    ca. 90 min Länge, Bearbeitung in der Lektion (keine SLE resp. Hausaufgabe)
    \begin{itemize}
        \item [\textbullet] Wähle 3 der 5 Texte aus. Die Texte sind auf Teams in den Ordner \textit{Innovation im Bauwesen} hochgeladen. Arbeitsform: Einzelarbeit.
        \begin{itemize}
            \item [\textbullet] Erstelle zu einem Text ein Mindmap. 
            \item [\textbullet] Erstelle zu einem Text eine schriftliche Zusammenfassung von ca. 150 Wörtern.
            \item [\textbullet] Erstelle zu einem Text mindestens 5 Kontrollfragen (inkl. Lösung) für deinen Tischnachbarn.
        \end{itemize}
        \item [\textbullet] Die Dokumente müssen in Teams hochgeladen  werden in den jeweiligen Aufgaben und werden mit bestanden / nicht bestanden bewertet.
    \end{itemize}
    \end{frame}

\begin{frame}{Übergeordnete Lernziele}
    Die übergeordnete Lernziele für diese Unterrichtseinheit sind: 
    \begin{itemize}
        \item Bauführer und Bauführerinnen informieren sich über neue Methoden und Technologien und den Einsatz von
        multifunktionalen und intelligenten Baustoffen in ihrem Arbeitsbereich.
        \begin{itemize}
            \item Sie informieren sich aus Fachpresse und Messen über Innovationen. \textbf{(K2)}
            \item Sie betreiben ein firmeninternes Wissensmanagement zukunftsorientiert. \textbf{(K4)}
            \item Sie erarbeiten Dokumentationen zur Einführung von kreislauffähigen Materialien und Baumethoden
            in ihrem Bereich. \textbf{(K4)}
        \end{itemize}
        \item Bauführer und Bauführerinnen leiten den fachgerechten und vorschriftsmässigen Einsatz neuer Methoden, Technologien und Baustoffe.
        \begin{itemize}
            \item Sie wenden neue Methoden, Technologien und Baustoffe bei Bauarbeiten an. \textbf{(K3)}
            \item Sie instruieren die Mitarbeitenden in neuen Bauabläufen. \textbf{(K3)}
            \item Sie führen Evaluationen zum Einsatz von neuen Baustoffen durch. \textbf{(K4)}
        \end{itemize}
    \end{itemize}
\end{frame}

%\section{Carolabrücke}
\BlueSectionSlide

\begin{frame}{Carolabrücke Reminder}
    \begin{itemize}
        \item \href{https://www.youtube.com/watch?v=8PvhMLrxSYA}{Video zur Carolabrücke}
    \end{itemize}
\end{frame}


\begin{frame}{Vorläufige Erkenntnisse zur Ursache und Hergang des Teileinsturzes der Carolabrücke}
    \begin{block}{Zwischenergebnisse}
        Die vorliegenden Zwischenergebnisse deuten darauf hin, dass wasserstoffinduzierte Spannungsrisskorrosion die Hauptursache für das Versagen ist.
        \footnote{\href{        \footnote{https://www.dresden.de/media/pdf/presseamt/2024_12_11_Carolabruecke_Zusammenfassung-Ergebnisse.pdf}}{www.dresden.de}}
    \end{block}

\end{frame}

\begin{frame}{Wasserstoffinduzierte Spannungsrisskorrosion (SCC)}
    \begin{Definition_BS}{Spannungsrisskorrosion}
        Wasserstoffinduzierte Spannungsrisskorrosion (SCC) ist ein Schadensmechanismus, der unter spezifischen Voraussetzungen in
        hochbelasteten Metallen, wie vergütetem Spannstahl (u.\,a. auch Hennigsdorfer Spannstahl), auftreten kann. Dabei diffundiert Wasserstoff in die innere Gefügestruktur und führt dort unter anhaltender mechanischer Spannung zu Mikrorissen, die sich fortschreitend ausbreiten und
        schlieddlich zum spröden Versagen des Stahls führen können. Mangelnder Schutz vor Feuchtigkeit, korrosive Umgebung oder
        Verarbeitungsfehler begünstigen diesen Prozess.
        \footnote{\href{        \footnote{https://www.dresden.de/media/pdf/presseamt/2024_12_11_Carolabruecke_Zusammenfassung-Ergebnisse.pdf}}{www.dresden.de}}
    \end{Definition_BS}
\end{frame}


\begin{frame}{Wichtigste Konsequenzen}
    \begin{block}{}
        \begin{itemize}
            \item[\textbullet] Die \textbf{gesamte }Brücke wird für den Verkehr gesperrt und muss abgerissen werden.
            \item[\textbullet] Wasserstrasse darf nur nach der Installation eines Schallemissionssystems wieder freigegeben werden.
        \end{itemize}
    \end{block}
    \pause
    \begin{block}{Verschulden}
        Eine umfassende Aktenlage belegt, dass
        das Bauwerk innerhalb der geltenden Regelwerke bewertet und betrieben wurde.
    \end{block}
\end{frame}


\begin{frame}{Wichtigste Erkenntnisse}
    \begin{itemize}
        \item [\textbf{→}] \textbf{Haupteinsturzursache:} Wasserstoffinduzierte Spannungsrisskorrosion
        \item [\textbf{→}] \textbf{Konsequenz:} Einsturz nicht vorhersagbar, da keine ausgeprägte Rissbildung.
        \item [\textbf{→}] \textbf{Schuldfrage:} Gesetzliche Vorgaben eingehalten, keine Versäumnisse.
        \item [\textbf{→}] \textbf{Spannstahldefekte:} Über 68 Prozent der Spannglieder in der Fahrbahnplatte von Zug C waren an der Bruchstelle stark geschädigt.
        \item [\textbf{→}] \textbf{Massnahmen:} Abriss der \textbf{gesamten} Brücke, temporäre Installation eines Schallemissionssystems. Bau einer neuen Brücke.
        \item [\textbf{→}] \textbf{Tausalze:} Sogenannte chloridinduzierte Korrosion hat an Brückenzug C stattgefunden, war jedoch nicht ursächlich für den Einsturz.
    \end{itemize}
\end{frame}

\begin{frame}{Lernziele: Einsturz der Carolabrücke}
    \begin{myLernziele}
        \begin{itemize}
            \item[\textbullet] Kenntnisse über die Ursache des Teileinsturzes der Carolabrücke
            \item[\textbullet] Verständnis für die Konsequenzen des Einsturzes
            \item[\textbullet] Wissen über die wichtigsten Erkenntnisse aus dem Zwischenbericht
        \end{itemize}
    \end{myLernziele}
\end{frame}


\begin{frame}{Wie würden Sie die Brücke abbrechen?}
    \begin{itemize}
        \item Diskussion mit Nachbarn
        \item Ziel: Kurze Skizze oder ähnliches und Überlegungen darlegen; Darf auch nur Teile der Brücke betreffen.
        \item Zeit: ca. 15 Minuten
    \end{itemize}
\end{frame}

\begin{frame}{Video zu den Abbrucharbeiten}
    \begin{itemize}
        \item \href{https://www.youtube.com/watch?v=_FJjkKMLzMc}{Video zu den Abbrucharbeiten}
    \end{itemize}
\end{frame}





\section{Holz- und Holzwerkstoffe}
\BlueSectionSlide
\folieFragen

\begin{frame}{Repetition: Schnittebenen im Holz}

	\begin{itemize}
		\item [\textbullet] Radialschnitt
		\item [\textbullet] Tangentialschnitt
		\item [\textbullet] Querschnitt
	\end{itemize}
\end{frame}

\begin{frame}{Verwendete Baumaterialien}

	\begin{itemize}
		\item [\textbullet] Bretter
		\item [\textbullet] Latten und Leisten
		\item [\textbullet] Kanthölzer
		\item [\textbullet] Konstruktionsvollholz
	\end{itemize}

\end{frame}

\begin{frame}{Repetition: Holzeigenschaften}
	
	\begin{itemize}
		\item [\textbullet] Wassergehalt $\Rightarrow$  natürliche und künstliche Trocknung
		\item [\textbullet] Quellen und Schwinden ($\Rightarrow$ Faserrichtung, Volumenänderung)
		\item [\textbullet] Dichte
		\item [\textbullet] Feuchtigkeit
	\end{itemize}
\end{frame}

\begin{frame}{Repertition: Verformung von Brettern bei der Trocknung}
	\begin{itemize}
		\item [\textbullet] Krümmung
		\item [\textbullet] Verwölbung
		\item [\textbullet] Verwindung
	\end{itemize}

\end{frame}

\begin{frame}{Weiter ab Folie 39}

\end{frame}






\section{Ausblick}
\BlueSectionSlide
\naechstePruefung{im Herbst}{Innovation im Bauwesen und Holz-und Holzwerkstoffe}

\begin{frame}{Ausblick auf nächste Woche}
	\begin{itemize}
		\item[\textbullet] Feedbackrunde
		\item[\textbullet] Rückblick auf die bisherigen Lektionen im Fach Baustoffe
	\end{itemize}
\end{frame}




\end{document}