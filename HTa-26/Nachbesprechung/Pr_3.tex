% !TEX root = /Users/patricpf/Documents/repos/Bauschule-Baustoffe/HTa-26/Nachbesprechung/Pr_3.tex
\def\customoptions{aspectratio=169} % Exclude handout
% !TEX program = lualatex
\documentclass[aspectratio=169, 
handout,
]{beamer}


\PassOptionsToPackage{dvipsnames,svgnames}{xcolor}

%\usepackage[utf8]{inputenc}
%\usepackage[ngerman]{babel} % Schweizer Rechtschreibung

\usepackage{polyglossia}
\setdefaultlanguage[variant = swiss]{german}
\usepackage{fontspec}
\setmainfont{Times New Roman}
\setsansfont{Arial}


\usepackage{amsmath}
\usepackage{siunitx}
\sisetup{
  locale = DE,
  inter-unit-product = \ensuremath{{\cdot}},
  detect-mode,             % Use the surrounding text font mode
  detect-family,           % Use the surrounding text font family
  detect-weight,           % Use the surrounding text font weight
  mode = text,             % Ensure that numbers and units are typeset in text mode
  %negative-powers = false, % Avoid using negative powers
  per-mode = symbol        % Use the division symbol for per units
}

\usepackage{tikz}
\usepackage{enumitem}
\usepackage{graphicx}
\usepackage{booktabs}
\usepackage{calc}
\usepackage{multicol}
\usepackage{amsmath}

\usepackage{tcolorbox}
\tcbuselibrary{skins}

\usepackage[dvipsnames]{xcolor}

%\usepackage[scaled]{helvet} % Arial-ähnliche Schriftart (Helvetica)
%\renewcommand\familydefault{\sfdefault} % Setzt die Standard-Schriftart auf sans-serif


\usetheme{Madrid} % This theme is visually appealing
%\usecolortheme{whale} % A color theme with blue tones
\usecolortheme{dolphin} % A color theme with blue tones


\newcommand{\mylogo}{
    \begin{tikzpicture}[remember picture,overlay]
        \node[anchor=north east, yshift=-2mm, fill=white, inner sep=2pt] at (current page.north east) % Verschiebe das Logo um 5mm nach unten
            {\includegraphics[height=0.4cm]{/Users/patricpf/Documents/repos/Bauschule-Baustoffe/template/bauschule-logo-5cm.png}}; % Grösse nach Bedarf anpassen
    \end{tikzpicture}
}

\logo{\mylogo}

\setbeamertemplate{headline}{%
  \begin{beamercolorbox}[wd=\textwidth,ht=0.5ex,dp=1ex]{upper separation line head}
  \end{beamercolorbox}
}

\setbeamercolor{upper separation line head}{bg=blau_bauschule}


% Anpassen des Frametitels, um ihn fett zu machen
\setbeamertemplate{frametitle}{%
    \nointerlineskip%
    \begin{beamercolorbox}[sep=0.3cm,left,wd=\paperwidth]{frametitle}%
        \usebeamerfont{frametitle}\bfseries\insertframetitle%
    \end{beamercolorbox}%
}

%\usebackgroundtemplate{
%    \includegraphics[width=\paperwidth,height=\paperheight]{my_pdf_copy_of_empty_ppt_template}
%}

% Setzen Sie hier den Namen des Fachs
\newcommand{\fachname}{Baustoffe}
\newcommand{\FinRes}[1]{\underline{\underline{#1}}}

% Anpassen der Fusszeile für Abstände zum Rand
\setbeamertemplate{footline}
{
  \leavevmode%
  \hbox{%
  \begin{beamercolorbox}[wd=.33\paperwidth,ht=2.25ex,dp=1ex,left,leftskip=1em]{author in head/foot}%
    \usebeamerfont{author in head/foot}\insertshortauthor
  \end{beamercolorbox}%
  \begin{beamercolorbox}[wd=.34\paperwidth,ht=2.25ex,dp=1ex,center]{title in head/foot}%
    \usebeamerfont{title in head/foot}\fachname % Hier wird der Fachname anstelle des Titels angezeigt
  \end{beamercolorbox}%
  \begin{beamercolorbox}[wd=.33\paperwidth,ht=2.25ex,dp=1ex,right,rightskip=1em]{date in head/foot}%
    \usebeamerfont{date in head/foot}\insertframenumber{} / \inserttotalframenumber\hspace*{2ex}
  \end{beamercolorbox}}%
  \vskip0pt%
}


\newtcolorbox{Merke}{
enhanced,
boxrule=0pt,frame hidden,
borderline west={4pt}{0pt}{red!75!black},
colback=white,
sharp corners,
before upper={\textbf{Merke:}\quad},
}

\newtcolorbox{Anwendungen}{
enhanced,
boxrule=0pt,frame hidden,
borderline west={4pt}{0pt}{brown!75!black},
colback=white,
sharp corners
}




% Colors
\definecolor{blau_bauschule}{RGB}{22,65,148}
\setbeamercolor{frametitle}{fg=blau_bauschule}
\setbeamertemplate{navigation symbols}{} % Remove navigation symbols


\newtcolorbox{Definition_BS}[1]{
enhanced,boxrule=1pt,
colback=green!5!white,
colframe=green!75!black,fonttitle=\bfseries, title = #1,
%after title={\hfill\colorbox{black}{Definition}}
}



\newtcolorbox{Masseinheit}[1]{
enhanced,
boxrule=1pt,colframe=blue,
colback=white,
sharp corners, 
colframe=blue!75!black,
title = #1, 
after title={\hfill\colorbox{blue}{Masseinheit}}
}


\newtcolorbox{myLösung}{
  enhanced,
  boxrule=1pt,
  colframe=gray!75!black, % Definiert die Farbe des Rahmens als dunkelgrau
  colback=gray!20, % Definiert die Hintergrundfarbe der Box als hellgrau
  %sharp corners, % Macht die Ecken der Box scharf (nicht abgerundet)
  title = {Lösung}, % Fest eingestellter Titel der Box
  after title={}, % Fügt das Label "Masseinheit" nach dem Titel hinzu
  coltitle=white, % Farbe des Titeltexts
  fonttitle=\bfseries % Schriftart des Titels
}


\usepackage{fontawesome}  % For check and cross symbols


% Set the title, author, and date
\title{\textbf{Nachbesprechung Prüfung: HTa-26: Abdichtungen, Bindemittel gefestigt, Kunststoffe}
}
\author{Patrick Pfändler}

\begin{document}
%Farben 
\definecolor{orangish}{wave}{620}


%Start of the slides
\frame{\titlepage}

\begin{frame}{Inhalsverzeichnis}
    \tableofcontents
\end{frame}


\section{Notenübersicht}
\begin{frame}{Notenverteilung}
    \begin{table}[h!]
        \centering
        %\caption{Statistische Kennwerte der Messdaten}
        \label{tab:statistische_werte}
        \begin{tabular}{ll}
            \hline
            \textbf{{}} & \textbf{Note} \\ \hline
            Minimum & 3.9 \\
            Median & 5.2 \\
            Mittelwert & 4.88 \\
            Maximum & 5.60 \\
            Standardabweichung (Std) & 0.65 \\ \hline
        \end{tabular}
    \end{table}
\end{frame}

\begin{frame}{Formel für die Notenberechnung}
    \begin{itemize}
        \item[\textbullet] Note 1 mit 5 Punkten
        \item[\textbullet] Note 6 mit 37.5 (von 39.84) Punkten  
    \end{itemize}
\end{frame}

\section{Nachbesprechung der Prüfung}

\begin{frame}{Frage 1: Mit Bindemitteln gefestigt}
    Künstlich geformte Mauersteine, Platten sowie tragende und nichttragende Fertig-bauteile aus Bindemitteln,
Füllstoffen, Wasser und ev. Bewehrung, mit tragender, raumabschliessender oder bekleidender
Funktion sind mit ... gefestigt.

\pause
    \vspace{\baselineskip} 

    \textbf{Bindemitteln gefestigte Baustoffe}

    
\end{frame}

\begin{frame}{Frage 3: Defintion 2}

    Uberzüge als Imitation auf Betonunterlage (Treppen und Bodenbeläge) an Ort und Stelle gegossen und
Terrazzo
geschliffen bezeichnet man als ... .

\pause
    \vspace{\baselineskip} 

    \textbf{Terrazzo}
\end{frame}

\begin{frame}
    \begin{block}{Begriffe}
        Bitte Unterscheiden zwischen Bezeichnung eines Baustoffes und dessen Verkaufsname.
    
        \vspace{\baselineskip}
    
        \textbf{Beispiel:}
        \begin{itemize}
            \item \textbf{Plexiglas:} PMMA
            \item \textbf{Ytong:} Porenbeton
        \end{itemize}
    \end{block}
\end{frame}

\begin{frame}{Frage 8: Zusammensetzung von Materialien}
    
    \begin{block}{Welches Material?}
        Zement,
Gesteinskörnung
en mit niedriger
Dichte wie
Blähton (Liapor),
gewaschener
Bims, PS,
Ziegelschrot,
Sand, Wasser
    \end{block}

\pause
    \vspace{\baselineskip} 

    \textbf{Leichtbetonsteine}

\end{frame}


\begin{frame}{Frage 8: Zusammensetzung von Materialien}
    \begin{block}{Welches Material}
        Quarzsand,
Zement, Kalk,
Treibmittel
(Aluminiumpulve
r, Wasser, ev.
Bewehrung)
\end{block}

\pause
    \vspace{\baselineskip} 

    \textbf{Porenbetonsteine}

\end{frame}

\begin{frame}{Frage 11: Um welche Abdichtung handelt es sich?}
   gleiche Frage wie im Quiz!
\end{frame}

\begin{frame}{Art der Fugen und Unterteilung von Kunststoffe (F21)}
    Erwartung war, dass ihr diese ohne Nachschauen könnt.

\end{frame}

\begin{frame}{Frage 13: Abdichtungsbahn}

    \begin{block}{Bild mit folgendem Text:}
        \begin{itemize}
            \item EVA 35
        \end{itemize}
    \end{block}

\begin{block}{Was ist falsch?}
    \pause
    Der Punkt fehlt.
\end{block}
\end{frame}

\begin{frame}{Frage 14: Dichtigkeitsklassen}
    \begin{block}{Zugangsstollen}
        \pause
        DK 4
    \end{block}

\end{frame}

\begin{frame}{Abdichtung mit Schlauch}
    Frage genau lesen: 

    Welche \textbf{Art von Fuge} dichtet dieser Schlauch ab?
    \pause
    \vspace{\baselineskip}
    \textbf{Arbeitsfuge}

\end{frame}

\begin{frame}{Löslichkeit von Plastik}
    \begin{block}{Plastik ist nicht löslich in Wasser.}
        Wahr oder Falsch?
    \end{block}

    \pause
    \vspace{\baselineskip}
    \textbf{Richtig}

\end{frame}

\begin{frame}{Name der Dichtungsbahne}
    \begin{block}{EGV 3.5 flam/flam}
        Ich hatte euch vor ChatGPT gewarnt\dots
        
    \end{block}

\end{frame}

\begin{frame}{Biostabilisatoren}
    Ich werde es nochmals anschauen! Sorry.

\end{frame}

\begin{frame}{Fragen zur Prüfung?}

\end{frame}

\begin{frame}{Rückmeldungen}

\end{frame}



\end{document}